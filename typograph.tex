%\chapter*{}
%\newpage
{
\thispagestyle{empty}
%
\small{
\begin{flushleft}
\textbf{%
	УДК \UDK \\
	ББК \BBK \\
	\BibCode \\
}
\end{flushleft}
%
%\vspace{3cm}
\vspace{1cm}
%
\begin{flushright}
{
\begin{tabular}[c]{>{\raggedright}m{14mm} >{\raggedright}m{95mm} }
	\textbf{\BibCode} & \MyVarAuthorName \tabularnewline
	~ & \MyVarBookName \tabularnewline
	~ & \MyVarBookNamesec \tabularnewline
	~ & М.:~ООО~<<ИПЦ~"`Маска"'>>,~\year\mdash \pageref{LastPage} c. \tabularnewline	
	~ & \textbf{\ISBN} \tabularnewline
	~ & \ciao{16+}
\end{tabular}
}
\end{flushright}
%
%\vspace{3.0cm}%\vspace{4.0cm}
\vspace{0.5cm}
\hspace{1.0cm} В повести рассказывается о байдарочном сплаве в~Карелии по маршруту <<Сунская цепочка>>\cite{Шилов}, а также о событиях, предшествующих этому походу. Своеобразный <<кольцевой>> маршрут, замкнутый на посёлок Гирвас, состоит из двух частей: в первой путешественники побывают на череде озёр и небольших речушек, а~во~второй, начинающейся с~Линдозера, предстоит штурм 14 порогов 2~категории сложности. Цепочка сменяющих друг друга рек и озёр приносит сплавщикам массу впечатлений, приправленных переменчивой карельской погодой, а~также местными ягодами, грибами и, конечно же, свежевыловленной рыбой. При заброске путешественники осмотрят крепость в~Старой Ладоге, побывают на водопаде Кивач, а~при~выброске\mdash на~палеовулкане Гирвас.

\vspace{4mm}
\noindent Ключевые слова: водный туризм, речной сплав, туристкий поход, байдарочный поход, Карелия, путевые заметки, дневник похода
%
%\begin{flushright}
%\textbf{%
%	УДК \UDK \\
%	ББК \BBK \\
%	\BibCode \\
%}
%\end{flushright}
%
%\vspace{\fill}
\vspace{4mm}
%\vspace{1.0cm}
%

\noindent\makebox[\textwidth][s]{\textbf{\ISBN}\hfill{\copyright~\MyVarAuthorName,~\year}}
}

%{
%\begin{longtable}[c]{>{\raggedright}m{56mm} >{\raggedleft}m{56mm} }
%	\textbf{\ISBN} & {\copyright~\MyVarAuthorName,~\year} \tabularnewline
%\end{longtable}
%}
}