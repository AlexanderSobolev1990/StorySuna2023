\chapter{Два отрывистых}
%\corner{64}
\vepsianrose

C утра поднялись относительно неплохо, хоть сон и был почти что прерывистый\mdash то и дело кто нибудь, пошатываясь в ночи брёл в уборную. Ну да ладно. С утра позавтракали печеньками, колбаской, залили чаю в термосы\mdash следующая горячая еда будет у нас только вечером. 

Снаряжение, перепакованное вчера, покидали в~машины уже как нибудь, не особо распихивая\mdash все равно сейчас будем перекладывать всё в другое авто, которое закинет нас на стапель. Созвонились с Олегом, классным мужичком в Гирвасе, который уже более 30 лет тут занимается заброской туристов на сплавы\mdash у~него двор большой, хозяйство, козлятки бегают, куры, во дворе поленница красиво сложенная, печь для копчения сала и мяса. Одним словом\mdash класс! Дядька оказался что надо. Мы быстро всё обговорили, расплатились за стоянку наших машин вперёд, а потом приобрели кусок копчёного сала и копчёный рулет тоже из сала. Кра\sdash со\sdash тень! 

Быстро перекидали вещи в прицеп, а сами все влезли в видавший 7\sdash местный паджерик. Свои авто с Серёгой загнали во двор к Олегу, поплотнее припарковались, а я ещё и скинул клемму с аккума на всякий случай\mdash мало ли мы тогда с Серёгой не до конца устранили все утечки тока. Всё это заняло не больше 15\ndash 20 минут.

И вот, наконец, старт! Было утро, часов 10 наверно. Мы повернули направо на старую дорогу Гирвас\mdash Петрозаводск, которая была достаточно приличной. Далее, через километров 30, свернули направо на Нёлгомозеро, как было написано на указателе. Дорога практически сразу стала гравийной, впрочем, достаточно в хорошем состоянии. Видно было по обочине, что её равняли грейдером совсем недавно, даже, может, в этом месяце.

Если до этого местность была с переменным успехом равнинной, то сейчас пошли холмы, спуски и подъёмы, крутые повороты между холмами. Пейзаж значительно стал меняться по мере нашего приближения к Нёлгомозеру. В самом посёлке Нёлгомозеро есть связь\mdash все из спортивного интереса стали проверять свои мобильники. Только у Кири появился даже интернет и он немного успел загрузить описание маршрута, так, почитать. 

Вскоре ещё спустя полчаса наверно, мы подъехали к крутому левому повороту, где направо вниз с холма уходила грунтовка. Олег сказал, что это наше место, приехали. Это согласовывалось с GPS и мы стали думать как нам выгружать вещи\mdash в итоге Олег смог заехать на грунтовку в лес с прицепом и развернуться на хорошей большой поляне, которая, судя по следам цивилизации, часто используется под стапель.

Вскоре ещё спустя полчаса наверно, мы подъехали к крутому левому повороту, где направо вниз с холма уходила грунтовка. Олег сказал, что это наше место, приехали. Это согласовывалось с GPS и мы стали думать как нам выгружать вещи\mdash в итоге Олег смог заехать на грунтовку в лес с прицепом и развернуться на хорошей большой поляне, которая, судя по следам цивилизации, часто используется под стапель.

Вскоре ещё спустя полчаса наверно, мы подъехали к крутому левому повороту, где направо вниз с холма уходила грунтовка. Олег сказал, что это наше место, приехали. Это согласовывалось с GPS и мы стали думать как нам выгружать вещи\mdash в итоге Олег смог заехать на грунтовку в лес с прицепом и развернуться на хорошей большой поляне, которая, судя по следам цивилизации, часто используется под стапель.

Вскоре ещё спустя полчаса наверно, мы подъехали к крутому левому повороту, где направо вниз с холма уходила грунтовка. Олег сказал, что это наше место, приехали. Это согласовывалось с GPS и мы стали думать как нам выгружать вещи\mdash в итоге Олег смог заехать на грунтовку в лес с прицепом и развернуться на хорошей большой поляне, которая, судя по следам цивилизации, часто используется под стапель.

Вскоре ещё спустя полчаса наверно, мы подъехали к крутому левому повороту, где направо вниз с холма уходила грунтовка. Олег сказал, что это наше место, приехали. Это согласовывалось с GPS и мы стали думать как нам выгружать вещи\mdash в итоге Олег смог заехать на грунтовку в лес с прицепом и развернуться на хорошей большой поляне, которая, судя по следам цивилизации, часто используется под стапель.

Вскоре ещё спустя полчаса наверно, мы подъехали к крутому левому повороту, где направо вниз с холма уходила грунтовка. Олег сказал, что это наше место, приехали. Это согласовывалось с GPS и мы стали думать как нам выгружать вещи\mdash в итоге Олег смог заехать на грунтовку в лес с прицепом и развернуться на хорошей большой поляне, которая, судя по следам цивилизации, часто используется под стапель.

Вскоре ещё спустя полчаса наверно, мы подъехали к крутому левому повороту, где направо вниз с холма уходила грунтовка. Олег сказал, что это наше место, приехали. Это согласовывалось с GPS и мы стали думать как нам выгружать вещи\mdash в итоге Олег смог заехать на грунтовку в лес с прицепом и развернуться на хорошей большой поляне, которая, судя по следам цивилизации, часто используется под стапель.



\begin{center}
	\psvectorian[scale=0.4]{88} % Красивый вензелёк :)
\end{center}
