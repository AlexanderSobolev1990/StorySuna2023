{
\chapter{В платье белом}
%\corner{64}
\vepsianrose

\fancyhead[LE]{\fancyplain{}{\bfseries \parttitle}}
\fancyhead[RO]{\fancyplain{}{\bfseries \rightmark}}

Зал был полон народу, Шурик наверно в первый раз видел Кирю в костюме\mdash похорошел, откормился. Невеста была, как и положено, в платье белом, <<как майская роза стройна и свежа>>. Веселье шло своим чередом, постепенно закручиваясь как бы по спирали с~увеличением градуса в крови. Тосты лились рекой, народ самоорганизовывался, поскольку Киря мудро не~позвал тамаду на сие действо. Серёга, их приятель, один среди <<своих>>, байдарочников, активно опрокидывал бокал за~бокалом, а Шурик, не~отставая, полвечера втихаря косился на кирину сестру:

\diagdash Шурик, у неё характер\mdash кабздец, не вздумай.\mdash шепнул просёкший Киря, проходя рядом. 

\diagdash Да хар\'{о}ш!\mdash отозвался тот.\mdash Таки не делайте мне м\'{о}зги, да?

\diagdash Я серьёзно.

Сеструха, тем временем, задорно отплясывала в центре зала, широко улыбаясь, светя забитой татухой рукой и~оголёнными плечами, справедливо упиваясь торжеством молодости, красоты, задора. Светлые завитые локоны подплясывали в такт движениям, вишнёвое вечернее платье подчёркивало все её сочные достоинства.

\diagdash Пошли потанцуем?\mdash вдруг жарко шепнула жена на~ухо Шурику.

\diagdash А пошли!\mdash он приобнял жену и закружил, словно желая выкружить все крамольные мысли из головы$\ldots$

\vspace{0.5cm}
$\ldots$Стол ломился, друзья разгульно праздновали:

\diagdash Дорогие Кирилл и Надежда$\ldots$\mdash полились речи, всё как обычно, ничего нового. От этих речей Шурику вдруг стало невообразимо скучно, да так, что свело скулы. Серёга, сидевший рядом, наклонился к нему:

\diagdash Ливани красненького?

\diagdash Серж, ты идёшь в поход?\mdash Шурик налил ему сухого. 

\diagdash П\sdash прям щ\sdash щас что ли, тащ Адмирал?

\diagdash Да ну тебя! В августе, как обычно.

\diagdash Обычно$\ldots$ Три года не ходили уже. Т\sdash ты cлышишь? Три года!\mdash Серёга хмельно потянулся.\mdash Пошли, чё~не~сходить, я~так\sdash то не против, только жена вот$\ldots$

\diagdash Переженились, черти, хрен кого выпрешь на~речку!\mdash моментально вспылил Шурик, перебивая. Ему как всегда было немного обидно, что даже маленькую команду проблематично собрать.

\diagdash Или хрен кого дома удержишь?\mdash тут же полоснула, словно опасной бритвой по горлу, жена Адмирала и обиженно отвернулась$\ldots$ 

\vspace{0.5cm}
$\ldots$Веселье вокруг дошло, как говорится, до точки. Шурик знал Серёгу и Кирю тысячу лет\mdash с 1\sdash го курса. Они в последние годы то ли от безысходности, то ли чёрт знает почему, стали вместе ходить в походы. Вероятно,~это давало каждому из них что\sdash то эдакое очень нужное. Кире\mdash возврат к детским ощущениям, когда тот ходил на байдарке с~родителями, Серёге\mdash совершенно легальный способ слинять из дому на недельку развеяться. Шурику же, как Адмиралу, нужна была команда единомышленников, и~лучшего выбора, чем старые друзья\sdash товарищи, в принципе не могло и быть. Словом, интересы плюс\sdash минус совпадали. 

Друзья ранее успели сходить в два совместных сплава по Песи и Чагодоще, рекам Вепсовской возвышенности, куда их затащил Шурик. Но то было уже несколько лет назад. С тех пор у них никак не получалось выбраться вместе на речку\mdash то заботы, то работы, то всякие пандемии и~прочий кошмар. Шурика всё это бесило просто страшно\mdash что с~возрастом нельзя уже как раньше просто взять, подорваться и умотать куда глаза глядят. Глаза, естественно, глядели на~речки Вепсовской возвышенности, что между Москвой и~Петербургом. Вепсовский край очень полюбился им своей заброшенностью, первозданностью, отсутствием толп туристов и сплавщиков, а также относительной близостью к Москве. Но в этом году Шурик твёрдо решил\mdash только Карелия, только пороги! Сколько можно матрасничать на~тихих речушках? Пора уже повысить категорийность своих приключений. Эти мысли день ото дня сверлили сознание и~не~давали покоя, и, хотя ещё была зима, все помыслы уже были об августе, походе, сплаве.

\newpage
Из~колонок, тем временем, лилось:

\vspace{0.01cm}
\noindent\textit{%
	\hspace*{1.4cm}Любовь зимой приходит в платье белом,\\
	\hspace*{1.4cm}Весной любовь приходит в платье голубом,\\
	\hspace*{1.4cm}Любовь приходит летом в платьице зелёном$\ldots$
}
\vspace{0.01cm}

%$\ldots$
Народ кружился в медляке. Шурик, держа жену в~танце, вдруг с грустью подумал, что, похоже, уже все товарищи\sdash друзья переженились, и дальше поводом массовых встреч, подобных этой, будут лишь немногочисленные юбилеи (кто в наше время ещё их отмечает?), да похороны. От~такого умозаключения у него вдруг сделалось как\sdash то скверно, черн\'{о}, противно на душе, и трое друзей, Киря, Серёга и Шурик, отлучившись от общей массы, осели в~закуточке:

\diagdash Ну, Кирь, банальности типа <<совет да любовь>> сказали уже все, так что я так скажу\mdash просто будь счастлив и~научись уступать жене, не давая при этом, тэк скэзэть, слабины,\mdash внезапно выдал Шурик,\mdash ну и от сплава ты не~отвертишься. Думаешь, женился и всё? Не прокатит! Мы~всё равно тебя вытащим на речку!

\diagdash Спасибо, Шурик! Спасибо вам, парни!\mdash Киря с~усилием опёрся на плечи друзей.\mdash Будем!!!\mdash коньяк был хорош.

Шурик специально затащил их сюда в закуточек\mdash ему хотелось запомнить их такими, какими они были сейчас\mdash навеселе, молодые, его друзья. Киря похорошел, как встретил~Надю. Та была немногословна, но, судя по~всему, умела держать Кирю <<в узде>>\mdash тот стал немного меньше пить и курить, следить за здоровьем, что, безусловно, лишним никогда не бывает. Шурик перевёл взгляд на Серёгу. С ним они прошли почти все институтские годы, огонь и воду, как говорят, в одной бригаде по~лабораторкам, выпускались с~одной кафедры, и продолжали, несмотря ни~на~что, работать по своей специальности. Волосы Серёги подёрнулись уже серебром. <<Рановато>>,\mdash c~грустью подумал Шурик. Друзья, в свою очередь, тоже запоминали его, или показалось? На~Шурика нахлынула какая\sdash то неподобающая месту и поводу тоска, ностальгия по~их~былым временам, но~он прогнал прочь эти воспоминания и смотрел, смотрел на~друзей, запоминая.  

Они обсудили чутка поход, никто не был против:

\diagdash Чтоб в Карелии, ик, узнали про нас!!!%\mdash~подняли~бокалы.% Серёга.

\diagdash Два отрывистых и одно раскатистое, тащ Замполит!!!\mdash пробасил Шурик.

\diagdash {\large УРА, УРА, УРА\sdash А\sdash А!!!}\mdash грянули старые друзья. Каждому хотелось верить в завтрашнее лучшее, светлое, вечное$\ldots$

%\vspace{2.0cm}
\vspace{0.5cm}
$\ldots$Утром у Шурика зажужжал телефон. 

\diagdash Чё хотел, Серёг?

\diagdash У тебя нет, случайно, моего пиджака?

\diagdash С чего бы?!

\diagdash Походу в кафе забыл$\ldots$

\begin{center}
	\psvectorian[scale=0.4]{88} % Красивый вензелёк :)
\end{center}
}