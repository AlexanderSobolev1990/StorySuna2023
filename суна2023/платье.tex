{
\chapter{В платье белом}
%\corner{64}
\vepsianrose

\fancyhead[LE]{\fancyplain{}{\bfseries \parttitle}}
\fancyhead[RO]{\fancyplain{}{\bfseries \rightmark}}

Люди. Как с ними всегда тяжело.

\diagdash Пойдёшь в поход?\mdash спрашиваешь одного.

\diagdash Не\sdash е\sdash е, там рюкзак тяжеленный$\ldots$

\diagdash Так то в пешем, а у нас\sdash то водный! Мы вещи в лодку погрузим и алга! Ничё тащить не надо особо!

\diagdash Да слушай, у меня ни отпуска, ни снаряжения$\ldots$

И такое\mdash регулярно. Шурик перебрал все заметки в телефонной книжке, напряг всех знакомых, привлёк всевозможные хитросплетения цепочек знакомств, случайных встреч и прочее\sdash прочее\sdash прочее. Нулевой результат. Как всегда собирался только костяк команды, новых людей не предвиделось. Вдобавок, Юрич, только заслышав слова <<Карелия>>, <<пороги>> и <<двухдневная заброска>>, посчитал что это всё\mdash уже не для него. Не~потянет. Колено травмировано, возраст в целом, да~и~убиться об камни\mdash такое себе пенсионное развлечение. Команда сжалась до четырех человек. Шурик ходил мрачнее тучи. Микроколлектив\mdash это хорошо, тем более схоженный не~одним походом, но$\ldots$

Давно уже он присматривался к карельским речкам. Хотелось открыть Карелию красивой, но в то же время несложной рекой. То и дело в описаниях маршрутов встречались пороги 3\sdash й категории, что в планы как\sdash то не~входило, ровно как и обносы. Искал он, короче говоря, реку 2\sdash й категории где\sdash нибудь в южной или центральной Карелии. В итоге выбор пал на интересный кольцевой маршрут <<Сунская цепочка>>, замкнутый на посёлок Гирвас. Далековато, конечно, от Москвы. Но кого и когда это останавливало? Постепенно, за разговорами и обсуждениями маршрута с Юричем, Шурик всё более свыкался с мыслью, что его команда, костяк которой постепенно формировался годами, сможет осилить такое путешествие с двухдневной заброской и~выброской. 

Шурик как обычно ехал домой с работы в электричке с Курского вокзала. Занюханная электричка медленно отползала от вокзала, раскачиваясь на ушатанных стрелках. Хотелось тишины, покоя. Но внутри потянулись торговцы, большинство из них торговало тут годами. Один из них особенно нравился Шурику:

\diagdash Шариковые ручки по ценам брежневских времён! Самовывоз со склада в Антарктиде!\mdash вещал прокуренным голосом весельчак с трёхдневной седой щетиной.\mdash Я же, со своей стороны, гарантирую вам бесплатную доставку в любую точку$\ldots$\mdash мужик делал драматическую паузу, наслаждаясь производимым эффектом.\mdash $\ldots$вагона! Итак,~пжалста! Ручки по ценам брежневских времён$\ldots$

Шурик прислонился к стеклу окна в вагоне и уныло посмотрел вслед уходящему мужичку с ручками. Как ему безумно, должно быть, всё осточертело, думал Шурик, как и~сотням, тысячам, сотням тысяч людей$\ldots$ В вагон ввалилась следующая торговка и стала что\sdash то впаривать. Но разве можно было переплюнуть <<Антарктиду>>? Спустя 40 минут Шурик высыпался из электрички на конечной и побрёл домой, заглянув в магазин. Внутри боролись примерно два чувства\mdash дикое желание сдохнуть и дикое желание пойти в~поход, чтобы не сдохнуть. Сопротивляться последнему день ото дня становилось просто невозможно$\ldots$

$\ldots$Вечерело. Шурик вышел в тапках на общий балкон, раскурил трубку и набрал Замполиту\mdash Кире.

\diagdash Слушаю!

\diagdash Кирь, такое дело, пошли в Карелию?

\diagdash Как 2 пальца. Когда?

\diagdash Ну как обычно, начало августа$\ldots$

\diagdash До августа ещё дожить надо$\ldots$

\diagdash Оптимист, ёпт! Короче\mdash все наши в теме, Юрич\mdash пас. Ты как?

\diagdash Шурик, я женюсь скоро.

\diagdash Ё\sdash ё\sdash ё$\ldots$

\diagdash Поэтому я и говорю\mdash до августа ещё дожить надо!

Шурику вдруг стало невообразимо горько, но в то же время радостно за Кирю\mdash таки женится$\ldots$ небольшой кусочек семейного счастья\mdash островок спокойствия, так~нужный, пожалуй, каждому$\ldots$ 

Трубка потухла, снова раскуривать не захотелось и~Шурик, бессменный Адмирал Сплава, вышел с~балкона с~такой~же неопределённостью, как и заходил.

Жена суетилась на кухне, запахи витали умопомрачительные. Шурик тихо разделся, вытряхнул потухшую трубку, хотел бесшумно проскользить в комнату. Паркет скрипнул. 

\diagdash Почисть картошки?\mdash попросила, услышав, с кухни жена.

Шурик замер на предательской паркетине в коридоре.

\diagdash Ты слышишь?

\diagdash Да, сейчас$\ldots$

Стоя у мойки и совершенно машинально счищая овощным ножом кожуру, Шурик мысленно унёсся в~Карелию$\ldots$ такое желанное место представлялось ему, почему\sdash то, чем\sdash то особенным. Исполинские сосны на камнях, дождливая погода, рыба$\ldots$ он даже прикрыл глаза, чистя картошку на ощупь. Он внезапно понял, что все эти предыдущие сплавы были лишь тренировкой перед Карелией, перед взятием порогов. А вокруг чтобы\mdash красивейшие берега с буйством растительности, скалы, видевшие ещё динозавров, сосны с янтарной корой и~душистой хвоей$\ldots$ от желания почувствовать это чуть не~закружилась голова. И чтобы людей вокруг\mdash ни души, только своя бравая команда. Ух, парус с собой ещё взять по~озёрам походить$\ldots$ от одной только мысли об этом у~Шурика застучало в висках.

\diagdash Всё хорошо?\mdash спросила жена.

\diagdash А? Да$\ldots$\mdash Шурик открыл глаза.

\diagdash Кому звонил?

\diagdash Кире.

\diagdash Как он? 

\diagdash Жениться надумал$\ldots$ 

\diagdash А сам чего смурной? А, понятно! Вы опять намылились в поход, а тут такая подстава, да?\mdash ледяным холодом скользнула жена взглядом.

\diagdash В целом, да.\mdash не стал отпираться Шурик.

\diagdash Всё образуется.\mdash пророчески сказала жена и~вытащила из духовки ужин.\mdash Хрен вас кто удержит от~ваших пьянок на воде.

Шурик хотел было устало прочитать ей лекцию о~технике безопасности распития на воде, но не стал$\ldots$ 

\vspace{0.5cm}

$\ldots$зал был полон народу, Шурик наверно в первый раз видел Кирю в костюме\mdash похорошел, откормился. Невеста была, как и положено, в платье белом, <<как майская роза стройна и свежа>>. Веселье шло своим чередом, постепенно закручиваясь как бы по спирали с увеличением градуса в крови. Тосты лились рекой, народ самоорганизовывался, поскольку Киря мудро не позвал тамаду на сие действо. Серёга активно опрокидывал бокал за бокалом, а Шурик, не~отставая, полвечера косился на кирину сестру:

\diagdash Шурик, у неё характер\mdash кабздец, не вздумай\mdash шепнул Адмиралу Замполит, проходя рядом. 

\diagdash Да харош!\mdash отозвался тот.\mdash Таки не делайте мне м\'{о}зги, да?

\diagdash Я серьезно.

Сеструха, тем временем, задорно отплясывала в центре зала, широко улыбаясь, светя забитой татухой рукой и~оголёнными плечами, справедливо упиваясь торжеством молодости, красоты, задора. Красное вечернее платье подчёркивало все её сочные достоинства.

\diagdash Пошли потанцуем?\mdash вдруг жарко шепнула жена на~ухо Адмиралу.

\diagdash А пошли!\mdash Шурик приобнял жену и закружил, словно желая выкружить все крамольные мысли из головы. 

\diagdash Дорогие Кирилл и Надежда$\ldots$\mdash полились речи, всё как обычно, ничего нового. От этих речей Адмиралу вдруг стало невообразимо скучно, да так, что свело скулы. Серёга, сидевший рядом, наклонился к Шурику:

\diagdash Ливани красненького?

\diagdash Серж, ты идёшь в поход? 

\diagdash П\sdash прям щ\sdash щас чтоли?

\diagdash Да ну тебя! В августе, как обычно.

\diagdash Обычно$\ldots$ Три года не ходили уже. Пошли, че не~сходить, я так\sdash то не против, только жена вот$\ldots$

\diagdash Переженились, черти, хрен кого выпрешь на~речку!\mdash вспылил Шурик.

\diagdash Или хрен кого дома удержишь?\mdash полоснула, словно опасной бритвой по горлу, жена Адмирала. 

Веселье дошло, как говорится, до точки$\ldots$ Шурик знал Серёгу и Кирю тысячу лет\mdash с 1\sdash го курса, они в последние годы то ли от безысходности, то ли чёрт знает почему, стали вместе ходить в походы. Вероятно это давало каждому из них что\sdash то эдакое очень нужное. Замполиту\mdash возврат к детским ощущениям, когда тот ходил на байдарке с родителями, Серёге\mdash совершенно легальный способ слинять из дому на недельку развеяться. Словом, интересы плюс\sdash минус совпадали. Шурику же, как Адмиралу, нужна была команда единомышленников и лучшего выбора, чем старые друзья\sdash товарищи, в принципе не могло и быть. Из~колонок, тем временем, лилось:

\vspace{0.1cm}
\noindent\textit{%
	\hspace*{1.4cm}Любовь зимой приходит в платье белом,\\
	\hspace*{1.4cm}Весной любовь приходит в платье голубом,\\
	\hspace*{1.4cm}Любовь приходит летом в платьице зеленом$\ldots$
}

$\ldots$народ кружился в медляке. Шурик, держа жену в~танце, вдруг с грустью подумал, что, похоже, уже все товарищи\sdash друзья переженились, и дальше поводом массовых встреч, подобных этой, будут лишь немногочисленные юбилеи (кто в наше время ещё их отмечает?), да похороны. От~такого умозаключения у него вдруг сделалось как\sdash то скверно, черн\'{о}, противно на душе, и трое друзей, Киря, Серёга и Шурик, отлучившись от общей массы, осели в~закуточке:

\diagdash Ну, Кирь, банальности типа совет да любовь сказали уже все, так что я так скажу\mdash просто будь счастлив и~научись уступать жене, не давая при этом, тэк скэзэть, спуску.\mdash внезапно выдал Адмирал.\mdash ну и от сплава ты не отвертишься. Думаешь, женился и всё? Не прокатит!

\diagdash Спасибо, Шурик.\mdash Киря с усилием опёрся на плечи друзей.\mdash Будем!!!\mdash друзья выпили за молодых, коньяк был хорош.

\diagdash Чтоб в Карелии, ик, узнали про нас!\mdash молвил Серёга.

\diagdash Два отрывистых и одно раскатистое, тащ Замполит!!!\mdash пробасил Шурик.

\diagdash {\large УРА, УРА, УРА\sdash А\sdash А!!!}\mdash грянули старые друзья$\ldots$ каждому хотелось верить в завтрашнее лучшее, светлое, вечное$\ldots$

\vspace{1.5cm}

$\ldots$утром у Шурика зажужжал телефон. 

\diagdash Че хотел, Серёг?

\diagdash У тебя нет, случайно, моего пиджака?

\diagdash С чего бы?!

\diagdash Походу в кафе забыл$\ldots$

\begin{center}
	\psvectorian[scale=0.4]{88} % Красивый вензелёк :)
\end{center}
}