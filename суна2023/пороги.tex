\chapter{Штурм}
%\corner{64}
\vepsianrose

По закону подлости с утра начал моросить дождь, облачность повисла низко\sdash низко, казалось, что облака задевают верхушки сосен, растущих вокруг их днёвочной стоянки. К тому моменту, как парни закончили завтрак, дождь усилился.

\diagdash Тащ Адмирал? Какие будут указания?\mdash Замполит нацепил свой жёлтый дождевик, заправившись протеином перед греблей.

\diagdash Сворачиваемся, а куда деваться?! Дождь\mdash значит дождь. У природы нет плохой погоды, слыхал про такое?

Адмирал уже облачился к штурму\mdash нацепил вместо привычной тельняшки термобельё, а сверху свой видавший виды старый дождевик зелёно\sdash синего цвета:

\diagdash Пакуемся, мужики, потихонечку. Может дождь прекратит.\mdash он сидел под тентом и запрессовывал продукты в гермы, уплотняя их. Удалось из двух герм уже сделать одну\mdash минула половина похода, запасы уменьшились.

Парни тоже приоделись кто во что: 

\diagdash Шурик, а дождевик как?\mdash спрашивал Серёга.\mdash Я~имею в виду поверх или под спас?

\diagdash Спас поверх дождевика, иначе ерунда выйдет.\mdash Адмирал собрал прикостровое имущество, запаковал продукты, а Замполит пока сворачивал их палатку.

\diagdash Едрить\sdash мадрить, ещё на воду не встали, а уже все промокли!!!\mdash Замполит потихоньку начал таскать вниз к воде и байдаркам вещмешки. Вдоль тропинки рос папоротник, широкие мокрые от дождя листья которого моментально смачивали проходящего там, где длины дождевика уже не~хватало\mdash низ штанов у всех уже был сырой.

\diagdash Нормально! Это тренировка перед порогами!\mdash приободрял всех Адмирал как мог, в своей манере.

Дождь ещё усилился. Адмирал крикнул от стоянки вниз к~воде чтобы там накрыли полиэтиленовой плёнкой вещи и они не мокли понапрасну, пока их ещё не погрузили в~байдарки. К тому моменту, как лагерь был почти полностью свёрнут, а вещи собраны и пора было грузиться, пошёл натуральный ливень.

\diagdash Замполит, пошли байды грузить.\mdash Адмирал, осмотрев стоянку на предмет забытых вещей, шёл вниз к~воде.

\diagdash Может переждём? Дождило такой!\mdash влез Серёга.

\diagdash И сколько ждать?\mdash Замполит согласился с~Адмиралом и они, уже перевернув байды на ровный киль, стали потихоньку грузить гермы. 

Ситуация осложнялась тем, что уберечь байды от~заливания дождём, пока гребцы не сидят в них, не было никакой возможности, и потоки воды, лившиеся с небес, собирались в трюмах.

Паша, Руслан и Серёга цепочкой таскали вещи сверху от~стоянки к воде по склону, который стал очень скользким от~дождя:

\diagdash Аккуратно!!!

\diagdash Ух, ё!\mdash чуть не навернулся кто\sdash то из них.

\diagdash Чё там у вас?\mdash услышав, спросил Адмирал. Они с~Замполитом стояли по колено в воде у байд и запихивали вещи каждый в свой корабль.

\diagdash Чё, чё. Грунт размок!\mdash ответили парни.\mdash Уф, это последние мешки.

\diagdash Шурик, ты видал скока воды в трюмах???\mdash Серёга, отпустив нос байды, после того как погрузил вещи, увидел, как вода, собравшаяся там, хлынула по всему днищу.

\diagdash Ну а что я тебе сделаю? Могу вот так только,\mdash он~поднял голову к небу и прислонил ладонь ко рту,\mdash Роман Менделич, выключите, п\sdash жалста, нам дождь!!!

\diagdash Не работает нихрена. Надо {\'{O}}дина или кого там просить, Шурик.

\diagdash Не иначе! Так, где моя, пардон, дрочилка?

\diagdash Что, прости?!?!

\diagdash Спокуха!\mdash Адмирал рылся в вещмешке где были байдарочные принадлежности.\mdash Во! Нашёл!\mdash он~вытащил грушу для отсасывания бензина с~обратным клапаном.\mdash Надо ей воду со дна таки откачать перед отплытием!\mdash он пару раз зажал и разжал грушу, отчего клапан в ней издал чавкающе\sdash булькающий звук.

\diagdash Ы-ы-ы! 

\diagdash Ну как\sdash то так, парни!\mdash Адмирал ещё раз проверил берег на~предмет забытых вещей и велел всем садиться в~байды.\mdash Пока мы собой и байдарочными юбками не~закроем проёмы, байды так и будет заливать дождём!

Команда уселась на свои места и закрыла юбки, натянув их на фартук и герметизировав таким образом себя от~дождя. Для этого пришлось помучиться\mdash пластиковая трубка, вшитая в фартук, видимо где\sdash то разошлась, а~юбки на~это не~были рассчитаны. 

Адмирал вывел наружу шланг, подсоединённый к груше, и начал откачивать воду из трюма:

\diagdash Чавк\sdash чавк! Чавк\sdash чавк! Чавк\sdash чавк!

\diagdash Ы-ы-ы!

\diagdash Очень смешно!

\diagdash Ы-ы-ы! Дай потом нам, у нас тоже воды по~кильсон!\mdash Замполит обошёл адмиральскую байду.

На воду они встали только около полудня, но Адмирал не особо переживал по поводу позднего выхода на воду, он~вообще решил, ещё в первый день, что постарается как можно меньше в~этом сплаве переживать и как можно больше всё пускать на~самотёк. Они потихоньку правили по~Линдозеру в Суну, параллельно откачивая воду из трюмов. Справа на~мысу, где заканчивалось озеро и~начиналась уже река, ребята увидели хорошую большую стоянку с большим песчаным пляжем:

\diagdash И чего сюда не пошли позавчера?\mdash спросил Серёга.

\diagdash Не знаю, спроси у Адмирала.\mdash отозвался Замполит.

\diagdash Так занято же! Вон палатки в глубине стоят! И~позавчера стояли! Да и хотелось на острове постоять$\ldots$ \mdash признал Адмирал.\mdash На тебе грушу! Я откачался!\mdash он~передал Серёге приспособу.

\diagdash Чавк\sdash чавк! Чавк\sdash чавк! Чавк\sdash чавк!\mdash доносилось теперь с замполитовой байды.

Проходя занятую байдарочниками стоянку, Адмирал увидел там трёх\sdash четырёх$\ldots$ детей! Если когда позавчера эскадра обходила мыс на~острове, те ребятишки не~произвели на Адмирала никакого впечатления, потому что плавсредство там было\mdash моторка, то тут это были дети байдарочников. У Адмирала сразу отлегло и стало как\sdash то светло и~хорошо на душе, хотя погодные условия этому совсем не~способствовали\mdash он окончательно понял, что ничего особо страшного от~этой реки второй категории ждать не~придётся: <<Если здесь с~детьми ходят, то мы\sdash то уж точно выдюжим этот маршрут>>,\mdash подумал Адмирал и~подналёг на весло.

Вода была, без преувеличения, везде. В озере, по~которому они стремительно приближались к Суне и~активной части маршрута, в небесах, откуда их обильно поливало потоками, в байдарках, откуда они, яростно <<чавкая>> грушей, откачивали воду. Вскоре они вошли в~реку, и по левому берегу сразу началась деревня Линдозеро, правда выглядела она как\sdash то заброшено. На берегу была табличка с~телефоном какой\sdash то турбазы:

\diagdash А чойта мы в гостишку не приплыли?\mdash угорал Замполит.

\diagdash Ага, крова-а-ать, ко-о-офэ с булочкой, да?\mdash поддержал, угорая, Паша.

\diagdash Кстати кофе молотый у меня где\sdash то есть запасец, сварим завтра! Так, парни, впереди гряда островов, идём посередине!\mdash проинструктировал всех Адмирал.

Островки они вскоре миновали, оставив деревню по~левому борту, и стали подходить к огроменному, метров на~150, старому деревянному мосту через Суну. Мост, когда\sdash то бывший, видимо, автомобильным, безнадёжно обветшал. Вместо каких\sdash то пролётов лежало всего пару досок\mdash местным из деревни ходить в лес на том берегу.

Пока они приближались к мосту, особо не гребя, дождик как\sdash то подутих. Адмирал спросил:

\diagdash А чё, парни, деревня же. Связь есть?

Серега, Киря и Руслан полезли проверять:

\diagdash Если чё не загрузил вчера\mdash давай щас,\mdash обратился Адмирал к Замполиту,\mdash а то скоро порог.

\diagdash Описание всё загружено, полностью$\ldots$\mdash Замполит высоко поднимал весло, отрабатывая утренний протеин.

Эскадра вплотную приблизилась к~мосту, облачность стала не такой угрюмой, чуть посветлело, дождь стал совсем редким, как редкая водяная пыль, скорее. 

\diagdash Шурик!\mdash Киря упахивался веслом.

\diagdash А?

\diagdash В каком пролёте пойдём?

\diagdash Да давай в центральном! Дай, я первый пойду.\mdash Адмирал вырулил в середину пролёта и сказал своему экипажу не грести\mdash они плавно по инерции медленно приближались к опорам моста. 

\diagdash Шурик, аккуратно!\mdash Замполит заходил следом.

Адмирал, оттабанив, почти полностью погасил скорость, и они шли по течению, поскольку он~перестраховывался от всякой гадости, которая по~обыкновению может лежать или перед мостом, или под~ним\mdash брёвна всякие с гвоздями, балки и прочая ерунда. К счастью, никакой дряни под мостом не оказалось, и они плавно миновали его, стали набирать ход. Адмирал отметил про себя, что мост очень хлипок, проходим пешком просто на~грани экстрима.

Впереди команда заприметила мужиков на двух надувных моторках\mdash те рыбачили на ходу под правым берегом. Адмирал же с Замполитом правили по левому берегу, как было сказано в описании порога, к которому они неумолимо приближались$\dots$ 

Шум Уйтуженкоски они заслышали уже метров за~четыреста. Адмирал взял ещё левее, под самый берег, чтобы было дальше видно порог, который, следуя изгибу реки вправо, уходил за поворот впереди на полкилометра. Ширина русла была огромна\mdash метров 70 у~начала порога. Костяк команды, привыкший к относительно узким и спокойным речушкам вепсовской возвышенности, был маленько в шоке\mdash всё широченное русло Суны, насколько хватало взора, бурлило валами с~белой пеной и барашками, сплошной шумящей и гремящей кашей устремлялось вдаль за поворот.

\diagdash Чалимся у левого берега!\mdash отдал команду Адмирал.\mdash Аккуратно только заходим, там камни!

\diagdash Охренеть каша белая!\mdash экипажи прибалдели.

Команда, подзатащив байдарки на прибрежные валуны у мелководья левого берега и оставив их так, прошла пешком чуть дальше, насколько позволяли камни:

\diagdash Охренеть!!!

\diagdash Ты видишь куда тут идти?!?!

\diagdash Валы по метру!!!

\diagdash Ёпрст!!!

Они стояли и чесали затылки, рассматривая бурлящую и кипяющую, словно в каком\sdash то колдовстве реку\mdash действительно, ни на секунду не прекращающееся холодное кипение было похоже на что\sdash то такое нереальное, неестественное, чего не может быть в сознании человека, выросшего на спокойных размеренных реках средней полосы~России.

\diagdash Ну, надо идти. Киря, доставай и читай описание порога!\mdash очнулся от завораживающего созерцания~Адмирал.

\diagdash Ага! Так$\ldots$ <<На входе под левым берегом шивера с~бурным течением.>> Ну вон она, походу!\mdash он неуверенно показал куда\sdash то вперёд.

\diagdash Ну типа. Дальше давай!

\diagdash <<Далее надо переходить под правый берег.>> Хм-м-м!

\diagdash Логично, парни, смотрите какие валы слева! Там~до~метра, походу!\mdash Адмирал показал вперёд чуть левее, примерно на середину того, что было перед их~глазами.

\diagdash Охренеть, охренеть!\mdash только и повторял Замполит. Серёга и Руслан молча стояли, смотрели на бурлящую пену, а Паша с Адмиралом, стоя на камне, прикидывали как и что.

\diagdash Кирь, давай читай, хар{\'о}ш рефлексировать!

\diagdash Ага. <<Ближе к~выходу поток стесняется остатками плотины и отмелью слева, образуя крутую горку со стоячими валами. В низкую воду в~струе появляются два опасных обливняка, хорошо заметных по пенным гребням за ними>>$\ldots$

\diagdash Хорошая новость\mdash низкая вода это не про нас\mdash смотрите на прибрежные камни! Когда вода спадает, на них такие полосочки образуются, а сейчас ничего такого нет.\mdash заключил, воодушевляя всех, Адмирал.

\diagdash Да, походу ты прав$\ldots$

\diagdash Короче, парни! Паш, смотри тоже сюда!\mdash Адмирал встал повыше на камни и стал размахивать руками.\mdash Заходим ровно по центру сначала\mdash тут всё равно как идти, похоже. Дальше, как и сказано в описании, прижимаемся к правому берегу, обходя вон ту жесть,\mdash он показал на самые высокие валы в пороге слева,\mdash а потом как пойдёт, но я подозреваю, что лучше правого берега держаться.

\diagdash Шурик, ты уверен?\mdash Серёга грустновато смотрел в~порог.

\diagdash А есть ещё варианты? Погнали! Замполит, включай рацию, как пройдём, я выйду на связь!\mdash и, обернувшись к~своему экипажу, махнул рукой.\mdash По коням, парни!!!

Хоть Адмирал и хорохорился, на душе у него были такие терзания, что просто не передать. Он притух от~уходящего в неизвестность белого бурлящего потока, но в то же время вспомнил про тех байдарочников с детьми\mdash это немного придало ему сил. Его экипаж уселся в~байду и стал отчаливать. Адмирал был, как ему казалось, на~эмоциональном пике, хотя основный пик ещё был впереди:

\diagdash Так, отходим подальше, разворачиваемся, выравняемся по руслу и ровненько заходим на <<язык>>.\mdash Адмирал застегнул байдарочную юбку, герметизировав себя в фартуке.

\diagdash Куда заходим?\mdash экипаж тоже копошился с~байдарочными юбками.

\diagdash Ну на <<язык>>\mdash вон в первой ступени порога читается такой треугольничек тёмной воды на сливе\mdash там должно быть поглубже.\mdash Адмирал упёр весло в~фартук и~фальшборта, усаживаясь повыше.\mdash Спасжилеты проверьте! Начали!

\diagdash Сань, да погоди ты, я юбку не закрыл!\mdash бросил Пашка.

\diagdash А?\mdash не расслышал на корме Адмирал.

\diagdash Погоди говорю!!!

\diagdash Мы заходим!!!

\diagdash Уй-ё-ё-ё!!!

Адмирал выравнял курс и, сам не зная почему, поднёc два прямых перста правой руки ко лбу:

\diagdash {\'{O}}дин, всеотец, будь с нами$\ldots$\mdash почти беззвучно прошептал он, а потом заорал:\mdash {\large ВПЕРЁ-Ё-ЁД!!!}

Они зашли на <<язык>> и нырнули в порог\mdash стихия моментально разогнала их байдарку, они успешно миновали вход. Порог стремительно увлекал их судёнышко в самую гущу событий, в око бури:

\diagdash {\large ЛЕВЫМ!!! ЛЕВЫМ!!!}\mdash заорал Адмирал, когда волны кидали их корабль, как щепку, а беспорядочное течение и завихрения потока стали заворачивать их левее, где были огроменные валы.\mdash {\large УПЁРЛИСЬ ЖЁСТКО ПО~ЛЕВОМУ!!!}\mdash продолжал орать Адмирал, перекрикивая шум порога. Парни молниеносно отгреблись, а он сильно затабанил слева, вновь выровняв курс и встав ровно перпендикулярно валам. 

\diagdash {\large ПОЛНЫЙ НАХОД!!!}\mdash они пошли пробивать огроменные валы, нараставшие один за другим впереди\mdash один\mdash БАМ!!!\mdash второй\mdash БАМ!!!!!!\mdash третий\mdash БАМ!!!!!!!!!

Волны окатывали их по полной, не щадя! Нос байдарки зарывался в~валы так, что Адмирал чувствовал, как кровь стынет у него в жилах! Холодная вода порога окатывала и Пашку, сидящего впереди, и Руслана, и даже успевала докатиться до Адмирала на корме, ему тоже досталось\mdash залило в юбку, а уж с вёсел экипажа ему и~вовсе летели брызги прямо в глаза. Экипаж чётко отрабатывал все его команды, которые приходилось орать, перекрикивая шум, грохот даже, воды. Они ровнёхонько встречали вал за~валом, продвигаясь вперёд. Наконец, посреди порога, уже на~cамом повороте, максимальной точке, которую они смогли просмотреть перед началом штурма, вода внезапно чуть подуспокоилась, можно было подрасслабиться. Адмирала захлестнула просто такая эйфория, какую он, пожалуй, никогда и ни от чего в своей жизни не испытывал. Это было чувство выжившего, чувство превосходства над~стихией, восторженность от преодоления и порога, и самого себя, своих страхов и неуверенностей, это было то чувство, тот эмоциональный подъём, за~которым они все без исключения оказались тут. 

Их стало немного разворачивать к правому берегу:

\diagdash {\large Э!!! Саня!!!}\mdash заорал спереди Пашка.

Адмирал, очнувшись, воспрял и жёстко затабанил слева, они вновь пошли ровно на валы\mdash начиналась вторая ступень порога:

\diagdash Готовимся, парни!!!

\diagdash Ага!!! Ровненько!

%\renewcommand*{\thefootnote}{\arabic{footnote}}
\renewcommand*{\thefootnote}{\fnsymbol{footnote}}
\setcounter{footnote}{0}
Секунда и новые валы окатили их, байдарка ходила ходуном, дифферент\footnote{Дифферент судна (от лат. differens, differentis\mdash разница)\mdash наклон судна в продольной плоскости\cite{МорскойСправочник}.} на нос и корму достигал каких\sdash то жутких значений. Адмирал старался править так, чтобы заходить к новому валу максимально перпендикулярно, исключая опасность боковой качки и оверкиля, врезаясь в~бурлящую воду носом. Чувство эйфории не проходило\mdash они успешно проходили остаток порога. Впереди у правого берега была гряда торчащих из воды камней, а у левого мель и брёвна. Посередине, S\sdash образным изгибом влево и тут же вправо, шла основная струя. Адмирал понял, что это, возможно, и есть остатки какой\sdash то там плотины из описания, и им туда\mdash пора было выходить из\sdash под правого берега ближе к середине русла, чтобы не напороться на камни:

\diagdash {\large ПРАВЫЙ НАХОД}\mdash заорал он, а сам слегка оттабанил слева и они идеально вошли в стремнину. Далее он оттабанил справа, и парни, легко пройдя уже не~такие высокие, как ранее, валы в конце порога, вышли на~спокойную воду, по которой дальше плыла взбитая стихией пена.

\diagdash {\large ПРОШЛИ!!!}\mdash кричали они.

\diagdash Руслан, Паш, как вы?\mdash Адмирал правил к~островочку посередине заводи после порога.

\diagdash Я те орал: <<Погоди!>>, а ты попёрся!!!\mdash Пашка обернулся.\mdash Я ж юбку не до конца закрыл!!!

\diagdash Налило?

\diagdash Да я весь мокрый! Все штаны!!!

\diagdash Ёпрст!!!

\diagdash Ладно, чё теперь!!!

Адмирал развернулся на $180\degree$ по курсу против течения и подошёл к острову посреди русла, стал промерять веслом глубину:

\diagdash Парни, тут гл{\'ы}боко! Не вылезешь особо, давай ещё чуть поближе, за траву прибрежную хватайтесь!

Они подсадили нос байды на гладкие валуны, открыли байдарочные юбки и поглядели на днище байды:

\diagdash Мама дорогая!!! Воды по щиколотку!!!\mdash Адмирал прибалдел от увиденного.

\diagdash А я тебе о чём?!?!?! Залило капитально!!!\mdash Пашка вымок и, естественно, негодовал.
 
\diagdash Руслан, а ты как?

\diagdash Ну, через юбку не натекло, сверху только брызгами намочило нехило так$\ldots$

\diagdash Блин$\ldots$ Извиняй, Паш! Я не понял, что ты не~закрылся!

\diagdash Ладно, забей$\ldots$ Всё равно вымокнем$\ldots$

Адмирал достал рацию и вышел в эфир:

\diagdash Киря, мы прошли, вообще круто!!! Просто огонь!!! Короче, входите ровненько, как мы, а дальше идите чуть левее самых высоких гребней валов у правого берега. В~середине вода чуть поуспокоится перед второй ступенью порога. Как понял?

\diagdash Шурик$\ldots$ Там есть вторая ступень?

\diagdash Да, мы же читали! Она проще первой, там будет такой S\sdash образный поворотик, надо уходить из под правого берега ближе к центру. В~целом, ничего страшного, везде глубоко! Давайте, жмите! Закрывайтесь только юбками хорошо, нам по щиколотку нахлестало воды!

\diagdash Понял, ждите$\ldots$ Мы потихонечку начинаем\mdash голос у Замполита был, конечно, так себе. <<Очкует>>,\mdash подумал Адмирал и отложил рацию.

\diagdash Сань, надо отчёрпываться$\ldots$ \mdash Пашка возился спереди, пытаясь вылезти.

\diagdash Тут фиг вылезешь, да и место не располагает. Блин, дрочилка забилась, не фурычит!\mdash Адмирал попытался прочистить грушу, но не получилось\mdash видимо клапан заело грязью.\mdash Как отчёрпываться\sdash то?

\diagdash Да этой грушей мы вечность выкачивать будем!

\diagdash Ну да$\ldots$ \mdash Адмирал хотел достать свою кружечку, чтобы отчерпаться ей, но вспомнил, что отчего\sdash то переложил её с утра в вещмешок из штормовки. А лезть в вещмешки на плаву было занятием так себе, поэтому в итоге он решил отчёрпываться крышкой термоса, который был под рукой. Занятие было монотонным и требующим отрешения, Адмирал сидел у себя на корме и потихоньку осушал трюм.

Вид на бурлящую порогом реку уходил вдаль вверх по~течению, ребята то и дело пристально вглядывались в белую пену, силясь увидеть второй экипаж:

\diagdash Что\sdash то не идут! Может вызвать их по рации?\mdash переживал Адмирал, отчёрпываясь.

\diagdash Если бы чё\sdash нить случилось, уже бы вещи поплыли$\ldots$ \mdash отжигал Паша.\mdash Так что не кипишуй.

\diagdash Тьфу ты!

\diagdash Ы-ы-ы!

Адмиралу наскучило черпаться, хотелось перекурить после прохождения порога. Он передал крышку от термоса Руслану и тот продолжил дело. Адмирал же откинулся назад, вытащил портсигар, уселся поудобнее, насколько это было возможно, и задымил.

\diagdash Идут!\mdash Пашка увидал второй экипаж, показавшийся из\sdash за поворота. 

\diagdash Первую ступень прошли, черти. Дальше\mdash проще.\mdash Адмирал, щурясь, вглядывался в кашу порога, следя за~замполитовой байдой, то и дело исчезавшей за высокими валами и вновь появлявшейся, пробив очередную водную стену.




\begin{center}
	\psvectorian[scale=0.4]{88} % Красивый вензелёк :)
\end{center}
