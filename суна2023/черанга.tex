\chapter{Бочка}
%\corner{64}
\vepsianrose

Проснулись пацаны поздно\mdash утро было пасмурным и сонным, да и вчера засиделись. Костёр долго не хотел разгораться, все ощущали невыносимую леность. Адмирал решил приготовить на завтрак пшёнку с изюмом на сгущенке и сливках. Каша вышла\mdash класс. Команда, как обычно расселась вокруг костра и трепезничала:

\diagdash Шу-у-урик, от пуза.\mdash Серёга звенел ложкой по~тарелке.

\diagdash Ну.

\diagdash Изюма не поскупился.\mdash Замполит уминал за обе щеки.

\diagdash Ты ж меня знаешь!\mdash Адмирал развалился в~походном кресле и попивал чаёк.\mdash Насчёт голодать в~походе\mdash это не про меня.

\diagdash Сань, чё у нас сегодня?\mdash Паша покончил с кашей и~достал конфеты к чаю.

\diagdash Вчера обсуждали вродь как. Сегодня\mdash ещё 4 порога последних и дальше по широкой Суне почти до Викшозера. Кирь, доставай описание порогов\mdash самое время освежить в~памяти.

\diagdash Дай доесть?\mdash тот сидел в жёлтом дождевике на~бревне и был ещё занят пшёнкой.

\diagdash Давай-давай! Всё в темпе надо бы, уже к полудню времячко\sdash то.

Наконец Замполит добил завтрак и достал мобильник, открыл описание:

\diagdash Так, ну вот$\ldots$ <<Порог Длинный, 3 к.с.>>$\ldots$\mdash Киря оторвал глаза от телефона,\mdash $\ldots$Что это за нахрен? Откуда тройка?

\diagdash Врут. Маршрут заявлен как второй категории. Читай давай.\mdash Адмирал открыл карту в планшете.

\diagdash Да хрен знает! Может из\sdash за одного порога третьей категории не стали поднимать сложность всего маршрута? 

\diagdash Кирь, хар\'ош! Читай.

\diagdash <<$\ldots$В высокую воду это ключевое препятствие нижней части Суны>>$\ldots$ Э-э-э$\ldots$ %Да твою ж налево, Шурик!

\diagdash Ну, вода у нас высокая, как мы поняли. Ты~читай\sdash читай.\mdash ребята сидели в кружок у костра и~слушали.

\diagdash <<Основная струя оттесняется влево плитой в центре, образуя$\ldots$>>\mdash так, ну это ерунда\mdash <<$\ldots$крутые метровые валы>>. Ага. Дальше <<$\ldots$после бочки за плитой, лучше уходить вправо, в просвет между основной струей и отбойными валами у правого берега>>.\mdash Замполит поднял полный укоризны взгляд.\mdash Шурик, опять <<бочка>>!

\diagdash Подытожим. Пишут, что это самый сложный порог на реке. Вода у нас высокая, как мы поняли уже с~вами. Итак, <<бочка>> опять где\sdash то за плитой. Прошлую мы с вами не заметили, а тут как пойдё\sdash ё\sdash ёт.\mdash протянул своё любимое Адмирал.

\diagdash Так. А что ещё пишут?\mdash спросили Серёга с~Русланом.

Замполит бегло почитал дальше и огласил:

\diagdash Ну тут альтернативный путь прохождения есть\mdash пишут, что надо прижиматься к невысоким сливам справа$\ldots$

\diagdash Это уже хрен с ним\mdash раз есть альтернатива, то~не~всё так страшно, на воде соориентируемся. Ты про <<бочку>> ещё разок почитай, п\sdash жалста.

\diagdash <<$\ldots$Cтруя оттесняется влево плитой в центре>>, а~дальше <<после бочки за плитой, лучше уходить вправо>>. Вот.

\diagdash Итак, плита в центре, за ней <<бочка>>, обходить справа. Так?

\diagdash Ну да$\ldots$\mdash неуверенно отозвался Замполит.

\diagdash Все запомнили?\mdash Адмирал оглядел команду.

\diagdash Угу!\mdash Пашка наклонился к костру и бросил в пламя фантики от конфет.

\diagdash Длина какая у всего этого?\mdash уточнил Адмирал.

\diagdash Километр двести.\mdash отозвался Замполит.

\diagdash Ну на то и <<Длинный>>! Явно не короткий.\mdash поязвил Серёга.

\diagdash Нормас. Дальше бегленько прочитай про остальные и начинаем паковаться.\mdash сказал Адмирал, чтобы отвлечь взбудораженную команду от мыслей о <<бочке>>.

Замполит продолжил:

\diagdash Дальше Корбикоски без особых каких\sdash то проблем. Потом порог Блин, где пишут, что можно сигануть с~плиты у~правого берега. Та\sdash а\sdash ак. Потом ещё один порог и~последний\mdash Ледяной, где никакой жести$\ldots$

\diagdash Кирь, на карте Блина нет, но есть Руозмикоски, я~помню там какая\sdash то путаница была в описаниях.\mdash Адмирал посмотрел на карту.

\diagdash Хм! Походу Блин\mdash это и есть Руозми\sdash что\sdash то там?\diagdash предположил Серёга.

\diagdash Руозмикоски. Ну или это первая или последняя ступень, соответственно, последующего или предыдущего порога. Короче так, давайте сначала Длинный пройдём, а~дальше по~ситуации?

\diagdash Ладно, про <<бочку>> уяснили, дальше всё равно ниче не запомним.\mdash заключил Паша.

\diagdash Пакуемся!\mdash распорядился Адмирал$\ldots$

\vspace{0.5cm}
$\ldots$На воду они встали уже когда перевалило за~12~часов. Со стоянки на острове не хотелось уходить\mdash вроде бы, на~первый взгляд с воды, неуютный берег открывался чудесной поляной у костра с завораживающими соснами и елями вокруг. Эдакое <<потайное>> место, открывшееся, вероятно, в~первый раз тем, кто не гнушался основательно вести разведку стоянок. Адмиралу вдруг подумалось\mdash а~как вообще люди ходили по маршрутам когда не было ни навигации спутниковой, ни нормальных карт, ни толком описания? Наверняка кто\sdash то когда\sdash то пошёл первым, да хоть и по их маршруту по Суне, на байдарках типа <<Салют>> или ещё чёрт знает на чём самодельном в отрыв от окружающей действительности и принёс первый отчёт по маршруту. Это~если <<по\sdash официальному>> от какого\sdash нибудь турклуба. А если <<дикарями>>, как они? <<Непостижимо, как люди не боялись идти в неизвестность порогов, не имея чёткого понимания о~них>>,\mdash думал Адмирал,\mdash <<и если уж они сдюжили, то мы\sdash то~просто обязаны>>. Кроме того, наверняка местное население веками ловило тут рыбу на порогах.

Замполит залил костёр и осмотрел стоянку:

\diagdash Готово дело, уходим! Шурик, готов?

\diagdash Готов. Осмотритесь как следует! Тут трава высокая, может что забыли или выронили. Ну и отчаливаем.\mdash и~пошёл к~берегу.

На берегу Руслан в своей сиреневой болоневой курточке помогал Паше доканчивать погрузку. Адмирал закатал штанины, зашёл в воду и докончил погрузку своей байды, закрыв грузовой отсек заглушкой, как можно туже завязав шнур, стягивющий эту самую заглушку. Все собрались на~берегу и были готовы. 

\diagdash Отчаливаем!\mdash велел Адмирал, садясь в~байдарку.

\diagdash Ага!\mdash отозвалась команда.

\diagdash Начинаем потихонечку. Помним, что~далеко не~расходимся, радиосвязи нет~больше.

\diagdash Ага-ага, ты давай первым, а~мы за~тобой.\mdash сказал Замполит.

\diagdash По классике, так сказать. Давайте, начали! Помним, <<бочка>> где\sdash то по центру, обходить справа.\mdash и Адмирал оттолкнулся от~берега веслом.

Две байдарки вышли в разлив после порога Каданлоама и дрейфовали\mdash все застёгивали юбки, проверяли замки спасжилетов, морально готовились. Миновав то место, откуда вчера удирали от засасывающего течения, они завернули за поворот реки и Длинный или, как ещё было в скобочках уточнено на адмиральской карте, Каменный, начался.

Последовало сужение русла, впереди выросли огроменные валы, Адмирал похолодел, но проходить\sdash то как\sdash то надо было, и направил их байдарку аккурат в валы, стараясь заходить максимально перпендикулярно к ним:

\diagdash {\large НА ВАЛЫ! РОВНО!}\mdash заорал Адмирал, а сам приготовился табанить для возможных манёвров.

БАХ! И вода залила Пашку, Руслана, окатила Адмирала на корме. БАХ!!! Ещё один вал. БАХ!!!!!! Ещё и~ещё! Высоченные валы они миновали отделавшись только лёгким испугом и брызгами. Дальше началась какая\sdash то сплошняковая белая каша воды. 

Экипаж молчал, а Адмирал соображал куда править дальше. Улучив момент, он обернулся и увидел, что Замполит заходит следом за ними в отдалении метров сто. Впереди начиналось очередное сужение, струя стремнины по центру косо уходила к правому берегу. <<Ага, там рекомендовали под правый берег>>,\mdash вспомнил Адмирал и взял чуть правее, довольный собой, думая как он сейчас обойдёт бурлящую кашу, что~была по центру и левее в русле. Справа по~курсу начинались прибрежные камни, и он молниеносно подправил их байдарку в такой узенький <<коридорчик>> между правым берегом и~косой стремниной$\ldots$ и тут же осознал, что~происходит\mdash их утягивало течением прямиком в <<бочку>>. Адмирал увидал ту самую плиту, о которой они прочли в~описании порога. За плитой стал виден водопад с бешеным бурлением, где притаилась коварная <<бочка>>. Скорость у них была аховая, всё это случилось настолько быстро, что Адмирал только и~успел, что рот открыть, чтоб заорать что\sdash нибудь матерное на~прощание, и~в~этот самый момент они рубанули байдаркой прямо по плите, почувствовался скользящий удар в днище и~байдарка носом зарылась в~<<бочку>>.

\diagdash {\LARGE ГРЕБЁМ!!!!!!!}\mdash орал Адмирал, сам тоже бешено лопатя веслом.

Их байдарка, как в замедленно кино



\begin{center}
	\psvectorian[scale=0.4]{88} % Красивый вензелёк :)
\end{center}
