%\chapter{<<Яичница, телек, герани цветы!>>}
\chapter{Мярандукса}
%\corner{64}
\vepsianrose

Адмирал проснулся, нащупал часы. Выспался он замечательно, ночь была тёплой и безветреной. На часах было около половины девятого, Киря ещё спал. Адмирал повернулся на другой бок и подумал, что с погодой им, пока что, относительно везёт\mdash ночь выдалась такой тёплой, что пришлось даже скинуть тельняшку.

\diagdash Что, пора?\mdash сонно протянул Замполит.

\diagdash С добрым утром! Пора!\mdash Адмирал натягивал штаны и собирался вылезать из палатки.

\diagdash С добрым, я поваляюсь чутка ещё.

\diagdash Я пошёл завтрак готовить!\mdash Адмирал выполз из~палатки и увидал Пашу, который уже с удочкой наготове шёл к берегу.

\diagdash Сань, чё на завтрак?\mdash Пашка насаживал наживку.

\diagdash Всё по классике, каша чемпионов!\mdash Адмирал хлопал себя по карманам штормовки в поисках чего\sdash нибудь прикурить.

Вскоре проснулись и Серёга с Русланом. Адмирал быстренько развёл костёр, сделал утренний чай, поставил кашу вариться, а сам принялся измельчать ножом орехи и~сухофрукты для добавления в~кашу\mdash тот самый завтрак чемпионов\mdash высококалорийная еда, на которой они и~держались, фактически, весь день до ужина, делая только один\sdash два перекуса батончиками спортпита вместо обеда. 

День разгорался тёплый, солнечный. И если с~самого утра на небе не было ни облачка, то сейчас, когда приготовление завтрака было почти окончено, в небе над озером повисли красивые кучерявые белые облачка. Настроение у всех было отличнейшим\mdash народ традиционно собрался в кружок у костра. Адмирал с Замполитом доделывали бутерброды с плавленым сыром и копчёной колбасой, на земле у костровища стояли миски в ожидании раздачи каши. Когда всё было готово, Адмирал взял крышку котелка и постучал по ней половником:

\diagdash К\sdash у\sdash у\sdash ушать! Айда на раздачу!\mdash он быстренько раскидал большим половником кашу.\mdash И бутеры не~забываем! Кушаем хорошо, сегодня у нас интересный насыщенный день!

\diagdash Отсюда поподробнее, п\sdash жалста!\mdash попросил Серёга, остужая кашу в миске, из которой шёл пар.

\diagdash Да, что у нас сегодня?\mdash Руслану, как и остальным, было интересно.

Шурик наворачивал овсянку с сухофруктами:

\diagdash Ща, дайте заправиться\sdash то!\mdash и, утолив лёгкий утренний голод и переведя дух, продолжил\mdash Сегодня продолжение <<Венеции>>\mdash ещё один канал, потом по цепочке озёр лопатим дальше, сегодня рек не будет.

\diagdash То есть течения не будет весь день, упахиваемся вёслами\mdash это я вам перевожу c адмиральского!\mdash хохотнул Паша.

\diagdash Не, упахиваться не будем, сегодня по плану пройдём меньше, чем вчера\mdash надо дать рукам втянуться в греблю. Если вчера мы все были на энтузиазме и подъёме, то сегодня ходовой день может показаться сложнее вчерашнего. Поэтому сегодня не более 10\thinspace\nobreakdash---\thinspace 12 километров. По стоячей озёрной воде этого будет вполне достаточно.\mdash закончил Адмирал и откинулся в стульчике, потягивая чаёк с бутером.

\diagdash А что насчёт каналов?\mdash уточнил Замполит.

\diagdash Ну, сегодня один канал точно, а дальше как пойдёт$\ldots$

\diagdash Шурик$\ldots$ Это твоё коронное <<как пойдёт>>$\ldots$

\diagdash Так! Тащ Замполит! Спецом для тебя поясняю\mdash это означает, что если пройти по каналам не удастся, то пойдём по короткой речушке, соединяющей Торосозеро с~озером Мярандукса.

\diagdash Во\sdash о\sdash от! Это уже чуть больше конкретики.\mdash Киря прикурил от костра и откинулся в стульчике.\mdash А то вечно у тя сюрпризы.

\diagdash Никаких сюрпризов, аллес унтер контролле! Так, ещё чаю поставьте\mdash надо залить в термосы и потихоньку начинаем паковаться.\mdash распорядился Адмирал.

Стоянку они покинули только в половину первого, сборы проходили совершенно лениво и не слишком организованно. Адмирал всегда учитывал эту особенность второго дня похода\mdash людям надо пройти слаживание, привыкнуть что и куда паковать, как размещать снарягу в байдарке. Ну и в целом, должен установиться ритм похода, на это тоже нужно время, примерно 2\thinspace\nobreakdash---\thinspace 3 дня. К~третьему дню команда, как правило, уже представляет собой более\sdash менее слаженный организм. 

Адмиральская байдарка была готова первой\mdash Руслан и Паша быстро собрались и перетаскали вещи, потом помогли Шурику всё распихать под борта. Серёга с Кирей всё еще возились с упаковкой, а когда настал их черёд погрузки герм в байдарку, Шурику пришлось отчалить и отойти от~берега\mdash места не хватило под зачаливание двух байдарок параллельно. Адмиральский экипаж отошёл от берега и дрейфовал. Внезапно ветер стал усиливаться, Адмирал оживился. Пашка обернулся на носу байды:

\diagdash Ты думаешь о том же, о чём и я?

\diagdash Ставь парус!!!\mdash отдал Адмирал команду, ему очень хотелось попробовать как пойдет байдарка под даже таким простым и небольшим парусом. Прямо перед походом он заказал складной парус, который представлял собой полусферу с прозрачной вставкой\mdash чтобы видеть, куда плывёшь\mdash и с тонким стальным тросом, вшитым по кругу для придания формы. Всю эту конструкцию можно было сложить в круг небольшого диаметра, так что сложенный парус легко помещался в грузовом отсеке байды.

\renewcommand*{\thefootnote}{\arabic{footnote}}
Пашка разобрался с такелажем\footnote{совокупность судовых снастей, в т.ч. для управления парусами\cite{МорскойСправочник}}, привязал шкоты\footnote{снасть бегучего такелажа, с помощью которой <...> оттягивают назад углы парусов\cite{МорскойСправочник}} к~стальному тросу, приладил нижнюю часть паруса на~носовом шпангоуте и, наконец, полностью подготовившись, передал Шурику назад своё весло и взял в руки шкоты, ловя ветер:

\diagdash И\sdash и\sdash иха! 

\diagdash КАЙФ!!! ПОТАЩИЛО!!!\mdash Адмирал в полном восторге перестал грести веслом по\sdash байдарочному и взял его под мышку на манер руля.\mdash Погнали!!!

Ветер дул, конечно, не особо сильный, но этого было достаточно, чтобы дать им ход в парочку км/ч. Второй экипаж, тем временем, наконец\sdash то отчалил и догнал их:

\diagdash Пацаны, классно смотритесь!\mdash Серёге и Кире понравилась идея паруса.\mdash Но мы вас сделаем!\mdash они налегли на~вёсла и обошли адмиральскую байдарку вперёд.

\diagdash Нас обходят!\mdash Руслан схватился за весло.

\diagdash Ща мы их догоним в два весла плюс ветер! Держи парус, Паш!\mdash Адмирал подналёг на весло и уже спустя пару минут они настигли второй экипаж.

Ветер стремительно тащил их на северо\sdash северо\sdash запад. Они так увлеклись парусом и гонкой, что позабыли про всё на~свете и уж конечно про исток речушки Сяпчи, соединявшей Сяпчозеро с Торосозером. Впрочем, Адмирал и не хотел идти по этой реке, надеясь пройти через канал, о~котором говорил команде с утра.

\diagdash Шурик! Там озеро кончилось!\mdash заявил Серёга, грёбший в почти полностью лежачем положении, вытащив и растянув свои ноги на носу кириной байды.

\diagdash Куды рулим, тащ Адмирал?\mdash Киря рыскал по~курсу.

\diagdash Паш, сворачивай парус, надо свериться с GPS, да~и~ветер стих к концу озера.\mdash сказал Адмирал и, перехватив весло, достал навигатор, который висел у него на шее через плечо на длинной верёвке.\mdash Хм, пацаны, а~канал\sdash то правее! Мы его незаметно просквозили как\sdash то.\mdash обескуражено произнёс он, копаясь в навигаторе.

\diagdash Там не было ничего, Шурик!\mdash лёг на новый курс Замполит.

\diagdash Издалека может не разглядели, давай поближе подойдём!\mdash Адмиралу был уверен, что канал на месте, поскольку, соотнеся пройденные каналы с фотографиями в~Интернете, он сделал вывод, что то, что было изображено на фото, они ещё не прошли, а, следовательно, канал в Торосозеро должен быть широким, глубоким и ярковыраженным. 

Спустя минут 15 эскадра вырулила к небольшим редким прибрежным зарослям\mdash из воды торчали тростинки. За~этими редкими зарослями начинался канал:

\diagdash А вот и он! Кирь, налегай!\mdash Серёга увидал канал и~они с Замполитом первые ворвались в его створ. Следом шёл адмиральский экипаж:

\diagdash Пацаны, снимайте на фото и видео! Где ещё такое увидите!\mdash Адмирал подруливал одной рукой веслом, а~второй умудрялся снимать видео на память, обозревая окрестности.

Канал был неплох\mdash шириной метров восемь, глубиной до полуметра, он простирался ни много ни мало на километр вперёд. Борта его были также выполнены из брёвен на~манер сруба, по берегам росла черника, брусника. Вид этого канала был ещё круче того, первого, который они прошли вчера. Экипажи притормозили, ухватившись за прибрежные заросли, и стали кушать чернику. Ягоды были крупными, спелыми, сочными:

\diagdash М-м-м! Вкуснятина!\mdash Серёга объедался с куста.\mdash Надо бы с собой набрать!

\diagdash Зачем? Я думаю её тут как грязи везде.\mdash ответил Адмирал.

\diagdash Не, в Москву, с собой набрать!

\diagdash Ы-ы-ы! Скиснет пока довезёшь!

\diagdash Точно$\ldots$ Ну тогда на вечер на компотик.

\diagdash Вот эт можно! Компотик эт хорошо.\mdash согласился Адмирал,\mdash Ну что, поели с куста? Поехали дальше!\mdash и~оттолкнулся от берега канала веслом.

Они плавно, почти не гребя, подходили к середине канала. Течение было слабым, но всё же было\mdash их желтые байдарки плавно покачиваясь, скользили вперёд по водной глади, столь редко разрезаемой в этих местах байдарочниками. На нос замполитовой байды сели две стрекозы\mdash показатель экологической чистоты местности:

\diagdash Серёг! Снимай стрекоз! Вон, на носу!

\diagdash Точно! Красотища!\mdash он перевёл фокус видео на~стрекоз.\mdash Благодать какая вокруг, пацаны!

\diagdash Да\sdash а\sdash а$\ldots$\mdash соглашался Адмирал.

Всем, надо полагать, нравилось подобное времяпрепровождение\mdash в спокойной тишине и наслаждении природой под ярким солнышком и лёгким ветерком. 

\diagdash Так, впереди мель и коряга!\mdash Серёга закончил с~видео и принялся выруливать левее по курсу. 

\diagdash Паш, не греби, я вырулю\mdash кинул вперёдсмотрящему Адмирал, и их маленькая эскадра успешно миновала коряги и мель посреди канала. Дальше ребята столь же безмятежно, как и до этого, потихоньку достигли конца канала.

\diagdash Тащ Адмирал, чё у нас дальше по плану?\mdash оживился Замполит.

\diagdash Дальше? Дальше ставим парус! П\sdash а\sdash а\sdash аш?

\diagdash Так точно, херр майор!\mdash Пашка развернул парус и на этот раз привязал его к веслу на манер мачты, чтобы не держать высоко руки со шкотами, что было утомительно. Конструкция вышла даже лучше прежней\mdash попутный ветер снова потащил их. 

\diagdash Ну а всё\sdash таки?\mdash не унимался Серёга.

\diagdash Дальше щас будет что\sdash то типа перекопа, короткого канала из Тороса в Мярандуксу.\mdash вещал Адмирал.

\diagdash А если нет?

\diagdash А если нет, то попрёмся левее, там короткая, на один километр примерно, речка, соединяющая два озера!

Так, ловя ветер парусом и слегка отставая при этом от~кириной байды, адмиральский экипаж шёл вторым. Но это нравилось им гораздо больше гребли\mdash Паша вообще не грёб, а держал своё весло вертикально с расправленным парусом, Руслан изредка подгребал, придавая им дополнительный импульс, а Адмирал только в основном выправлял курс веслом, корректируя вносимый ветром снос.

Не желая петлять как перед прошлым каналом, Адмирал положил GPS прибор себе на колени и время от времени сверял курс на следующий канал, идя под правым берегом озера Торос. Водная гладь была хороша и, хотя озеро и было небольшим, всего около полутора квадратных километров по площади, ребятам нравился даже такой простор. Впереди их ждала Мярандукса, такая же большая по площади, как и Сяпчозеро, на котором была их первая стоянка. 

Вскоре впереди замаячил ярко выраженный перешеек между озёрами, посреди которого виднелся проплыв. Сам перешеек был примерно метров 30\thinspace\nobreakdash---\thinspace 40 в ширину и метров 5\thinspace\nobreakdash---\thinspace 7 в~высоту\mdash берега были крутыми, почти неприступными. Над небольшим каналом, вид на который вскоре открылся взору сплавщиков, лежало поперёк упавшее дерево, так что команда проплыла под ним, как под мостом.

\diagdash Где пристанем?\mdash команде нетерпелось взобраться наверх.

\diagdash Давайте пройдём канал и зачалимся со стороны Мярандуксы, там, я думаю, должен быть спуск к воде, тут везде неудобно, сами видите!\mdash Адмиралу тоже очень хотелось осмотреть перешеек сверху\mdash там Шурик издали заприметил огромные сосны.

С обратной стороны перешейка берег оказался каменистым и крутым, но команде всё\sdash равно удалось причалить, надёжно закрепив чалки на деревьях. Шурик вытащил термос из гермы, снял штормовку, оставшись в~одной тельняшке, неторопливо прикурил и вскарабкался вместе с командой по обрыву наверх. Вид, открывшийся его взору, был шикарен\mdash прямо перед ними росла огроменная, в полтора обхвата, высоченная сосна с раскидистой кроной и~старыми, причудливо изогнутыми ветвями. Ветер продувал узкий перешеек насквозь, ребята стояли под огромной сосной:

\diagdash Мы нашли Иггдрасиль!\mdash огласил Адмирал.

\diagdash Чего?\mdash спросил Серёга.

\diagdash Иггдрасиль! Мировое дерево в мифологии древних скандинавов! Корни его уходят в Хельхейм, царство мёртвых, ствол его пронзает Мидгард, мир людей, а верхушка достигает Асгарда, обители воинов\sdash асов.\mdash вещал Шурик. 

\diagdash Похоже маленько$\ldots$

\diagdash Шурик, там ясень был, у скандинавов.\mdash сказал подошедший Киря.

\diagdash А у нас сосна будет! Мне всегда нравилась такая мифология\mdash древнегреческая, древнеримская, древнескандинавская$\ldots$ Красиво всё расписывали, черти!

\diagdash Да ты язычник?

\diagdash Это церковники христианские придумали\mdash язычники, язычники. Тьфу! Сами они язычники. У меня настольная книга была в школе\mdash <<Легенды и мифы Древней Греции и Древнего Рима>>\cite{Кун}. Потом на~Скандинавию переключился$\ldots$ Красиво же! А не всё это дерьмо в духе <<ударили по левой щеке, подставь правую>>, тьфу! Ударили по одной щеке\mdash достал меч и~порубал обидчика нахрен, ёпрст!\mdash Адмирал выругался и~сплюнул.\mdash А так прав был дедушка Ленин: <<Религия есть  опиум для народа!>>\cite{ЛенинОпиумДляНарода}. 

\diagdash Эк заворачиваешь! Ты лучше гляди какие грибы растут!\mdash Серёга притащил 2 огромных подберёзовика, диаметром шляпки около двадцати пяти сантиметров.  

\diagdash Мощно, мощно! Ты смотри, ты смотри! Прям под ногами!\mdash Шурик сделал шаг в сторону, нагнулся и поднял ещё один небольшой подберёзовик.\mdash Офигеть! Сегодня жарёху замутим вечером тогда!\mdash сказал Адмирал, увидав сколько грибов притащил Серёга. 

\diagdash Там дальше столик, костровище.\mdash сказал вернувшийся из разведки Паша и показал дальше вдоль по перешейку. Ребята пошли глянуть, но место никому не~понравилось, потому что перешеек насквозь продувался ветром, гулявшим между озёрами, к тому что на ночёвку становиться было ещё рано. 

Шурик вернулся к <<Иггдрасилю>> и зачем\sdash то обнял дерево. Ему было так хорошо, что хотелось навеки раствориться в красоте окружавшей его северной природы, природы, которую он так любил. Он вдохнул аромат хвои и смолы, исходивший от этой исполинской сосны. Этот запах, звучание которого глубоко засело в мозгу ещё с детства, со~времён вылазок в окрестные с деревней владимирские леса, всегда будоражил в Шурике какие то приятные, но в то же время первобытные чувства, заставлял хоть на секундочку очутиться в детстве$\ldots$ Сосны, тёплый запах смолы, мшаник, прибрежные камни и песочек, лёгкий ветер и иссиня\sdash синее августовское небо уносили Шурика в его какую\sdash то другую реальность. На Землю заставили спуститься лишь голоса команды:

\diagdash Ну что, Адмирал, идём дальше?\mdash Серёга и Киря собрали грибы в пакетик на вечер и готовы были спускаться к байдаркам.

\diagdash Да, щас пойдём.\mdash Шурику не хотелось уходить от <<Иггдрасиля>>, он всё ещё стоял, прислонившись к нему. Паша и Руслан пошли вниз к байдаркам, пора было отчаливать.

Ребята аккуратно спустились с перешейка к своим кораблям, уселись, отчалили. Адмирал отходил от берега кормой, совершая разворот на 180 градусов по курсу через берег, чтобы ещё раз окинуть взором шикарной красоты место, которое они покидали. 

\diagdash Тащ Адмирал, впереди просто Атлантика! Куда идём?\mdash поинтересовался Замполит.

\diagdash Кильватерной колонной за мной! Держим курс по~спутниковой навигации!\mdash отозвался Адмирал.\mdash Впереди мыс виднеется слева\mdash правим на него, там должен открыться простор озера Мярандукса во всей красе!

\diagdash Мярандукса, прикольное название.\mdash сказал Руслан.

\diagdash Ну да, тут деревня была раньше не правом берегу с~таким же названием. Или у озера такое же название, как у~деревни, поди разбери теперь. На месте деревни уже ничего нет, так что пойдём тут, под левым берегом.\mdash заключил Адмирал.

\diagdash Сань, ветер не очень, парус не ставлю?\mdash Пашке, естественно, очень понравилось сидеть рулить парусом.

\diagdash Ага, ветер почему\sdash то почти встречный, хотя на прошлом озере был попутный. В лавировку с таким парусом не пойдёшь$\ldots$\mdash сказал Адмирал и погрузился в свои мысли о парусе. Ему безумно хотелось хоть когда\sdash нибудь выйти да даже хоть и на такое озеро, но под более менее приличным парусом, квадратов на 4 или 5 площади. Юрич всё убеждал его сделать парусное вооружение на байдарку, но Шурику хотелось непременно парусный катамаран$\ldots$

Вскоре ребята достигли мыса\mdash им открылся вид на остальной простор озера, который был закрыт мысом до этого момента. У всех дух перехватило от простора и красотищи:

\diagdash Ух!\mdash не то с восторгом, не то маленько со страхом сказал Серёга.

\diagdash Ух-ух!\mdash передразнил Шурик.\mdash Правим во-о-он на те островки, обойдем их справа.\mdash провёл краткий инструктаж Адмирал.

\diagdash Каков расчёт?\mdash осведомился замполит.

\diagdash Править к левому берегу, там есть один мыс характерный, сто процентов там есть хорошая стоянка. Под правым берегом высоковато, не хочу там идти, да и сосны нас там от солнца скроют.

\diagdash Ну лады! Кто быстрее?\mdash и Замполит принялся лопатить веслом.

\diagdash Врёшь! Не возьмёшь! У нас три весла! А ну, парни, сделаем этого выскочку!\mdash Адмирал скомандовал полный вперёд и они в режиме гонки дошли до скопления маленьких островов посреди Мярандуксы. Догнать Замполита не вышло\mdash его двушка была менее загружена и имела меньшую осадку, недели чем адмиральская трёшка. Парни шли в полукилометре от берега, временами было не очень комфортно осознавать себя вдали от берега, но это чувство быстро проходило за греблей. Наконец они достигли последнего маленького островка среди этого небольшого архипелага и прошли совсем рядом с ним:

\diagdash Заберёмся?\mdash спросил Серёга.

\diagdash Нафига?\mdash парировал Адмирал.

\diagdash Ну просто, интересно.

\diagdash ТАм заросли такие, да и остров настолько мелкий, что стоянку на нём не устроить, пошли дальше!

В районе следующего острова, спустя минут 15, они скорректировали курс эаскадры, потому что Адмирал чуть отвлекся от навигатора и перепутал на какой мыс впереди править\mdash по итогам оказалось, что на ближний. Около него вскоре издалека показалась то\sdash о\sdash оненькая полосочка песка, пляж.

\diagdash А? Что я вижу? Пляжик, тащ Адмирал!!!\mdash Замполит приободрился и налёг на весло.

\diagdash А то?\mdash подмигнул тому Адмирал.\mdash Я плохой стоянки не посоветую!

Они бодро налегли на вёсла, держа курс на выбранный мыс. Однако по мере приближения к пляжу Адмирал понял, что не всё так радужно\mdash пляж был в стороне от мыса и тропинки между ними не просматривалось. В~половину четвёртого эскадра зашла в маленький заливчик, заканчивающийся пляжиком, и выбросилась байдарками на песок, ребята вылезли. Водичка в озере бодрила.

\diagdash Так! Матросы охраняют байды, поскольку чалиться тут особо не за что. Паш, Киря\mdash за мной!\mdash и они пошли пробираться через заросли возвышенности на мысу, где по адмиральским предположениям, должно было быть стояночное место.

\diagdash Шурик, куда мы прёмся? Место\mdash отстой! Болотина какая\sdash то!\mdash изрёк замполит, заплетаясь ногами и стараясь не грохнуться в чёрную мокрую грязищу. От их поступи в размокшем грунте раздавалось чавканье.

\diagdash Не гунди, вон уже подъем начинается!\mdash Адмиралу тоже было тяжко, но ему больно хотелось поглядеть на~стоянку. 

Вскоре они взобрались по подъему от болотины наверх. Адмирал чуть не ахнул\mdash рай на Земле!
















\vspace{5 cm}

По Мярандуксе пошли ближе к левому берегу, мимо островов до места предполагаемой стоянки, где виднелась узкая полоска песка. Высадились там, а после нее болотина и подъем наверх. Наверху отличнейшее место для стоянки - со столиком. 

\begin{center}
	\psvectorian[scale=0.4]{88} % Красивый вензелёк :)
\end{center}
