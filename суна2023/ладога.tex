\chapter{Варяги}
%\corner{64}
\vepsianrose
\fancyhead[LE]{\fancyplain{}{\bfseries \parttitle}}
\fancyhead[RO]{\fancyplain{}{\bfseries \rightmark}}

\diagdash Так, отряд, подъём!!!\mdash громогласно огласил Адмирал.

\diagdash Шурик, имей совесть, дай поспать!!!

\diagdash Вперёд! Нас ждут великие дела! Сегодня по плану осмотр крепости и ещё 400 километров до Карелии!

\diagdash Кто сёдня дежурный по завтраку, на?

Команда тяжеловато выползла из номера и спустилась на кухоньку завтракать\mdash гостиница была и впрямь скромная, но на кухне имелась посуда, техника, короче всё что нужно. Ребята быстренько перекусили и двинули в~сторону крепости пешком\mdash идти было около километра.

\diagdash А давай вон туда?\mdash Серёга показал на ответвление к реке через парк, хотя к крепости было прямо.

\diagdash Нафига? Нам прямо$\ldots$\mdash изнывая от жары отбивался Адмирал.

\diagdash Забей, пойдём глянем реку! Поздороваемся, так сказать!\mdash настаивал Серёга.

И они прошли сквозь парк вдоль стены монастыря к~реке. Вид на реку Волхов был, надо признать, шикарен. Берег утопал в буйной растительности\mdash непролазные заросли кустов и прибрежной травы устилали весь подъём от~воды. Степенные в\'{о}ды ползли, как и сотни лет назад, вперёд к~Ладожскому озеру.

\mdash Давай вдоль монастыря срежем?

\mdash Ой не нравится мне это$\ldots$\mdash попробовал отбиться Адмирал

\mdash Не гони, давай срежем угол.

Ребята пошли вдоль стены монастыря, обращенной к~реке. Примерно посредине той стены им встретились старые монастырские ворота. Вероятно, тут когда\sdash то был причал. Однако сейчас выхода к воде не было\mdash сплошные заросли. Вскоре впереди показалась крепость. Команда миновала узкую тропинку, идущую вдоль монастыря, и~вышла к~скверу, в торце которого обнаружился памятник Рюрику и~Вещему Олегу. 

\diagdash Монументально!\mdash Замполит обошёл кругом памятник.

\diagdash А то! Цветмета сколько, натурально!!!

Дальше друзья спустились к смотровой площадке, с~которой открывался замечательный вид на Стрелочную башню крепости, названную так, собственно, потому что она стоит на самом мысу, на стрелке, у впадения реки Ладожки в~Волхов\mdash просто идеальнейшее место для оборонительного сооружения\mdash c двух сторон естественное препятствие для атакующих\mdash река.

Подкатил туристический автобус, из которого высыпалась группа с экскурсоводом, тут же начавшим свою нудятину:

\diagdash Здесь вы можете видеть$\ldots$

Ребята, нафоткавшись на фоне башни, поспешили из~сквера к мосту, чтобы уже перейти к подножию самой крепости.

Если крепость, когда они проезжали вчера мимо неё, казалась какой\sdash то маленькой, как игрушка почти, то~сейчас, стоя у подножия её стен, ребята воочию могли убедиться, что это не так. Высоченные стены и грозные башни, жалко, конечно, что восстановленные, производили самое суровое впечатление. Шурик вдруг на~минуту подумал о~том, как тяжело, если не невозможно, было в~IX\thinspace\nobreakdash---\thinspace X~веке штурмовать такое сооружение и какое угнетающее впечатление производило всё это на атакующих.

Восстановление крепости было начато ещё в советское время, но тогда успели сделать только 2 башни. Сейчас же, к 900\sdash летию Старой Ладоги, восстановили почти всё, кроме крепостной стены, обращённой к Волхову и земляного городища, располагавшегося уже за пределами крепости к~югу. Современный вид крепости со стороны дороги и вовсе шикарен\mdash наверное почти не уступает историческому. 

\diagdash Нет, не скифы мы, не азиаты мы! Мы\mdash варяги, мать их ети!\mdash распалялся Шурик, воображение которого разыгралось не на шутку от увиденной крепости.

\diagdash Пошли, варяг, блин, нашёлся!

В крепости была устроена отличная экспозиция\mdash внутри башен. Ребята посетили все доступные места, облазили стены и вышли на самый верх Раскатной башни, где была устроена смотровая площадка. Шурик сделал панорамный снимок этой красоты и поспешил за командой, которая уже собралась во внутреннем дворике крепости, около церкви.

Осмотрев в крепости, пожалуй, всё, что можно было осмотреть, команда двинулась на выход\mdash время было к~обеду, осмотр отнял прилично времени, но его было не~жаль\mdash крепость действительно того стоила. 

По дороге обратно Шурику стало маленько дурно\mdash яркое солнце напекло голову, жара вокруг стояла приличная:

\diagdash Кирь, а ты боялся холодов!\mdash изрёк Адмирал, умирая от жары.

\diagdash Погоди, до Карелии ещё не добрались!

Команда вновь разделилась\mdash Серёга и Руслан отправились обедать в местный ресторанчик, а Киря, Шурик и Паша решили двинуть дальше в путь и отобедать где\sdash нибудь по пути. В машине их ждал спасительный кондиционер.

\diagdash Так! Тащ Замполит!

\diagdash Я!

\diagdash Поищи нам где отобедать по пути, чёт голодно.

\diagdash Ну это уже в Сясьстрое наверно.

\diagdash Далеко?

\diagdash Момент$\ldots$\mdash Замполит, он же Штурман, он же Киря копался в навигаторе.\mdash Через 25 минут будем на месте!

\diagdash Там есть что\sdash нибудь приличное?\mdash жалобно подал сзади голос Паша.

\diagdash Кафе <<Встреча>> приветствует вас!!!\mdash огласил Киря.

Друзья отобедали в очень приличном кафе самообслуживания и двинули дальше, созвонившись с~остальной частью команды\mdash те только выползли из~трапезной в Старой Ладоге. Вскоре, в Лодейном Поле, друзья пересекли реку Свирь по огромному разводному мосту, весь центральный пролёт которого может подниматься горизонтально вверх для пропуска высоких судов. Мост впечатлил\mdash махина что надо!

Дальше начались уже типично карельские пейзажи\mdash хвойные леса с мшаниками, узкая дорога. До Петрозаводска было, казалось, рукой подать, но дорога выматывала монотонностью. После 4 часов дня Шурик стал ну просто засыпать за рулём, один раз даже наехал правыми колёсами на акустические <<пробудители>>:

\diagdash Э\sdash э\sdash э! ШУРИК!!! Не спать!\mdash всполошился Киря.

\diagdash Не сплю, не сплю, всё норм!\mdash сам ошалевая от звука и вибрации пробудителей очнулся Шурик. Только сейчас он заметил, что уже давно, с полчаса точно, вдоль правого края дороги часто шли прямоугольные выемки\mdash он ещё подумал зачем они нужны. Теперь на собственной шкуре понял зачем.

\diagdash Хочешь, я поведу?\mdash спросил Паша и они сделали остановку на 5 минут перекурить и поменяться местами. После Шурик устроился на переднем пассажирском и уже спустя минут 15 его просто вырубило\mdash походу, таки напекло солнышко, пока друзья ходили по Староладожской крепости$\ldots$

\vspace{0.5cm}

$\ldots$очнулся Шурик уже за Петрозоводском. Паша сел на хвост какому\sdash то дальнобою и они плелись неспеша. 

\diagdash Вот это меня рубануло, пацаны$\ldots$\mdash Шурик тёр усталые глаза.

\diagdash Ну да, скоро Кондопога.\mdash отозвался сзади Замполит.

\diagdash А давайте на Кивач заедем?\mdash предложил Шурик.\mdash Когда ещё тут будем\sdash то?

\diagdash Это по пути? Там водопад и всё?

\diagdash Агась. Вообще\sdash то это один из самых больших водопадов Карелии. Я хотел потом, на выброске, туда зарулить, но что\sdash то мне подсказывает, что лучше сейчас.\mdash заключил Шурик.

\diagdash Потом уже будет домой хотеться, так что давайте реально щас.\mdash подал сзади голос Замполит.

\diagdash А поехали! Напиши Серёге, чтоб тоже заезжал.\mdash согласился Паша.

Вскоре ребята свернули у указателя <<Кивач>> и~запарковались перед въездом на территорию. Совсем скоро подкатила и вторая часть команды и они вместе пошли в~заповедник, приобретя билеты.

Вечерело, было около половины восьмого, все лавочки с сувенирами и прочим были закрыты. Путешественники вышли к водопаду, рёв которого начинался уже за сотню метров. Кивач был шикарен, кто бы что ни говорил! Несметные толщи воды низвергались с десятиметровой высоты, протекая между скал и разбиваясь внизу на~миллиарды миллионов белоснежных брызг, окатывающих синеющие скалы. Рёв вблизи был просто оглушающим\mdash чувствовалась мощь природной стихии. Смотровая площадка располагалась у второй ступени водопада, самой высокой, а немного выше была видна предыдущая ступень с чуть меньшим водопадиком. Друзья забрались в самый дальний край топинки на гору, где была установлена лавочка и~открывался чудесный вид вниз на водопад и долину. Присели на лавочку передохнуть и подождать Серёгу с~Русланом. 

Потом друзья пошатались по дендрарию, посетили памятник павшим советским воинам во время боёв в~Карелии в~Великой Отечественной Войне, пошли не торопясь назад. Высокие сосны вокруг памятника, как и всегда в таких местах, угнетающе давили на сознание$\ldots$ 

Времени уже было к закрытию заповедника и все сувенирные лавочки закрылись\mdash пришлось обойтись без традиционных магнитиков на холодильник. Шурик всё ещё ощущал последствия полуденного теплового удара и ему было нехорошо, так что плёлся он позади всех, не забывая фотографировать окрестности. Что ж, снова в путь\mdash друзья вернулись к машинам и резво рванули преодолеть последнюю часть пути до Гирваса, в который они и прибыли в половину десятого вечера.

Запарковались у гостиницы, проехав через почти весь посёлок в район Пальозёрской ГЭС. Расположились, перетаскали вещи. Пашка начал накачиваться пенным уже таская шмот, остальные присоединились почти сразу же\mdash усталость так и валила всех с ног. А надо было ещё перепаковать снарягу, но сначала все решили перевести немного дух\mdash длинная дорога давала о себе знать. Команда разбрелась по большому номеру, состоявшему из большой комнаты, террасы и санузла.

Шурик первым занял душевую по адмиральскому праву и освежился\mdash водичка была еле тёплой и мощно взбодрила Адмирала так, что он даже замёрз, выйдя оттуда.

\diagdash Бр-р-р, пацаны! Водичка огонь! Давай, кто следующий?

\diagdash Совсем огонь?\mdash поинтересовался Киря.

\diagdash Ну потеплее, чем завтра в реке, ы\sdash ы\sdash ы!!!\mdash заржал Адмирал, доставая, чтобы утеплиться, из вещмешка цветастое перуанское пончо из шерсти альпаки.

\diagdash Пацаны, Шурик а-ля Клинт Иствуд!

\diagdash А то! Перефразируя его фразу: все сплавщики делятся на два типа\mdash на Адмирала и на тех, кто перепаковывает вещи, ы\sdash ы\sdash ы!\mdash выделывался Шурик.

\diagdash Ща перепакуем, всё будет в наилучшем виде!\mdash Замполит сновал по комнате с пивной банкой.

Адмирала как осенило:

\diagdash Замполит!!! Знамя полка!

Киря расправил красное полотнище и они с Шуриком грянули что есть мочи, прихлебнув пенного:

\vspace{0.1cm}
\noindent\textit{%
	\hspace*{1.2cm}Но на фуражке на моей\mdash серп и молот и звезда,\\
	\hspace*{1.2cm}Как это трогательно\mdash серп и молот и звезда,\\
	\hspace*{1.2cm}Лихой фонарь ожидания мотается\\
	\hspace*{1.2cm}И всё идёт по плану$\ldots$\\
	\hspace*{1.2cm}\large{ВСЁ ИДЁТ ПО ПЛАНУ!}\\
	\hspace*{1.2cm}\Large{ВСЁ ИДЁТ ПО ПЛАНУ!!!}\\
	\hspace*{1.2cm}\LARGE{ВСЁ ПО ПЛА-А-АНУ!!!!!}
}

\newpage
\diagdash Панки хой!!! Хэви метал форева!!! У\sdash у\sdash х\sdash у\sdash х\sdash у!!!\mdash орала команда.

Всё шло своим чередом\mdash народ по очереди стал освежаться в душе. Сейчас оттуда доносились восторженные возгласы Замполита по поводу температуры воды. Серёга разлёгся на большой кровати, Руслан пока перекусывал. Паша решил перепаковать герму и вывалил всё на свою кровать на терраске. Шурик же, укутавшись в пончо, чуть передохнул и согрелся. Надо было собраться с мыслями, всё доделать и лечь спать. Киря, тем временем, вынырнул из~душевой:

\diagdash Ж-ж-жесть! 

\diagdash С лёгким паром, ы-ы-ы!

Следующий пошёл закаляться в душ, а Киря переоделся и распотрошил свои гермы и сумку. 

\diagdash Тащ Замполит! Давай сюды сухофрукты, орехи, энергетический батончики! Перепакуем в вещмешки всё как надо!\mdash вдруг очнувшись, стал командовать Адмирал. Пенное как будто не усыпило его, а придало импульс дожить этот день. 

\diagdash Лови!!!\mdash отозвался тот и швырнул пакеты. Шурик поймал и уложил орехи и сухофрукты в видавшие виды и выцветшие от~времени вещмешки. Питательные батончики спортивного питания решили паковать как есть\mdash в~картонных коробочках, чтобы те не помялись.

\diagdash Адмирал! Ещё мой протеин!\mdash гулко выпалил Замполит, разбирая свой шмот.

\diagdash Ты издеваешься?!\mdash отозвался Адмирал.\mdash Качок хренов! Ладно раньше надо было девок кадрить, а щас то что? Женился ж уже?! Сейчас\sdash то нахрена?\mdash Шурик был недоволен наличием 5\sdash литровой банки со спортивным питанием\mdash она никуда не влезала.

\diagdash Как нахрена? Жену молодую впечатлять, ну чё ты!\mdash хором отозвались Серёга и Руслан и всё дружно подогрето заржали.

\diagdash Ы!\mdash отозвался Паша.\mdash Там и компоты есть в~мешках, 100 процентов! Потроши мешки!!!\mdash не унимался рыбак и носовой гребец.\mdash А~бензопилу мою не взяли, ироды!!! У\sdash у\sdash у!!!\mdash и, шатаясь, потопал в душ.

Народ, тем временем, казалось, одновременно был занят перепаковкой вещей, ужином, пенным, помывкой в~д\'{у}ше. К~полуночи, когда последние приготовления были закончены, а пенное выпито, народ стал укладываться. Адмирал, забив себе, как ему казалось, самую кошерную кровать, лёг и, достав из малой гермы свой командирский планшет и~судовой журнал, стал делать путевые заметки\mdash расписал сегодняшний и вчерашний день, поведал о планах на~завтра, зафиксировал время по своим китайским ролексам и, напрочь окончательно устав, провалился в такой глубокий и тёмный молодецкий сон, какой только может быть у~человека, который завтра должен будет возглавить эскадру. Спать оставалось часов шесть.

\begin{center}
	\psvectorian[scale=0.4]{88} % Красивый вензелёк :)
\end{center}

 






