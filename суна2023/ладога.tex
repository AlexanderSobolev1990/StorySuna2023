\chapter{Варяги}
%\corner{64}
\vepsianrose
\fancyhead[LE]{\fancyplain{}{\bfseries \parttitle}}
\fancyhead[RO]{\fancyplain{}{\bfseries \rightmark}}

\diagdash Так, отряд, подъём!

\diagdash Шурик, имей совесть, дай поспать!!!

\diagdash Вперёд! Нас ждут великие дела! Сегодня по плану осмотр крепости и ещё 400 километров до Карелии!

\diagdash Кто сёдня дежурный по завтраку, на?

Команда тяжеловато выползла из номера и спустилась на кухоньку завтракать\mdash гостиница была и впрямь скромная, но на кухне имелась посуда, техника, короче всё что нужно. Ребята быстренько перекусили и двинули в сторону крепости пешком\mdash идти было около километра.

\diagdash А давай вон туда?\mdash Серёга показал на ответвление к реке через парк, хотя к крепости было прямо.

\diagdash Просто, интересно$\ldots$

\diagdash Ну пойдём$\ldots$

И они прошли сквозь парк вдоль стены монастыря к реке. Вид на реку Волхов был шикарен. Берег утопал в буйной растительности\mdash непролазные заросли кустов и прибрежной травы устилали весь подъем от воды. 

\mdash Давай вдоль монастыря срежем?

\mdash Ой не нравится мне это$\ldots$\mdash попробовал отбиться Адмирал

\mdash Не гони, давай срежем угол.

Ребята пошли вдоль стены монастыря, обращенной к реке. Примерно посредине той стены им встретились старые монастырские ворота. Вероятно, тут когда\sdash то был причал. Однако сейчас выхода к воде не было\mdash сплошные заросли. Вскоре впереди показалась крепость. Команда миновала узкую тропинку, идущую вдоль монастыря, и вышла к скверу, в торце которого обнаружился памятник Рюрику и Вещему Олегу. 

\diagdash Монументально!

\diagdash А то! Цветмета сколько, натурально!!!

Дальше друзья спустились к смотровой площадке, с которой открывался замечательный вид на Стрелочную башню крепости, названную так, собственно, потому что она стоит на самом мысу, на стрелке, у впадения реки Ладожки в Волхов\mdash просто идеальнейшее место для оборонительного сооружения\mdash c двух сторон естественное препятствие для атакующих\mdash река.

Подкатил туристический автобус, из которого высыпалась группа с экскурсоводом, тут же начавшим свою нудятину:

\diagdash Здесь вы можете видеть$\ldots$

Ребята, нафоткавшись на фоне башни, поспешили из сквера к мосту, чтобы уже перейти к подножию самой крепости.

Если крепость, когда они проезжали вчера мимо неё, казалась какой\sdash то маленькой, как игрушка почти, то сейчас, стоя у подножия её стен, ребята воочию могли убедиться, что это не так. Высоченные стены и грозные башни, жалко, конечно, что восстановленные, производили самое суровое впечатление. Шурик вдруг на минуту подумал о том, как тяжело, если не невозможно, было в IX--X~веке штурмовать такое сооружение и какое угнетающее впечатление производило всё это на атакующих.

Восстановление крепости было начато ещё в советское время, но тогда успели сделать только 2 башни. Сейчас же, к 900\sdash летию Старой Ладоги, восстановили почти всё, кроме крепостной стены, обращённой к Волхову и земляного городища, располагавшегося уже за пределами крепости к югу. Современный вид крепости со стороны дороги и вовсе шикарен\mdash наверное почти не уступает историческому. 

\diagdash Нет, не скифы мы, не азиаты мы! Мы\mdash варяги, мать их ети!\mdash распалялся Шурик, воображение которого разыгралось не на шутку от увиденной крепости.

\diagdash Пошли, варяг, блин, нашёлся!

В крепости была устроена отличная экспозиция\mdash внутри башен. Ребята посетили все доступные места, облазили стены и вышли на самый верх Раскатной башни, где была устроена смотровая площадка. Шурик сделал панорамный снимок этой красоты и поспешил за командой, которая уже собралась во внутреннем дворике крепости, около церкви.

Осмотрев в крепости, пожалуй, всё, что можно было осмотреть, команда двинулась на выход\mdash время было к обеду, осмотр отнял прилично времени, но его было не жаль\mdash крепость действительно того стоила. 

По дороге обратно Шурику стало маленько дурно\mdash яркое солнце напекло голову, жара вокруг стояла приличная:

\diagdash Кирь, а ты боялся холодов!\mdash изрёк Адмирал, умирая от жары.

\diagdash Погоди, до Карелии ещё не добрались!

Команда вновь разделилась\mdash Серёга и Руслан отправились обедать в местный ресторанчик, а Киря, Шурик и Паша решили двинуть дальше в путь и отобедать где\sdash нибудь по пути. В машине ждал спасительный кондиционер.

\diagdash Так! Тащ Замполит!

\diagdash Я!

\diagdash Поищи нам где отобедать по пути, чёт голодно.

\diagdash Ну это уже в Сясьстрое наверно.

\diagdash Далеко?

\diagdash Момент$\ldots$\mdash Замполит, он же Штурман, копался в навигаторе.\mdash Через 25 минут будем на месте!

\diagdash Там есть что\sdash нибудь приличное?\mdash подал сзади голос Паша.

\diagdash Кафе <<Встреча>> приветствует вас!!!\mdash огласил Киря.

Друзья отобедали в очень приличном кафе самообслуживания и двинули дальше, созвонившись с остальной частью команды\mdash те только выползли из кафе в Старой Ладоге. Вскоре, в Лодейном Поле, друзья пересекли реку Свирь по огромному разводному мосту, весь центральный пролёт которого может подниматься горизонтально вверх для пропуска высоких судов. Мост впечатлил\mdash махина что надо!

Дальше начались уже типично карельские пейзажи\mdash хвойные леса с мшаниками, узкая дорога. До Петрозаводска было, казалось, рукой подать, но дорога выматывала монотонностью. После 16 часов Шурик стал ну просто засыпать за рулём, один раз даже наехал правыми колёсами на акустические <<пробудители>>:

\diagdash Э\sdash э\sdash э! ШУРИК!!! Не спать!\mdash всполошился Киря.

\diagdash Не сплю, не сплю, всё норм!

\diagdash Хочешь, я поведу?\mdash спросил Паша и они сделали остановку на 5 минут перекурить и поменяться местами. После Шурик устроился на переднем пассажирском и уже спустя минут 15 его просто вырубило\mdash походу, таки напекло солнышко, пока друзья ходили по Староладожской крепости$\ldots$

\vspace{0.5cm}

$\ldots$очнулся Шурик уже за Петрозоводском. Паша сел на хвост какому\sdash то дальнобою и они плелись неспеша. 

\diagdash Вот это меня рубануло, пацаны$\ldots$\mdash Шурик тёр усталые глаза.

\diagdash Ну да, скоро Кондопога.\mdash отозвался Замполит.

\diagdash А давайте на Кивач заедем?

\diagdash Это по пути? Там водопад и всё?

\diagdash Агась. Вообще\sdash то это один из самых больших водопадов Карелии. Я хотел потом, на выброске, туда зарулить, но что\sdash то мне подсказывает, что лучше сейчас.\mdash заключил Шурик.

\diagdash А поехали! Напиши Серёге, чтоб тоже заезжал.\mdash согласился Паша.

Вскоре ребята свернули у указателя <<Кивач>> и запарковались перед въездом на территорию. Совсем скоро подкатила и вторая часть команды и они вместе пошли в заповедник, приобретя билеты.

Вечерело, было около половины восьмого, все лавочки с сувенирами и прочим были закрыты. Путешественники вышли к водопаду, рёв которого начинался уже за сотню метров. Кивач был шикарен, что бы кто ни говорил! Несметные толщи воды низвергались с десятиметровой высоты, протекая между скал и разбиваясь внизу на миллиарды миллионов белоснежных брызг, окатывающих синеющие скалы. Рёв вблизи был просто оглушающим\mdash чувствовалась мощь природной стихии. Смотровая площадка располагалась у второй ступени водопада, самой высокой, чуть выше была видна первая с чуть меньшим водопадиком. Друзья забрались в самый дальний край топинки на гору, где была установлена лавочка и открывался чудесный вид вниз на водопад и долину. Сели передохнуть и подождать Серёгу с Русланом.



 






