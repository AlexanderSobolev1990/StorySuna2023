%\chapter{А это уже или ещё не?}
\chapter{Нормас?}
%\corner{64}
\vepsianrose

\diagdash Так! По плану сегодня речное приключение\mdash мы должны пройти речку Нурмис от начала, а она вытекает из Мярандуксы, до конца, то есть до Линдозера\mdash Шурик привычно варил утреннюю кашу чемпионов, Киря сидел рядом в весьма абстрактном виде.\mdash Что, нормас, Замполит? Капнуть пять капель, взбодришься?

\diagdash Лучше компотик$\ldots$

\diagdash То-то же!

Остальные тоже подтянулись к утреннему костру:

\diagdash Ну ты зажёг вчера, Кирь!\mdash Серёга зачерпнул вчерашний компотик.\mdash Как сегодня?

\diagdash Да вроде ничего$\ldots$

\diagdash Щас сала с утра и кашу чемпионов и будет грести быстрее всех. А, чуть не забыл, ещё протеин, да, Кирь?\mdash Шурик не преминул ввернуть про протеиновый бочонок.

\diagdash Ага$\ldots$

\diagdash Отстаньте от человека, все вчера отличились.\mdash Пашка сидел у~стола, тёр глаза и~делал потихонечку бутеры.

\diagdash А как речка называется, ещё раз?\mdash уточнил Серёга.

\diagdash Нурмис!\mdash Адмирал снял с~костра котелок с~кашей.\mdash Налетай, становись, разбирай живопись! Тару сюда!

\diagdash Нурмис? Норм{\'а}с!\mdash пошли упражнения в~остроумии.\mdash Норм{\'а}с с утра?  Норма\sdash а\sdash а\sdash а\sdash ас!

\diagdash Нур-султан, ёпрст, хар{'о}ш, мужики, давате позавтракаем.\mdash Пашка наворачивал кашу и стрелял конфеты вприкуску с подоспевшим чайком.

Потом ребята сделали ещё полный котелок чая и залили его в термосы на день, стали потихоньку сворачивать лагерь, что было в их состоянии непросто, работа шла со скрипом, медленно. Полностью подготовиться к~отплытию вышло уже только к одиннадцати часам.

Адмиральский экипаж был более боеспособен и потихоньку стартанул, пока Киря с Серёгой ещё усаживались. 

\diagdash Ну где они там?\mdash Адмиралу было лень оборачиваться, он подналёг правым находом, чтобы немного развернуться и посмотреть назад.

\diagdash Копаются еще!\mdash ответил Руслан.

\diagdash Долго чёт!

\diagdash Щас догонят, пошли потихоньку.

И они снова встали на нужный курс и потихоньку, абсолютно не напрягаясь, пошли по озеру. Спустя полчаса Киря с Серёгой нагнали их.

\diagdash Адмирал, у нас конкуренты!!!\mdash сказал Замполит.

\diagdash А?\mdash не понял Шурик.

\diagdash Оглянись!

Шурик обернулся в пол оборота и увидел огроменную флотилию из байдарок, катамаранов, надувных лодок, растянувшуюся по озеру чуть не на полкилометра. 

\diagdash Офигеть! Где на такую ораву стоянку\sdash то найти?\mdash Адмирал не рассчитывал вообще кого\sdash либо встретить на~первой части маршрута, до Линдозера.

\diagdash Ну где, где, нигде! Только проведя широкомасштабные шанцевые работы!\mdash отозвался Киря.

\diagdash Какие работы?\mdash спросил Руслан.

\diagdash Шанцевые.\mdash пояснил Адмирал.\mdash Шанцевый инструмент это лопата, топор, кирка, лом и так далее.

\diagdash А-а-а!

\diagdash Вот только лома нам еще в байдарке не хватает!\mdash Серёга активно грёб веслом и его экипаж обошёл адмиральский.\mdash Что будем с ними делать?

\diagdash Да ничего, надо их обойти, ибо нефиг! А ну\sdash ка, налегаем на вёсла, парни! У нас каркасные байдарки, а у них надувное гуано! Мы их как нефиг делать сделаем!\mdash отдал команду Адмирал.

\diagdash Неправда! Вон у них один чел на пластиковом каяке идёт!\mdash заметил Серёга.

\diagdash Это инструктор, сто процентов. Остальные лошары, смотри как гребут! И у них катамараны, они на озере просто тормоза тормознутые. Погнали! Паш, ставь парус, ветер попутный!\mdash Адмирал воинствовал.

Паша развернул парус, приладив его на лопасть весла и обмотав шкоты по спирали вдоль веретена весла. Ветер был почти попутным и стремительно тащил их к северной оконечности озера, где был исток Нурмиса. Как они и хотели, они стремительно обошли конкурентов и, можно сказать, с~разгону ворвались в Нурмис. Ветер почти сразу стих, а они по инерции вкатились в реку.

Пейзажи вокруг перестали радовать команду, началось петляние реки, лиственный характер заросших, низких берегов, полное отсутствие стояночных мест. Словом, почти то же самое, что было днём ранее на речке Кулапдеги. Поворотики порою достигали 160\thinspace\nobreakdash---\thinspace 170 градусов по курсу, столь сильным было петляние реки.

\diagdash Шурик, стояночных мест ваще нет!\mdash Киря упахивался веслом, выгоняя вчерашнее.

\diagdash А я о чём? Я читал, тут тема такая\mdash Нурмис надо проходить до конца, до Линдозера, тут нет нормальных стоянок!

Спустя часа полтора петляний по руслу они решили, что уже пора бы вылезти и размяться, как вдруг из\sdash за очередного поворота стал слышен звук не то переката, не то чего. Адмирал понял, что это автомобильный мост, первое препятствие на этой реке. Он заорал команде:

\diagdash Причаливаем к правому берегу, осмотр!

Оставив экипажи держать байдарки, два капитана, Шурик и Киря, потопали на мост. Течение было очень сильным, рокот воды стоял приличный, а вот просвет между водой и мостом был еле\sdash еле на высоту байдарки. Пришёл и~Паша поглядеть что да как:

\diagdash Ёпрст!

\diagdash И не говори!\mdash Адмирал с Замполитом прикурили и~пошли посмотреть на мост с другой стороны.

\diagdash Ну чё, проводим на чалке, однозначно.\mdash заключил Адмирал.\mdash Что скажешь?

Киря спустился у моста с одной стороны, а Паша с~другой и заглянули под него:

\diagdash Ну давай попробуем! Очень неохота разгружаться.\mdash неуверенно заключили они.

\diagdash Пошли! Тут всё однозначно! Проводим на чалке, на выходе принимающий подхватывает и дело в шляпе!\mdash Адмирал почти бегом пошёл обратно, к~Серёге с~Русланом, которые остались у байдарок.

Те маялись пока что бездельем, ожидая возвращения разведки. Адмирал выдал разнарядку:

\diagdash Так! По коням! Будем проводить байды под мостом! Руслан, иди берегом к Пашке, ловите там перед мостом меня. Серёга, ты берегом аналогично. Киря щас вернётся и~пройдет на байде. Погнали!\mdash Шурик сел в байду и плавно отчалил. 

Главное было не перескочить небольшой заливчик перед мостом, а то потом бы течение уже утянуло бы под мост. С~этим он~справился\mdash страхующие подхватили его в~заливчике. Адмирал вылез и вытащил чалку:

\diagdash Так! Дальше делаем финт ушами\mdash байды поворачиваем кормой вперёд, потому что чалка привязана ближе к носу\mdash так будет устойчивее её сплавлять!\mdash сказал Адмирал и моментально проделал это со своим кораблём.\mdash Дальше предельно аккуратно!!! Парни!!! Начали!!!\mdash Адмирал держал байду за чалку и начал плавно заводить её в поток, а Паша побежал принимать её с другой стороны моста, Руслан тоже пошёл туда страховать.

 





\begin{center}
	\psvectorian[scale=0.4]{88} % Красивый вензелёк :)
\end{center}
