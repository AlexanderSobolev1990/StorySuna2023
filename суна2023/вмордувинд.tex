\chapter{Вмордувинд}
%\corner{64}
\vepsianrose

Как обычно в последний день всеми овладела леность. Осознание того, что пройти осталось совсем немного\mdash расхолаживало. Адмирал, проснувшись, хотел подольше поваляться, но не вышло\mdash погода выдалась на удивление жаркой и солнечной. Яркое утреннее солнце нагрело их с~замполитом палатку, стало душно. Пришлось вылезать, раскачиваться и варить кофе, который пришёлся как нельзя кстати. День разгорался жаркий и безоблачный\mdash самая что ни на есть днёвочная погодка, а им сегодня сниматься с маршрута. Адмирал чуть не скрежетал зубами, пока варил утреннюю кашу. 

Команда тоже вяленько раскачивалась, потихоньку собираясь у костра:

\diagdash Опять овсянка?

\diagdash Завтрак чемпионов, ну!\mdash парировал Адмирал.

\diagdash Нам десяточку сегодня?\mdash спросил Замполит.

\diagdash Угу!\mdash Адмирал попробовал кашу и посолил.

\diagdash Фигня! Даже протеином можно не заряжаться.

\diagdash Фигня\sdash то фигня, но ветер будет, походу, встречный. 

\diagdash О, точно. Тогда протеином всё\sdash таки заряжаться.\mdash Замполит пошёл искать свой бочонок в красной герме$\ldots$

$\ldots$На воду ребята встали поздно, около полудня. Как и предсказывал Адмирал, поднялся сильный встречный ветер, вмордувинд. Идти по озеру стало настоящей пыткой. Сверху нещадно палило солнце, они все скинули штормовки, футболки и тельняшки, постоянно смачивали панамы водой\mdash пекло стояло совсем не карельское:

\diagdash Роман Менделич, ну вот вчера бы так!\mdash лопатил Замполит веслом.  

\diagdash Не договорились опять?\mdash Серёга вытащил ноги на нос байды и грёб полулёжа.

\diagdash Вообще никак, жара!

Адмирал снял тельняшку и повесил её себе на спину, положил GPS\sdash прибор на колени и правил по нему, чтобы не вилять по курсу на большой воде, когда нет особых ориентиров:

\diagdash Чёртов ветер! Щас бы двигатель.

\diagdash Может Гирвас в другой стороне? И нам под ветер надо?\mdash Замполит тяжело грёб.

\diagdash Да если бы$\ldots$

\vspace{1em}
$\ldots$ В борьбе с ветром прошёл где\sdash то час. Маленькая эскадра дошла до середины Викшозера. Параллельно Адмирал старался запримечивать стояночные места на будущее, но их, как назло, практически не было:

\diagdash Вот если бы не встали позавчера на том мысу и попёрлись бы в озеро дальше\mdash вот тогда бы точно влипли.

\diagdash Да-а-а, стояночек что\sdash то нема.\mdash Серёга всё грёб полулёжа.

Они прошли средних размеров остров но правому борту:

\diagdash А на острове?\mdash поинтересовался Руслан.

\diagdash Да тоже, гляди, ни пляжиков, ни сходов к воде.\mdash грёб Адмирал.

\diagdash Вода ещё высокая, как мы поняли,\mdash вспомнил Паша,\mdash Вот пляжей и не видно.

\diagdash Может$\ldots$

\diagdash Долго нам ещё?\mdash поинтересовалась команда, поскольку ветер стихать даже и не думал.

Адмирал сверился с GPS:

\diagdash Половину только прошли.

\diagdash Да твою ж$\ldots$

\vspace{1em}
$\ldots$Спереди чуть левее по курсу начала нарастать гора. Сверившись с картой, Адмирал понял, что перед ней должен быть поворот реки и им туда\mdash гора должна скрыть их от ветра:

\diagdash Вперёд, парни! У этой горы поворачиваем и там на мысу отдохнём, я выжат как лимон, жара ещё.

\diagdash Правь давай$\ldots$

Спустя ещё минут 15 они высадились у истока Суны из озера на левом берегу у хорошей добротной стоянки с баней. Замполит тут же полез в вещмешок раздавать всем батончики спортпита подзарядиться:

\diagdash Вот самое нелюбимое в Карелии после дождя\mdash это лопатить по озёрам. Особенно против ветра!

\diagdash Полностью согласны!\mdash вздохнули все, запивая батончики чаем из термосов. 


\begin{center}
	\psvectorian[scale=0.4]{88} % Красивый вензелёк :)
\end{center}
