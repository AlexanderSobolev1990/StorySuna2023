\chapter{Робинзоны}
%\corner{64}
\vepsianrose

В предыдущие два дня погода была как по заказу\mdash яркое солнце, белые кучевые облачка, красота. Но на днёвку парням погода выдалась так себе, не очень. Если с утра ещё было ничего и они нормально приготовили завтрак, но~потом налетел шквал, с юга начала сгущаться дождевая облачность, краски дня пожухли, всё стало серым, мрачным.

После традиционной каши чемпионов с утра все развалились на стульчиках под тентом у стола, состояние было ленивым. Островные Робинзоны выспались, отдохнули. Киря внезапно развёл деятельность:

\diagdash Шурик, мне нужная вторая чалка!

\diagdash На кой она тебе?

\diagdash Ну мы завтра же пороги берём, а у меня только носовая чалка, кормовой нету!

\diagdash Кирь, у меня тоже только носовая чалка, а кормовой нету и не было отродяся.

\diagdash А вот когда мы ходили по Керети$\ldots$

\diagdash Зачем она нужна, кормовая?

\diagdash Да мало ли что!\mdash Замполит что\sdash то начал разводить нервы.\mdash Смотри, это может пойти на чалку?\mdash он достал моток белой синтетической ленты.

\diagdash Ё-моё! Зачем ты это тащишь с собой?\mdash Адмирал всегда ратовал за уменьшение количества вещей, без которых можно обойтись.

\diagdash Ну, ты говорил, что вещи вязать будем к каркасу$\ldots$

Адмирал вспомнил, что действительно были такие разговоры:

\diagdash Да, говорил, но судя по тому, как мы вчера всё прошли, по\sdash моему это лишнее. Кстати, из этой ленты можно сплести чалку, раз ты так хочешь.

\diagdash Сплести?!\mdash переспросили Серёга с Кирей.

\diagdash Ну да, а что? Как косичку, из трёх лент. Тащи сюда свой моток!

Они нашли старый ржавый гвоздь, вбитый в сосну поблизости\mdash остатки чьей\sdash то старой стоянки\mdash и, вколотив его топором в сосну повыше, привязали три ленты:

\diagdash Так, ну теперь плетём.\mdash Адмирал сел на складной стульчик и начал пробовать.\mdash Руслан, подсоби!

Тот тоже взял раскладной стульчик, сел рядом и~они в~четыре руки начали плести\mdash сначала они просто переплетали ленты, поджимая получающуюся косичку, но потом поняли, что по мере плетения, <<хвост>> этой косы в~бухте приходится разматывать в обратную сторону, что было жутко неудобно. Поэтому они решили сделать три <<катушки>> с лентой, намотав её на палочки, чтобы получились такие большие шпули. После того, как три шпули были готовы, дело пошло гораздо веселее, ничего не запутывалось:

\diagdash Перехватывай.\mdash Адмирал передавал одну шпулю Руслану, перекрещивая ленты в плетении.\mdash И давай сюда обратно!

\diagdash Ручной ткацкий станок!\mdash Руслану было по приколу поучаствовать в затее.

\diagdash Пацаны, у вас клёво получается!\mdash Замполит завороженно стоял рядом и наблюдал как парни плетут уже третий метр чалки. Руки их так и сверкали, перекрещиваясь и меняясь перехватом шпуль.\mdash Быстро у вас получается!

\diagdash Хочешь сам?

\diagdash Не-не-не, я смотрю вы такие профессионалы, что куда уже мне!\mdash отшутился Замполит.

На плетение чалки ушло часа два, а то и больше\mdash когда ребята закончили, погодка нахмурилась окончательно\mdash и~если с утра было достаточно тепло и Адмирал расхаживал с голым торсом, то сейчас ему пришлось утеплиться в~штормовку.

\diagdash Так, вяжи узел на конце, давай попробуем канат на~прочность!\mdash Адмирал докончил плести, встал, размялся и растянул их творение.

\diagdash Да вроде не рвётся!\mdash заключил Киря, наблюдая за~тем, как Адмирал оттягивает собой новоиспечённый канат.

\diagdash Пользуйся!\mdash Адмирал сбухтил канат и кинул Замполиту\mdash тот пошёл вязать его к байде.

Все, немного развлекшись с плетением, вновь расселись под тентом:

\diagdash А что у нас со связью всё таки?\mdash Паша сидел у~стола и пил чаёк с конфетами.

\diagdash О! А надо знаете что сделать? Надо взять ту же ленту, из которой чалку плели, обвязать какой\sdash нить камешек, перекинуть это дело через ветку сосны, да хотя бы вон той,\mdash Адмирал показал на росшую рядом с их поляной высокую красивую сосну с раскидистыми ветками,\mdash и поднять туда наверх телефон, скажем, в обрезке пластиковой бутылки!

\diagdash А что, идея! Чей телефон будем поднимать? Чур,~не~мой!\mdash ответил Паша.

\diagdash Ну чей, чей. Знамо, чей\mdash замполитовский, больше~то~чей?\mdash заключил Адмирал.

\diagdash Чой та?\mdash отозвался тот.

\diagdash Ну как <<чой та>>? Такая вот доля твоя!\mdash иронизировал Адмирал.\mdash Да лан, расслабься! На самом деле только у тебя в Нёлгомозере ловило, так что выбор очевиден.

\diagdash Блин, и не поспоришь. Ну давай!

Адмирал обвязал камушек лентой и перебросил через ветку. Потом связал ленту в кольцо и подвязал к ней обрезок пластиковой бутылки, которой пришлось пожертвовать. Замполит скрепя сердце подошёл и положил туда мобильник:

\diagdash Поехали!

Телефон медленно пополз вверх. Команда смотрела за~всем этим действом из\sdash под тента.

\diagdash Шурик, хар{\'о}ш!\mdash Киря переживал.

\diagdash Не очкуй, аллес унтер контролле!\mdash он поднял телефон вверх до предела, они подождали пару минут, спустили обратно.

\diagdash Ну и нифига!\mdash раздосадованно сказал Замполит.

\diagdash Как нифига???

\diagdash Ну так, не грузится страница, <<палочек>> связи нет!

\diagdash Да ёпрст!

Они промучились ещё минут пятнадцать, ничего у~них~не~вышло:

\diagdash Надо бы там, наверху когда телефон, нажать кнопку <<обновить страницу>>!\mdash догнал Замполит.\mdash Без этого ничего не получится.

\diagdash Умный, как сто китайцев! Если бы можно было туда залезть по голому стволу, то и верёвка эта нафиг не нужна!

\diagdash О, а ещё можно сплести верёвочную лестницу$\ldots$

\diagdash Я с вас угораю!\mdash ржал под тентом Паша.

\diagdash Ладно, версия с сосной провалилась, ещё варианты?\mdash вопрошал Адмирал.

\diagdash Хм-м-м! Ну, вчера была версия, что надо идти на~мыс пытаться там ловить связь, наскока я помню.\mdash сказал Серёга.\mdash Пошли, Кирь, сходим наверно?

\diagdash Походу да, придётся$\ldots$

Времени было уже к часу дня, когда ребята собрались в поход по острову. Погода нахмурилась и закапал редкий дождик. Остальная команда сидела, блаженно развалившись на стульчиках под тентом:

\diagdash Так, Робинзоны, давайте! Я в вас верю! Без описания маршрута не возвращайтесь!\mdash напутствовал Адмирал.

\diagdash Приободрил так приободрил!\mdash те взяли с собой топорик на всякий случай и пошли. Паша и Руслан с~Адмиралом остались у костра и продолжили чаёвничать:

\diagdash Что будем делать без описания маршрута?\mdash спросил Руслан.

\diagdash Ну как что, идти, а что ещё? Не сниматься же позорно? Это всего\sdash навсего вторая категория, не пятая.\mdash сказал Адмирал.

\diagdash Тем более мы всегда так ходим\mdash этот,\mdash Пашка махнул кружкой в сторону Адмирала,\mdash прочитает чё нить про маршрут и нихрена никому ниче не расскажет, а мы потом вляпываемся!

\diagdash Да лан тебе, разошёлся прям! Когда такое было?\mdash оправдывался Адмирал.\mdash Ну, может пару раз и то некритично!\mdash покопавшись в воспоминаниях, ответил он.

\diagdash Ну, на Песи с обносами$\ldots$\mdash Пашка вспомнил их поход пятилетней давности.

\diagdash Ну и обнесли, не умерли?\mdash напомнил Адмирал.

\diagdash Это да$\ldots$\mdash согласился тот.\mdash Потом ещё чёт было такое, не помню уже.

\diagdash Всё некритично\mdash все живы, здоровы, снарягу не~потеряли, сами не убились, значит всё в порядке!\mdash заключил Адмирал и прихлебнул.

\diagdash Так\sdash то оно так. Было. А вот завтра как будет\mdash хрен его знает!\mdash тоже прихлебнул Паша.\mdash C ним,\mdash опять показав на Адмирала,\mdash не пропадёшь, да\sdash а\sdash а.\mdash сказал он, немного стращая Руслана.

\diagdash Киря только что\sdash то меня расстраивает.\mdash поделился своими опасениями с экипажем Адмирал.\mdash Не захотел порог идти вчера, руками тащил. А что завтра будет? Завтра настоящие, всамделишние пороги пойдут.

\diagdash Тогда вчера что было? Не всамделишние?\mdash осведомился Руслан.

\diagdash То была разминка, они на топокарте как пороги\sdash то даже не обозначены$\ldots$

Робинзоны вернулись через час, не меньше:

\diagdash Так, ну что\sdash то получилось загрузить!!!\mdash обрадовал команду Замполит!

\diagdash Класс! Докуда дошли?\mdash спросил Адмирал.

\diagdash Да практически до той стоянки на мысу, которую видели вчера. У них лодка под мотором.\mdash ответил Серёга.

\diagdash Ух! Стало быть они сюда против течения дошли?\mdash предположил Адмирал.

\diagdash Почему? Может и с верховьев Суны?\mdash размышлял вслух Серёга.

\diagdash Не-е-е, с верховьев навряд ли, там же воды меньше, пороги жёстче. А они на моторке. Это с деревни, сто процентов.\mdash предложил народу наиболее убедительный вариант Паша.

\diagdash Да ну не факт! Подними в порогах движок и всё! Мы же не знаем что там выше по Суне.\mdash пошли толки.

\diagdash Забейте, главное теперь у нас есть описание порогов. Давайте читать!\mdash Адмирал приободрился.

\diagdash Так насухо как\sdash то$\ldots$\mdash предложил Паша.

\diagdash А у меня винишко есть$\ldots$\mdash сказал робко Руслан.\mdash Спецом на днёвочку.

\diagdash Ё-моё, ты в стекле тащил?\mdash Адмирал немного негодовал.

\diagdash Ага!$\ldots$\mdash сознался тот.

\diagdash Тащи! Уменьшим вес байды перед порогами!\mdash однозначно заключили все.

Погодка, тем временем, решила испортить им днёвку\mdash зарядил ливень, да такой, что в ослабший за ночь тент над столом стала набираться вода.

\diagdash Шурик, надо бы слить воду\sdash то$\ldots$\mdash Замполит посмотрел вверх на тент.

\diagdash Надо! Надо под дождь идти. Иди?\mdash Паша разливал всем желающим красненького.

\diagdash Ы-ы-ы!

\diagdash Пофиг, потом!

\diagdash Ы-ы-ы! Разливай, да подставляй, не зевай!

\diagdash 















\begin{center}
	\psvectorian[scale=0.4]{88} % Красивый вензелёк :)
\end{center}
