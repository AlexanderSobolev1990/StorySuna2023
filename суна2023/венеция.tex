\chapter{Венеция}
%\corner{64}
\vepsianrose

Погрузка заняла у них минут 40. Уклон местности выматывал с непривычки. Шурик нацепил на ноги гидроноски на плотной подошве и они с Замполитом спустили свои корабли на воду. Торжественный момент! Адмирал ждал его три года. Три года! Три года они не~ходили на реку! У него чуть не~кружилась голова, хотя махнули лишь на донышке. Он~снова на воде! Вот она, мокрая!!! Чистая!!! Карельская!!! Он~мечтал об этом очень, очень давно. И это свершалось.

В гидроносках шлёпать было гораздо удобнее, чем в~чём бы то ни было ещё\mdash можно было сразу с берега входить в воду и так же легко выходить, как босиком, но не боясь пораниться о камни, а толстая гидроткань защищала от~холода, напитываясь водой, которой передавалось тепло тела. Шурик, стоя в воде и ловя кирину байду, бросил Замполиту:

\mdash Кирь, привяжи за чалку\footnote{верёвка, трос для причаливания}\sdash то.

\mdash У меня нету$\ldots$

\mdash Да хар\'{о}ш! А чем чалиться собрался?!

Киря пошёл искать чалку в вещах, а Шурик, стоя по~колено в бодрящей водичке, стал принимать от Руслана и Паши вещмешки и раскладывать их по~грузовому отсеку. Делать это стало труднее, чем обычно, потому что в~этот поход друзья прикупили байдарочные фартуки\mdash они уменьшили размеры погрузочного проёма. Шурик, скрежетая зубами, раз за~разом пробовал по\sdash иному, компактнее, впихивать снарягу. Лезло плохо, но Шурик знал по опыту\mdash всего 2 дня и продуктов подуменьшится и всё будет влезать гораздо лучше.

\mdash Давай их снимем, а наденем только перед порогами?\mdash предложил подошедший Замполит. 

\mdash Умный, как сто китайцев! А куда сам фартук денем? 

\mdash На корму?

\mdash У меня там спасы, вот их точно только на порогах наденем. Не, надо крепить фартуки, скоро дождь, нальёт по кильсон в трюмы!

\mdash Ну, пожалуй$\ldots$

Серёга и Киря тоже, чертыхаясь, распихали всю снарягу под борта и по отсекам. У них была двушка и барахла влезло меньше. Начался дождь, повисла сплошная облачность, солнце не пробивалось. Шурик, поморщившись, вытащил из кармана брезентовой штормовки дождевик\mdash он всегда лежал у него в левом кармане штормовки, а в правом лежала кружка для отчерпывания воды и сугрева\mdash такова была привычка сплавщика. Команда тоже приоделась в~дождевики, кто во что.

\mdash Готовы???

\mdash ДА!

\mdash Хрена два! Пойду проверю что ничего не забыли!!!\mdash Шурик пошёл наверх по склону к поляне, где собирали байды, чтобы самолично убедиться в отсутствии оставленных вещей. Спустя пару минут, ничего не найдя, он спускался к~озеру, аккуратно ступая к мокрых гидроносках.


Про каналы

\begin{center}
	\psvectorian[scale=0.4]{88} % Красивый вензелёк :)
\end{center}
