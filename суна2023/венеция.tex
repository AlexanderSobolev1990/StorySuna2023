\chapter{Венеция}
%\corner{64}
\vepsianrose

Погрузка заняла у них минут 40. Уклон местности выматывал с непривычки. Шурик нацепил на ноги гидроноски на плотной подошве и они с Замполитом спустили свои корабли на воду. Торжественный момент! Адмирал ждал его три года. Три года! Три года они не~ходили на реку! У него чуть не~кружилась голова, хотя махнули лишь на донышке. Он~снова на воде! Вот она, мокрая!!! Чистая!!! Карельская!!! Он~мечтал об этом очень, очень давно. И это свершалось\mdash сейчас, прямо на глазах.

В гидроносках шлёпать было гораздо удобнее, чем в~чём бы то ни было ещё\mdash можно было сразу с берега входить в воду и так же легко выходить, как босиком, но~не~боясь пораниться о камни, а толстая гидроткань защищала от~холода, напитываясь водой, которой передавалось тепло человеческого тела. Адмирал, стоя в воде и ловя кирину байду, бросил Замполиту:

\diagdash Кирь, привяжи за чалку\sdash то\footnote{верёвка, трос для причаливания}.

\diagdash У меня нету$\ldots$

\diagdash Да хар\'{о}ш! А чем чалиться собрался?!

Киря пошёл искать чалку в вещах, а Шурик, стоя по~колено в бодрящей водичке, стал принимать от Руслана и Паши вещмешки и раскладывать их по~грузовому отсеку. Делать это стало труднее, чем обычно, потому что в~этот поход друзья прикупили байдарочные фартуки\mdash они уменьшили размеры погрузочного проёма. Шурик, скрежетая зубами, раз за~разом пробовал по\sdash иному, компактнее впихивать снарягу. Лезло плохо, но Шурик знал по опыту\mdash всего 2 дня и продуктов подуменьшится и всё будет влезать гораздо лучше.

\diagdash Давай их снимем, а наденем только перед порогами?\mdash предложил подошедший Замполит. 

\diagdash Умный, как сто китайцев! А куда сам фартук денем? 

\diagdash На корму?

\diagdash У меня там спасжилеты, вот их точно только на~порогах наденем. Не, надо крепить фартуки, а то дождь собирается\mdash нальёт по~кильсон в трюмы!

\diagdash Ну, пожалуй$\ldots$

Серёга и Киря тоже, чертыхаясь, распихали всю снарягу под борта и по отсекам. У них была двушка и~барахла влезло меньше. Начался дождь, повисла сплошная облачность, солнце не пробивалось. Шурик, поморщившись, вытащил из кармана брезентовой штормовки дождевик\mdash он всегда лежал у него в левом кармане штормовки, а~в~правом лежала кружка для отчерпывания воды и сугрева\mdash такова была привычка сплавщика. Команда тоже приоделась в~дождевики, кто во что.

\diagdash Готовы???

\diagdash ДА!

\diagdash Хрена два! Пойду проверю, что ничего не забыли!!!\mdash Шурик пошёл наверх по склону к поляне, где собирали байды, чтобы самолично убедиться в отсутствии оставленных вещей. Спустя пару минут, ничего не найдя, он спускался к~озеру, аккуратно ступая в мокрых гидроносках.

\diagdash Ну, отчаливаем! Серёга, Паш, отвязывайте чалки!\mdash сказал Адмирал и они с Замполитом заняли свои капитанские места\mdash в корме каждый своей байдарки. 

Поначалу только, неопытному байдарочнику, кажется это несправедливым\mdash мол, как же так, капитан и сидит на корме, где\sdash то на~галёрке. А все красоты реки, первым всё усматривает сидящий впереди матрос. А ему, капитану, как бы всё позже доходит. Лишь спустя несколько речных переходов приходит понимание, что нет, всё правильно\mdash капитану с~кормы виднее всё на свете. Он лучше видит с кормы курс корабля, лучше усматривает куда сносит байдарку, как надо грести, чтобы обойти препятствия, как маневрировать. Да,~спины экипажа, конечно, закрывают вид прямо, но~это у Шурика компенсировалось высоким ростом и~тем, что когда ходили по спокойным рекам, он, плевав на все правила, садился не~внутрь байдарки, а верхом на кормовой шпангоут. Таким образом, сидел он очень высоко, далеко видел, а насчёт устойчивости не~волновался\mdash матросы и вес груза в трюмах обеспечивали достаточный запас от~переворачиваемости. Сейчас же такой фокус не прошёл\mdash сверху на фальшборта надели фартук и~сидеть на кормовом шпангоуте стало невозможно\mdash тогда бы повредился фартук. Шурику это было непривычно\mdash он за все свои предыдущие сплавы байдарочные свыкся с~мыслью, что всегда высоко сидит, далеко глядит, а~тут, как в первых сплавах, придётся сидеть низко\mdash хуже обзор воды. А~на~маршруте\mdash пороги. Ничего, подумал Адмирал, в этот раз с ним был весь его предыдущий опыт, который, как известно, не пропьёшь. Он~с~силой оттолкнулся от~каменистого берега\mdash Cплав начался!

На небесах, тем временем, как кран открыли\mdash лило просто ужасно. У Руслана, который опрометчиво подумал, что, быть может, дождик скоро закончится, лёгкая курточка уже через 5 минут промокла насквозь, а кепку он и~вовсе никакую не взял\mdash плыл с непокрытой головой. Справа по~борту прошли торчащие изо дна озера жерди, на которых были растянуты то ли сети, то ли что\mdash похоже, тут~разводили рыбу. На берегу виднелся домик этого небольшого рыбохозяйства.

\diagdash Кирь, с Роман Менделичем, походу, договориться не~удалось, да?\mdash протирая ладонью капли дождя с носа, сказал Шурик.

\diagdash Сам видишь, со связью тут швах! Извиняй!\mdash отозвался тот с соседней байды. Вид у~них в~дождевиках был так себе, почти комический. <<Пофиг, зато сухо>>,\mdash думал Шурик. Руслан, отчего\sdash то не взявший дождевик, уже неслабо так промок.

У западного конца Вендюрского озера Шурик стал поглядывать в навигатор и прикидывать\mdash насколько велики шансы, что команда линчует его, если канал, по которому он хотел сократить путь примерно в полтора раза, попав в~речку Кулапдеги не напрямую, а через Сяргозеро, окажется непроходим для байдарок. Минуты раздумья тянулись за~греблей томительно долго\mdash нетренированные мышцы рук, получившие убойный натиск от весла, ныли и не давали сосредоточиться. Наконец, Шурик решил\mdash была\sdash не была\mdash попробует ломиться через канал\mdash интересно было глянуть что там.

К каналу подошли достаточно скоро и особо не искали его начало\mdash GPS был на контроле у Адмирала. Народ выразил ликование от такого сооружения, представшего их~взору. Канал был шириной примерно метров 8, а вот глубина оказалась крайне мала\mdash не более 30 сантиметров\mdash осадка байдарок еле\sdash еле позволяла проходить вперёд. Стенки канала были сделаны из брёвен, из воды показывалось около пяти венцов, то есть вода была достаточно низкой, понял Шурик. Состояние брёвен канала Адмирал оценил максимум ещё лет на 5\mdash 7, неизвестно когда их обновляли в~последний~раз$\ldots$ да и будут ли делать это снова\mdash кому это нужно?

\diagdash ШУРИК! Там мель!\mdash увидал Паша. 

\diagdash Полундра, покинуть байду!\mdash скомандовал Адмирал и далее по каналу они уже пошли шлёпая гидроносками по дну. Сверху было мокро от дождя, снизу от канала. <<Шикарное начало>>,\mdash подумал Шурик.

\diagdash И мне вылазить?\mdash спросил Руслан. 

\diagdash Сиди, вроде осадка позволяет пока что!\mdash отозвался Адмирал и они с Пашкой потихоньку стали проводить байду на чалке, стремясь не напороться на коряги и камни\mdash Паша шёл спереди, подтаскивая байду за чалку, а Шурик плёлся сзади, контролируя процесс.

Вокруг канала были сплошные заросли, выглядела эта вся картина несколько сюрреалистично\mdash вдруг тут, в глуши, по какой\sdash то причине бах и появился канал. Зачем появился, кто его сделал, с какой целью? Рыбу разводили, скорее всего, в советское время тут в б\'{о}льших масштабах, а может и нет\mdash кто теперь расскажет? Ребята потихоньку продвигались вперёд, то бредя по камням, то усаживаясь в байду, когда становилось чуть глубже. Ближе к концу канала он стал совсем мелким и им снова пришлось вылезти. Пеший переход по воде немного вымотал, надо было передохнуть. Через минут 15 впереди показался конец канала, весь заросший тростником\mdash команда вышла в Сяргозеро. Решили устроить небольшой привал\mdash канал кончился, дождь выматывал, мышцы с непривычки ныли, а Руслан промок до нитки.

\diagdash А ну, пацаны! Впереди стояночное место, что ли?\mdash изрёк Шурик.

\diagdash Да вроде бы.\mdash отозвался Паша.

\diagdash Правим туда, перекур. 

\diagdash Ура, ура, ура! Перекур!\mdash воскликнул Замполит, тоже порядком уже нагрёбшийся.

\diagdash Кирь, тебе привет от пульмонолога, какой перекур?\mdash язвил Шурик.

\diagdash Шурик, иди-ка ты?

\diagdash Ы-ы-ы! Сигарку будешь?\mdash не унимался тот.

\diagdash Вы поглядите, вы поглядите! Совращают!

Адмирал с Замполитом задымили, едва оказавшись на берегу. Серёга и Паша пошли осматривать стоянку, расположившуюся на левом берегу Сяргозера сразу после конца канала. Место было неоднозначным, но вполне себе подходящим под стоянку\mdash мусора почти не было, костровище имелось, полянка была хороша. 

\diagdash Руслан, переодевайся, ты же совсем промок!\mdash Адмиралу вдруг стало даже жалко своего новоиспечённого матроса. 

\diagdash Да, надо$\ldots$ Чёрт, мою герму заложили мешками, помоги?

\diagdash Давай, держу байду, доставай.\mdash Шурик придержал байду, а Руслан вытащил шмот и переоделся, отжав куртку. 

Шурик вернулся на берег\mdash осмотрел костровище, походил по поляне. Вокруг всё было, естественно, мокрым от дождя, который что\sdash то не собирался прекращаться. Теоретически, можно было бы остаться и~тут, за~километражом Шурик решил абсолютно не~гнаться в~этом сплаве\mdash хватит, набегались за прошлые походы, он решил почти что пустить это на самотёк, поскольку был уверен\mdash вторая часть маршрута по быстрой Суне с порогами будет скоростной$\ldots$ только если, конечно, не~придётся ремонтировать байдарки. От дум Адмирала отвлёк Замполит:

\diagdash Матрос насквозь?

\diagdash Ага$\ldots$

\diagdash Бывает. Он первый раз в походе?

\diagdash Вроде как.

\diagdash Ладно, посмотрим как дальше пойдёт. Пора дальше\mdash греться веслом, чёт я замерзаю\mdash Замполит затушил окурок.

\diagdash Бригада! По коням!\mdash скомандовал Адмирал и они стали отчаливать под дождём.

С двухкилометровым переходом через маленькое Сяргозеро Шурик справился с трудом\mdash после передышки на~стоянке вообще не греблось, а встречный ветер выматывал. Но~он~не~имел права подать виду\mdash ведь он Адмирал сплава\mdash и грёб наравне со всеми. Киря упахивался веслом, предварительно закинувшись спортивным питанием, надеясь, как всегда, за поход немного подкачаться. Думы вновь одолели Адмирала\mdash что если второй канал, по~которому он рассчитывал попасть из~Сяргозера в~реку Кулапдеги, зарос или обмелел или и то и то вместе взятое. Перспектива возвращаться назад в~Вендюрское озеро и~фактически начинать маршрут заново, с истока Кулапдеги, не то что не радовала, а просто уничтожала\mdash команда, во\sdash первых, такое не простит, а~во\sdash вторых это страшная и~бесполезная потеря времени и~сил. Шурик решил положиться на удачу и в случае чего устроить волок\mdash перешеек между озером и рекой там всего 160\mdash 170 метров, он уже успел промерить по карте на перекуре. Но всё же перспективы он оценил верно\mdash если отметки урезов воды на карте верны, то следующее озеро, как расположенное ниже по течению, должно было полноводнее, и воды в канале будет достаточно.

\diagdash Тащ Адмирал, разрешите обратиться?\mdash деланно язвил Замполит.\mdash А куда мы таки прёмся? Там озеро закончилось!

Шурик поглядел на навигатор:

\diagdash Левее берём, нас относит боковым ветром. Канал должен быть левее!

\diagdash А если$\ldots$

\diagdash Никаких если! Канал левее!\mdash Шурик что\sdash то мгновенно вскипел, отягощенный думами о канале и волоке.

\diagdash О, походу вон он!\mdash Серёга, вперёдсмотрящий Кири, неопределённо показывал куда\sdash то вперёд.

Через 5 минут ребята без проблем прошли по~совсем короткому непримечательному каналу и очутились в~речке Кулапдеги, уходящей влево дальше к Сяпчозеру. У~Адмирала отлегло\mdash первые два канала они миновали почти идеально. Дальше начался в~буквальном смысле байдарочный слалом по петляющей среди сплошняковых болотистых лиственных зарослей речушке. По~прямой выходило километра три по карте, но~бесконечные повороты удлиняли путь чуть не в три раза:

\diagdash Шурик, это похоже на петляние Песи перед Хвойной!\mdash Замполит вдруг вспомнил былое.

\diagdash Ага, только болотина вокруг глянь какая! Ни~причалить, ни чего!

\diagdash Зато поворотики клёвые!\mdash отозвался Серёга.\mdash И~нет ветра, как на озере. Ищи плюсы!

\diagdash Эт да-а-а.\mdash согласился Шурик. Ему речка эта, конечно, не понравилась. На стоянку в первый день при планировании маршрута  он решил вставать уже на~Сяпчозере, надо было просто, что называется, перетерпеть дождь и петляющую речку с неприветливыми болотистыми берегами. 

Адмирал погрузился в свои мысли. От~его взора на~карте не спрятался небольшой мыс по~правому берегу Сяпчозера. Он был расположен между двумя областями каменистого берега, обозначенного на старой топокарте. В~километре на северо\sdash восток оттуда располагался старый геодезический пункт на горе. Местность эта и этот мыс, расположенный между устьем Кулапдеги и истоком Сяпчи, понравились Шурику сразу. Сейчас он грёб, воплощая в~реальность своё перемещение в~пространстве к~стоянке. Он давно вывел для себя правило любого похода\mdash даже оказавшись на~маршруте в~первый раз, ты обязан проходить его минимум в пятый раз! Первый\mdash мысленно по~карте, топографической и спутниковой, второй раз\mdash читая описание маршрута в туристической литературе и отчётах в интернете, третий\mdash в~разговорах и~обсуждениях с~товарищами, четвёртый\mdash мысленнно соединяя, скрепляя воедино в своем сознании предыдущие три раза и,~наконец, в~пятый раз уже в~реальности, на~местности. Район мыса между двух каменистых отмелей подходил под первую стоянку идеально\mdash они прошли порядка 14~километров в~первый день похода\mdash более чем достаточно. Адмирал также очень рассчитывал, что~относительная непопулярность этой части их маршрута оставит эту стоянку свободной\mdash первая стоянка должна подвести черту под началом их путешествия, преобразить вчерашних горожан в речных волков, очистить и перегрузить сознание всех и каждого\mdash он знал по опыту, что так всегда почему\sdash то случается, хотят сами сплавщики этого или нет. 

Небо наконец расчистилось, дождь кончился, как и~порядком надоевшая всем Кулапдеги, ребята оказались в~Сяпчозере. Переменчивая карельская погода, показав свой нрав, порадовала команду. Ласковое вечернее солнышко мгновенно подняло всем настроение, озёрный простор всем пришёлся по нраву после узкой петляющей речушки. Паша приободрился:

\diagdash Кирь, а ты говорил, что у тебя подзорная труба есть?

\diagdash Есть, ага! Специально взял стоянки искать по~озёрам.\mdash отозвался тот.  

\diagdash Щас самое время!

Они прошли начало каменистой отмели, оценив правый берег в её начале как малоперспективный в качестве стояночного. Киря достал трубу:

\diagdash Паш, на! У тебя зрение получше.

\diagdash Как тут чего?

\diagdash Крышки сними, окуляр крутится сзади.

\diagdash О! Туда!\mdash Пашка приник к трубе.\mdash Туда!!!

\diagdash Чё там?

\diagdash Мыс! Прямо по курсу!

<<Какая неожиданность>>,\mdash подумал, мысленно усмехаясь, Адмирал.\mdash <<Конечно на мыс>>.

Спустя совсем немного времени ребята оказались напротив мыса, где виднелась старая баня из полиэтиленовой плёнки. Сам мыс был хорош\mdash высокие сосны, отсутствие кустарника наверху, снизу в воде\mdash камни. Типичная такая карельская стоянка, не чета верхневолжским. Выхода к воде со стоянки не было. 

\diagdash Шурик, где заберёмся?

\diagdash Давай напролом к мысу, тут сто процентов где\sdash то должен быть хороший спуск и проще его найти с~суши. Я~заберусь наверх и разведаю!\mdash они подошли к~юго\sdash западной оконечности мыса и Адмирал вскарабкался на~небольшой обрыв, опираясь на весло.

\diagdash Ёклмн, вещи тут таскать отстойно.\mdash начал Паша. 

\diagdash Спокуха, я в разведку, вы почильте тут пока! 

Шурик быстро прошёлся по~мысу. Место было шикарным! В~глубине было местечко под пару палаток, в~наличии имелось старое костровище, остатки бани, скамейки из~брёвен. На самом мысу было ещё одно более менее ровное место под палатки, спуск к~воде уходил к~северной части мыса, обращенной к~берегу озера. Шурик разведал место для причаливания и спустя ещё пару минут они подвели туда свои корабли. Под две байдарки места не хватило, так что выгружались по очереди. Из~минусов стоянки, конечно, было то, что таскать вещи к костру\mdash далековато. Байдарки Адмирал велел далеко не тащить, а перевернуть для просушки тут же, в~паре метров от~<<причала>>.

Наконец команда перетаскала вещи наверх, передохнули. Пока то да сё, Серёга и Паша успели забить место под палатку недалеко от костра, подальше от~воды вглубь мыса:

\diagdash Черти, а под адмиральскую палатку место?\mdash возмутился Шурик.

\diagdash Ну извиняй!

\diagdash На мысу комарья меньше будет!\mdash Замполит выбрал место под их с Адмиралом палатку на самом мысу, на второй ровной площадке.

\diagdash Я запомнил.\mdash медленно процедил Адмирал.

Команда быстренько натаскала дровишек, Шурик собрал костровую перекладину из железных уголков и~подвесил над огнём свои шикарные овальные котелки, которые он приобрел на замену двум старым ещё в~2020\sdash м году на~выставке <<Рыболовство и~охота>>, как раз за пару месяцев до~всеобщего карантина по ковиду$\ldots$ Котелки ждали своего часа целых 3 года\mdash из нержавейки, удобные, с~крышками, складывающиеся <<матрёшкой>> друг в друга\mdash просто мечта походника. Хотя, по правде сказать, Шурик мечтал о титановых котелках, но цены на них были явно не~на~его нии\sdash шную зарплату$\ldots$

$\ldots$Спустя час у них уже почти всё было готово к~ужину. %Паша достал из мешка купленное у Олега в Гирвасе сало, а Замполит, сидя у костра в походном складном кресле, записывал видеозарисовки:
Замполит, сидя у костра в походном складном кресле, записывал видеозарисовки:

\diagdash Так, на костре у нас там суп и чай! А макароны$\ldots$ Уже сварены, остывают! Ой остывают, Шурик!

Паша немедленно подыграл:

\diagdash Хоп, у нас тут скромное меню, да. Мы люди бедные, кочевники, в походе, можно сказать! Питаемся чем бог пошлёт, перебиваемся чем можем$\ldots$ Так, чё это? Убери, я~нарежу сальца!\mdash он расчистил себе импровизированный столик, сооружённый из складного стульчика и фанерного байдарочного сиденья.

\diagdash А боги послали кусочек сала, Ы!\mdash Шурик развалился во втором походном складном кресле в~предвкушении ужина.\mdash Ну, давайте, парни, подтягивайтесь, будем трапезничать!

Адмирал нарезал апельсин, достал, разлил: 

\diagdash Так, мужики, под горячее! Короче, без долгих речей! На маршрут встали\mdash молодцы, под дождём не скисли\mdash красавцы, два озера и два канала прошли\mdash ваще герои! Ну,~будем! Два отрывистых и одно раскатистое!!!

\diagdash {\large УРА, УРА, УРА\sdash А\sdash А!!!}\mdash грянула команда, послевкусие апельсинчика опосля тёмного выдержанного 7\sdash летнего рома было шикарным.

\diagdash Да, Венеция была хороша сегодня.\mdash подытожил Серёга.\mdash Зачем только эти каналы прорыли?

\diagdash Мне тоже интересно, но, скорее всего, это надо бабушек/дедушек спрашивать, и то не факт, что знают. Это~ведь копали, я так думаю, уже более полувека назад.\mdash отозвался Шурик.\mdash На наш век хватило увидеть и это здорово, я никогда не бывал в таких местах.

\diagdash Я не первый раз в Карелии,\mdash Киря нарезал ещё апельсина.\mdash Но такого я тоже ещё не видал! Чтобы прям натуральный канал, брёвна и всё такое$\ldots$

\vspace{0.8cm}
$\ldots$Команда отдыхала душой и телом. Место, на~котором Шурик, согласно своему правилу, оказался в~пятый раз, хотя в реальности первый, было прекрасным. Настолько прекрасным, что как будто специально созданным всеми богами для туристической стоянки. Красота, полное отчуждение от цивилизации вместе с бравой командой, а~вокруг$\ldots$ от природы дух захватывает! Высокие янтарные сосны, гладкие сереющие камни\sdash валуны, прибрежный зеленый камыш, оранжевое закатное солнце, окрашивающее всё вокруг в свои тёплые тона. Адмирал особенно любил начало августа, справедливо полагая это время года наиболее благоприятным для походов, сплавов их стиля. Он даже временами мечтал, чтобы была в мире сила, способная сделать вечный август\mdash высокое звёздное небо, сбор урожая, тёплые деньки, не жаркое солнце. Словом, среди всех времён года начало августа Адмирал любил больше~всего.

Вокруг медленно спускалась ночь, но было ещё довольно светло\mdash характерная черта северных широт. На китайских ролексах Адмирала в окошечке даты стало появляться 1 августа. Мобильной связи не~было, как он и~хотел. Первый день прошёл как надо, он был доволен началом похода. Выключив ненавистные будильники на~6~утра в~телефоне, Адмирал с наслаждением понял, что завтра он проснётся тогда, когда проснётся\mdash когда организм сам пожелает этого\mdash и~ни~одна дурацкая мысль о~будильнике и~необходимости вставать на~работу не~посмеет омрачить эту неделю. 

\begin{center}
	\psvectorian[scale=0.4]{88} % Красивый вензелёк :)
\end{center}