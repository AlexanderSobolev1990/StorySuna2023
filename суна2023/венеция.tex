\chapter{Венеция}
%\corner{64}
\vepsianrose

Погрузка заняла у них минут 40. Уклон местности выматывал с непривычки. Шурик нацепил на ноги гидроноски на плотной подошве и они с Замполитом спустили свои корабли на воду. Торжественный момент! Адмирал ждал его три года. Три года! Три года они не~ходили на реку! У него чуть не~кружилась голова, хотя махнули лишь на донышке. Он~снова на воде! Вот она, мокрая!!! Чистая!!! Карельская!!! Он~мечтал об этом очень, очень давно. И это свершалось\mdash сейчас, прямо на глазах.

В гидроносках шлёпать было гораздо удобнее, чем в~чём бы то ни было ещё\mdash можно было сразу с берега входить в воду и так же легко выходить, как босиком, но не боясь пораниться о камни, а толстая гидроткань защищала от~холода, напитываясь водой, которой передавалось тепло человеческого тела. Шурик, стоя в воде и ловя кирину байду, бросил Замполиту:

\mdash Кирь, привяжи за чалку\sdash то\footnote{верёвка, трос для причаливания}.

\mdash У меня нету$\ldots$

\mdash Да хар\'{о}ш! А чем чалиться собрался?!

Киря пошёл искать чалку в вещах, а Шурик, стоя по~колено в бодрящей водичке, стал принимать от Руслана и Паши вещмешки и раскладывать их по~грузовому отсеку. Делать это стало труднее, чем обычно, потому что в~этот поход друзья прикупили байдарочные фартуки\mdash они уменьшили размеры погрузочного проёма. Шурик, скрежетая зубами, раз за~разом пробовал по\sdash иному, компактнее, впихивать снарягу. Лезло плохо, но Шурик знал по опыту\mdash всего 2 дня и продуктов подуменьшится и всё будет влезать гораздо лучше.

\mdash Давай их снимем, а наденем только перед порогами?\mdash предложил подошедший Замполит. 

\mdash Умный, как сто китайцев! А куда сам фартук денем? 

\mdash На корму?

\mdash У меня там спасы, вот их точно только на порогах наденем. Не, надо крепить фартуки, скоро дождь, нальёт по кильсон в трюмы!

\mdash Ну, пожалуй$\ldots$

Серёга и Киря тоже, чертыхаясь, распихали всю снарягу под борта и по отсекам. У них была двушка и~барахла влезло меньше. Начался дождь, повисла сплошная облачность, солнце не пробивалось. Шурик, поморщившись, вытащил из кармана брезентовой штормовки дождевик\mdash он всегда лежал у него в левом кармане штормовки, а~в~правом лежала кружка для отчерпывания воды и сугрева\mdash такова была привычка сплавщика. Команда тоже приоделась в~дождевики, кто во что.

\mdash Готовы???

\mdash ДА!

\mdash Хрена два! Пойду проверю что ничего не забыли!!!\mdash Шурик пошёл наверх по склону к поляне, где собирали байды, чтобы самолично убедиться в отсутствии оставленных вещей. Спустя пару минут, ничего не найдя, он спускался к~озеру, аккуратно ступая к мокрых гидроносках.

\mdash Ну, отчаливаем! Серёга, Паш, отвязывайте чалки!\mdash сказал Адмирал и они с Замполитом заняли свои капитанские места\mdash в корме каждый своей байдарки. Поначалу только, неопытному байдарочнику, кажется это несправедливым\mdash мол, как же так, капитан и сидит на корме, где\sdash то на~галёрке. А все красоты реки, первым всё усматривает первый сидящий\mdash матрос. А ему, капитану, как бы всё позже доходит. Лишь спустя несколько речных переходов приходит понимание, что нет, всё правильно\mdash капитану с~кормы виднее всё на свете. Он лучше видит с кормы курс корабля, лучше усматривает куда сносит байдарку, как надо грести, чтобы обойти препятствия, как маневрировать. Да,~спины экипажа, конечно, закрывают вид прямо, но это у Шурика компенсировалось высоким ростом и тем, что когда ходили по спокойным рекам, он, плевав на все правила, садился не~внутрь байдарки, а на кормовой шпангоут. Таким образом, сидел он очень высоко, далеко видел, а насчёт устойчивости не волновался\mdash матросы и вес груза в трюмах обеспечивали достаточный запас от переворачиваемости. Сейчас же такой фокус не прошёл\mdash сверху на фальшборта надели фартук и сидеть на кормовом шпангоуте стало невозможно\mdash тогда бы повредился фартук. Шурику это было непривычно\mdash он за все свои предыдущие сплавы байдарочные свыкся с~мыслью, что всегда высоко сидит, далеко глядит, а~тут, как в первых сплавах, придётся сидеть низко\mdash хуже обзор воды. А на маршруте\mdash пороги. Ничего, подумал Адмирал, в этот раз с ним был весь его предыдущий опыт, который, как известно, не пропьёшь. Он с силой оттолкнулся от каменистого берега\mdash Cплав начался!

На небесах, тем временем, как кран открыли\mdash лило просто ужасно. У Руслана, который опрометчиво подумал, что, быть может, дождик скоро закончится, лёгкая курточка уже через 5 минут промокла насквозь. Справа по борту прошли торчащие изо дна озера жерди, на которых были растянуты то ли сети, то ли что\mdash похоже, тут разводили рыбу. 

У западного конца озера Вендюрское Шурик стал поглядывать в навигатор и прикидывать\mdash насколько велики шансы, что команда линчует его, если канал, по которому он хотел сократить путь примерно в полтора раза, попав в речку Кулапдеги не напрямую, а через Сяргозеро, окажется непроходим для байдарок. Минуты раздумья тянулись за греблей томительно долго\mdash нетренированные мышцы рук, получившие убойный натиск от весла, ныли и не давали сосредоточиться. Наконец, Шурик решил\mdash была\sdash не была\mdash попробует ломиться через канал!

К каналу подошли достаточно скоро и особо не искали его начало\mdash GPS был на контроле у Адмирала. Народ выразил ликование от такого сооружения, представшего их взору. Канал был шириной примерно метров 8, а вот глубина оказалась крайне мала\mdash не более 30 сантиметров. Стенки канала были сделаны из брёвен, из воды показывалось 4\mdash 5 венцов, то есть вода была достаточно низкой, понял Шурик. Состояние брёвен канала было максимум ещё лет на 5\mdash 7, неизвестно когда их обновляли в последний раз$\ldots$ да и будут ли делать это снова\mdash кому это нужно?

\diagdash ШУРИК! Там мель!!! 

\diagdash Полундра, покинуть байду!\mdash скомандовал Адмирал и далее по каналу они уже пойти шлёпая гидроносками по дну. Сверху было мокро от дождя, снизу от канала. <<Шикарное начало>>,\mdash подумал Шурик.

\diagdash И мне вылазить?\mdash спросил Руслан. 

\diagdash Сиди, вроде осадка позволяет пока что!\mdash отозвался Адмирал и они с Пашкой потихоньку стали проводить байду на чалке, стремясь не напороться на коряги и камни. 

Надо было передохнуть. Впереди показался конец канала, весь заросший тростником. Команда вышла в Сяргозеро. Пора было передохнуть\mdash канал кончился, дождь выматывал, мышцы с непривычки ныли, а Руслан промок до нитки.

\diagdash А ну, пацаны! Впереди стояночное место, что ли?\mdash изрёк Шурик.

\diagdash Да вроде бы.\mdash отозвался Паша.

\diagdash Правим туда, перекур. 

\diagdash Ура, ура, ура! Перекур!\mdash воскликнул Замполит, тоже порядком уже нагрёбшийся.

\diagdash Кирь, тебе привет от пульмонолога, какой перекур?\mdash язвил Шурик.

\diagdash Шурик, иди ты?

\diagdash Ы-Ы-Ы! Сигарилочки будешь?

\diagdash Вы поглядите, вы поглядите! Совращают!

Шурик с Замполитом задымили, едва оказавшись на берегу. Серёга и Паша пошли осматривать стоянку, расположившуюся на левом берегу Сяргозера сразу после конца канала. Место было неоднозначным, но вполне себе подходящим под стоянку\mdash мусора почти не было, костровище имелось, полянка была хороша. 

\diagdash Руслан, переодевайся, ты же совсем промок!

\diagdash Да, надо$\ldots$ Чёрт, мою герму заложили мешками, помоги?

\diagdash Давай, держу байду, доставай.\mdash Шурик придержал байду, а Руслан вытащил шмот и переоделся, отжав куртку. 



\begin{center}
	\psvectorian[scale=0.4]{88} % Красивый вензелёк :)
\end{center}
