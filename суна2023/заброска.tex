\chapter{Autobahn}
%\corner{64}
\vepsianrose
%\fancyhead[LE]{\fancyplain{}{\bfseries \parttitle}}
%\fancyhead[RO]{\fancyplain{}{\bfseries \rightmark}}

Шурик, облаченный в штормовку, сидел в машине, на~улице собирался дождь. Утро старта, наконец\sdash то! Багажник был почти под завязку\mdash упакованная байдарка занимала дочерта места, пришлось сложить половину заднего ряда сидений. Мыслей как будто бы не было, Адмирал парил где\sdash то в неведомых облаках. Ему с одной стороны было как\sdash то некомфортно и~даже стыдно оставлять жену с~ребёнком одних дома, а~самому отправляться в~путешествие, а~с~другой стороны он понимал, что жить не~может уже без реки, тем более он не был в~походе целых три года. В своём сознании Шурик был до~предела измотан всем, всем что его окружало\mdash грёбаными электричками, работой, семьёй, медиапространством и вообще всем плохим, что произошло в мире, стране и в целом в его жизни за~последние 2\thinspace\nobreakdash---\thinspace3 года. И~эти десять дней, на которые он вырывается из~всего этого\mdash были для него как глоток свежего воздуха, как припадение к чистейшему горному роднику, как некий абсолют желаний задолбавшегося человека\mdash не видеть вокруг себя толпы людей из транспорта и просто заниматься простыми древними вещами\mdash разводить огонь, готовить еду, плыть по реке$\ldots$ Очнувшись, Адмирал глянул на~часы\mdash пора! Он повернул ключ зажигания, завёл машину, вбил в~навигаторе адрес Кири, он~же~адрес <<пивотеки>>, отписался, что выехал и нажал на~газ. Заброска началась.

Спустя минут 30 он очень удачно запарковался аккурат перед подъездом Кири, тот вышел c баулами буквально через минуту.

\diagdash Кирь, опять вещей\mdash мама дорогая!

\diagdash Не гунди, Шурик, щас ещё один вещмешок.\mdash они грузились под накрапывавшим дождичком, который как всегда так невовремя начался.

\diagdash Тебе помочь?

\diagdash Не, сам$\ldots$\mdash Замполит пошёл за очередными баулами.

\diagdash Ладно, жене привет.

Безумно хотелось закурить, но Адмирал не поддался соблазну. Занавеска на 3\sdash м этаже подёрнулась. Замполит семимильными шагами вышел из подъезда, запихал последнюю герму в багажник и плюхнулся на~переднее пассажирское:

\diagdash Ну чё, погнали?

\diagdash Попрощался? Погнали$\ldots$

Дождь усиливался. Уже когда друзья оказались на~МКАДе\footnote{Московская кольцевая автомобильная дорога.}, зарядил настоящий ливень.

\diagdash Вы умеете выбрать погодку, тащ Адмирал!\mdash задумчиво произнёс Замполит. 

\renewcommand*{\thefootnote}{\fnsymbol{footnote}}
\diagdash Так, ля, Замполит! Срочно звони Роман Менделичу\footnote[1]{Роман Менделевич Вильфанд\mdash метеоролог, директор <<Гидрометцентра России>>.}, договаривайся там, короче, делай шо хош, чтоб погодка была тип\sdash топ!

\diagdash Не иначе! Приколи под таким дождём плыть 7 дней?

\diagdash Не накаркай!!!

Друзья подкатили во двор пашкиного дома и~перегрузили шмот из его машины в шурикову:

\diagdash Пацаны, столько шмотья, по\sdash моему компоты где\sdash то здесь, а бензопилу не взяли!\mdash иронизировал Паша.\mdash Мне~ж~совсем места не останется!

\diagdash Так, давай по-другому распихаем$\ldots$

\diagdash Как ни распихивай, уже под потолок барахла!

Старательно утрамбовав снарягу, сплавщики наконец\sdash то выехали из Москвы и вырулили на платную трассу М\sdash11\mdash Шурик незадолго до сплава приобрёл транспондер для бесконтактной оплаты проезда.

\diagdash Замполит, набери Серёге, где они там?\mdash попросил Адмирал узнать что там со вторым экипажем.

\diagdash Момент$\ldots$

Оказалось, что Серёга с Русланом ехали практически параллельно им, но по обычной, бесплатной дороге. По~расчётам Шурика выходило, что разница в их прибытии в~Старую Ладогу составит от 40 минут до часа.

Дорога была шикарна\mdash как масло. Троица проехала пункт оплаты, где транспондер, пикнув, открыл им шлагбаум. Шурик бодро рванул вперёд, утапливая педаль газа в пол и поздно переключаясь на повышенную передачу, отчего движок знавшего и лучшие годы 10\sdash летнего фольксвагена гулко выл. 

\diagdash Э, Шумахер, блин, я хочу живым домой вернуться!\mdash отвлёкся Киря от телефона.

\diagdash Не очкуй, аллес унтер контролле!\mdash отозвался Шурик.\mdash Руссише аутобанен приветствуют вас, за бортом +22 градус\'{а} выше н\'{о}ля, полёт проходит норм\'{у}льно, расчётное время прибытия\mdash айн унд цванцихъ часен ровно. В полёте вам будут предложены напитки и бутерброды.

\diagdash Да?! Это где?\mdash поинтересовался Паша, не взявший с собой перекуса.

\diagdash На заправке, Ы-ы-ы!

Дождь помаленьку прекращался по мере их удаления от Москвы\mdash скорее они просто выехали из его полосы. Но~трасса по\sdash прежнему была мокрой, поэтому Шурик всё же не лихачил. За трёпом потихоньку долетели до Твери, где заправились бензом и перекусили на заправке.

\diagdash Лишь бы не как в прошлый раз с погодой.\mdash изрёк Паша, доедая бутерброд.

\diagdash Да всё наладится! Смотри, уже и дождя нет, щас солнце выглянет!\mdash потягивал Замполит кофе с сигареткой.

После заправки парни сели на хвост какой\sdash то бмв\sdash ухе и~в~таком режиме допилили до съезда с платной М\sdash 11 на старую питерскую трассу М\sdash10, с которой они вскоре свернули после Чудово. Шурик порядком притомился за~рулём, потому что поддержание высокоскоростного режима требовало и высокой концентрации внимания. 

И тут Замполит внезапно выпалил:

\diagdash ШУРИК! Едем обратно!!!

\diagdash Утюг не выключил?\mdash лениво ввернул Адмирал.

\diagdash ХУЖЕ!!! Я забыл ложку и миску свои титановые!

%\diagdash Кирь, ну всё, придется жменькой суп хлебать или того хуже\mdash как простые смертные, из нержавейки.
\diagdash Кирь, ну всё, придется хлебать суп как простые смертные, из нержавейки.

\diagdash Шурик, катастрофа, я собирался, собирался, вот как чувствовал, что что\sdash то забыл и вот тебе раз! Мой тита\sdash а\sdash ан!

\diagdash Забей, чё там дальше по карте?

\diagdash Ща гляну$\ldots$\mdash Замполит уткнулся в карту.\mdash Кириши какие\sdash то$\ldots$

\diagdash А? КирЮши?

\diagdash Ки-ри-ши, тащ Адмирал. О! Там и спортивный магаз есть! Приколи?

\diagdash А ты думал? Растёт благосостояние народа! В~КирЮшах есть спортивный магазин, никогда бы не~подумал.\mdash вовсю иронизировал Шурик.

\diagdash А давайте там покушаем нормально, а?\mdash отозвался с~заднего сиденья Паша.

\diagdash Лад\'{ы}! Кирь, выбери там чё нить?

И Замполит проложил по~навигатору ответвление с~маршрута в эти самые Кириши, которые вскоре показались после очередного поворота. Отзвонился Серёга\mdash они плелись далеко позади на М\sdash 10:

\diagdash Пацаны, Руслан кружку забыл!

\diagdash Классика! Купим ему, мы как раз в спортивный щас заедем\mdash Киря тоже забыл!

%\renewcommand*{\thefootnote}{\arabic{footnote}}
\renewcommand*{\thefootnote}{\fnsymbol{footnote}}
\setcounter{footnote}{0}
\diagdash Прикол! Ладно, мы тут пока в пробках чилим\footnote{Чилить (от англ. chill\mdash прохлаждаться)\mdash расслабленно отдыхать.}$\ldots$

Киря, Шурик и Паша запарковались у спортивного и достаточно быстро прикупили всё, что забыли\mdash благосостояние народа таки растёт. А потом, перекусив в~общепите, немного передохнули и снова двинулись в путь\mdash до Старой Ладоги оставалось совсем немного, порядка 60~километров.

И вот, спустя где\sdash то час, который прошёл относительно спокойно\mdash на трассе местного значения\mdash друзья проехали Волхов, где ужасным дымом дымило множество труб заводов. Совсем скоро после этого показалась и~Старая Ладога, их конечная точка на сегодня. Уже стемнело, когда путешественники проехали мимо Староладожской крепости, которую хотели завтра посетить. Она подсвечивалась в ночи прожекторами и выглядела монументально, величественно. Как будто хоть сейчас через её ворота готова была показаться княжеская конная дружина. Высокие крепостные стены и~башни, покрытые сверху деревянными конусами, смотрелись ну просто великолепно\mdash Шурику всегда нравилась такая архитектура и сейчас он, даже на~мгновение оторвав взгляд от дороги на крепость, ощутил прилив позитива. 

Команда проехала дальше и~свернула к гостинице, во дворе которой они запарковались в 21:30, разойдясь с расчётным временем прибытия всего на~полчаса\mdash нормально, с учётом того, что ещё заезжали в~Кириши.

\diagdash Так, пацаны, я за пенным!\mdash огласил Паша, вылезая из машины.\mdash Сил нету!

\diagdash Пошли!

Вскоре друзья расположились на балкончике номера, задымили и, откинувшись на табуреточках, с~наслаждением прихлебнули светлого, наблюдая вдалеке над лесом восход полной Луны. Прошло около 40 минут, пацаны расслаблено кайфовали на балконе, болтая о всякой ерунде:

\diagdash Полнолу\sdash у\sdash уние! Всякая не\sdash е\sdash ечисть вылезает из~угло\sdash о\sdash ов! Вурдалаки и русалки подстерегают пу\sdash у\sdash утников, забредших в неведомые дали!!!\mdash заворачивал Шурик.

\diagdash Хар\'{о}ш! Смотри чтоб из матраса твоего нечисть не вылезла!\mdash сказал вошедший в комнату Серёга и~стал, переворачивая матрас, пристально осматривать место ночёвки на предмет клопов\mdash друзья и не заметили как подрулила вторая часть команды.

\diagdash Ну как? Есть тараканчики?\mdash лениво протянул Шурик.

\diagdash Да вроде нет$\ldots$ Ну и дыру ты выбрал на ночёвку! 

\diagdash Забей, нам просто поспать. Другое всё было или занято или сильно дороже. Выдыхай, мы тут всего до утра.

Друзья продолжили посиделки и прочий трёп, который, впрочем, достаточно скоро стих и в комнате раздался чей\sdash то раскатистый храп. Шурик сквозь сон подумал, что лишь бы это не Киря, ведь в сплаве им делить одну палатку на двоих.

\begin{center}
	\psvectorian[scale=0.4]{88} % Красивый вензелёк :)
\end{center}