\chapter{Autobann}
%\corner{64}
\vepsianrose
\fancyhead[LE]{\fancyplain{}{\bfseries \parttitle}}
\fancyhead[RO]{\fancyplain{}{\bfseries \rightmark}}

Шурик сидел в штормовке в машине, утро старта, на улице собирался дождь. Багажник был почти под завязку\mdash упакованная байдарка занимала дочерта места, пришлось сложить половину заднего ряда сидений. Мыслей как будто не было, Адмирал парил где\sdash то в неведомых облаках. Ему, с одной стороны было как\sdash то некомфортно и стыдно оставлять жену с ребёнком одних дома, а самому отправляться в путешествие, а с другой стороны он понимал, что жить не может уже без реки, тем более он не был в походе целых два года. В своём сознании Шурик был до предела измотан всем, всем что его окружало\mdash семьёй, работой, грёбаными электричками, медиапространством и вообще всем плохим, что произошло в мире, стране и в целом за эти 2 года. И~эти десять дней, на которые он вырывается из всего этого\mdash были для него как глоток свежего воздуха, как припадение к чистейшему горному роднику, как некий абсолют желаний задолбавшегося человека\mdash не видеть вокруг себя толпы людей из транспорта и просто заниматься простыми, древними, вещами\mdash разводить огонь, готовить еду, плыть по реке$\ldots$ очнувшись, Шурик глянул на часы на руке\mdash пора! Он повернул ключ, завёл машину, вбил в навигаторе адрес Кири, он же адрес <<пивотеки>>, отписался Кире что выехал и нажал на газ. Заброска началась.

Спустя минут 30 он очень удачно запарковался аккурат перед подъездом Кири, тот вышел буквально через минуту.

\diagdash Кирь, опять вещей\mdash мама дорогая!

\diagdash Не гунди, Шурик, щас ещё один вещмешок.\mdash они грузились под накрапывавшим дождичком, который так невовремя начался.

\diagdash Тебе помочь?

\diagdash Не, сам$\ldots$

\diagdash Ладно, жене привет.

Безумно хотелось закурить, но Адмирал не поддался соблазну. Занавеска на 3\sdash м этаже подёрнулась. Киря семимильными шагами вышел из подъезда и плюхнулся на переднее пассажирское:

\diagdash Ну чё, погнали?

\diagdash Попрощался? Погнали$\ldots$

Дождь усиливался. Уже когда друзья оказались на МКАДе, зарядил настоящий ливень.

\diagdash Вы умеете выбрать погодку, тащ Адмирал!\mdash задумчиво произнес Замполит. 

\diagdash Так, ля, Замполит! Срочно звони Роман Менделичу\footnote{Роман Менделевич Вильфанд\mdash метеоролог, директор <<Гидрометцентра России>>}, договаривайся там, короче, делай шо хош, чтоб погодка была тип\sdash топ!

\diagdash Не иначе! Приколи под таким дождём плыть 7 дней?

\diagdash Не накаркай!!!

Друзья спустя час подкатили во двор пашкиного дома и перегрузили шмот из его машины в шурикову:

\diagdash Пацаны, столько шмотья, по\sdash моему компоты где\sdash то здесь, а бензопилу не взяли!\mdash иронизировал Паша.\mdash Мне ж совсем места не останется!

\diagdash Так, давай по-другому распихаем$\ldots$

\diagdash Как ни распихивай, уже под потолок шмотья!

Как всегда, старательно утрамбовав снарягу, сплавщики, наконец\sdash то выехали из Москвы и вырулили на платную трассу М\sdash11\mdash Шурик незадолго до сплава приобрёл транспондер для бесконтактной оплаты.

\diagdash Замполит, набери Серёге, где они там?\mdash попросил Шурик узнать что там со вторым экипажем.

\diagdash Момент$\ldots$

Оказалось, что Серёга с Русланом ехали практически параллельно им, но по обычной, бесплатной дороге. По расчётам Шурика выходило, что разница в их прибытии в Старую Ладогу составит от 40 минут до часа.

Дорога была шикарна\mdash как масло. Троица проехала пункт оплаты, где транспондер, пикнув, открыл им шлагбаум. Шурик бодро рванул вперёд, утапливая газ. 

\diagdash Э, Шумахер, я хочу живым домой вернуться!

\diagdash Не очкуй, аллес унтер контролле!\mdash отозвался Шурик.\mdash Дойчен аутобанен приветствуют вас, за бортом +19 градус\'{о}в выше н\'{о}ля, полёт проходит нормально, расчётное время прибытия\mdash айн унд цванцихъ часен ровно.

Дождь помаленьку прекращался по мере их удаления от Москвы\mdash скорее они просто выехали из его полосы. Но трасса по\sdash прежнему была мокрой, поэтому Шурик всё же не лихачил. За трёпом потихоньку долетели до Твери, где заправились бензом и перекусили на заправке.

\diagdash Лишь бы не как в прошлый раз с погодой.\mdash изрёк Паша, доедая бутерброд.

\diagdash Да всё наладится! Смотри, уже и дождя нет, щас солнце выглянет.





\begin{center}
	\psvectorian[scale=0.4]{88} % Красивый вензелёк :)
\end{center}