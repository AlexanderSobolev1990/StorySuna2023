{
\cleardoublepage
\phantomsection

\fancyhead[LE]{\fancyplain{}{}}
\fancyhead[RO]{\fancyplain{}{}}

\addcontentsline{toc}{chapter}{Пролог}
%\addtocontents{toc}{\vspace{-4mm}}
\section*{Пролог}

Карелия$\ldots$ что чувствует водник, слыша это слово? Это не передать словами! Сердце начинает стучать чаще, разум затуманивается! Ведь мы едем в Карелию! Карьяла, Karjala\mdash зовёт, манит, завлекает! Чистейшие сосновые леса на каменистых берегах, а под ногами мшаник или поля черники, голубики$\ldots$ Карелия это красивая и строгая северная природа, скалы, исполинские янтарные сосны, чистейший воздух, обилие ягод, грибов, рыбы, и, естественно, пороги. Куда же без них? Сотни, если не тысячи, водников каждый год посещают эти заповедные места, эти порожистые речки, заставляющие неслабо напрячься на порогах капитанов и впередсмотрящих туристких судёнышек. Но этот небольшой экстрим с~лихвой окупается и природой, окружающей сплавщиков на маршруте, и дарами природы, и чувством единения с карельской природой\mdash северной, порою суровой и~жестковатой, а порою ласковой и приветливой, отчего делается так хорошо на душе, что не передать словами; уха на свежевыловленной рыбе, вкуснейшие ягоды, пьянящий своей чистотой воздух, шум порогов, белая пена и валы которых заставляют тебя испытать порою животный страх и чувство преклонения перед силой стихии\mdash за этим едут сюда люди, которым опостылел город, кто хочет духовно очиститься, опустошить сознание, дать природе омыть тебя бодрящей водой и смыть городскую грязь\mdash физически и духовно. Это место\mdash место припадения к истокам, место силы, место древа жизни. Такой представлялась мне Карелия. Такою она и оказалась!  

Откуда есть пошла Земля Русская? Да не скажет никто, все эти теории и теоретики\mdash сплошное надувательство. Дальше нескольких сотен лет назад никто не скажет наверняка что и как там было. Считается, что Старая Ладога была местом, куда был призван княжить Рюрик. Впрочем тут же вам скажут, что, скорее всего, этим местом должен быть Великий Новгород и пошло поехало\mdash куча теорий. Но все они, конечно же, не на пустом месте. Считается, что так называемое призвание варягов на Русь имело место и Рюрик все же сначала обосновался в Старой Ладоге. Именно поэтому мы решили остановиться в этой городке по пути в Карелию, поскольку интересно и крепость посмотреть и ехать до Карелии от Москвы что называется за 1 присест\mdash очень тяжело. Посему, решили ехать с ночёвкой в Старой Ладоге, отведя утро второго дня на осмотр крепости и окрестностей. 

Отчего же вдруг такой поворот? С чего мы отошли от привычного бассейна Мологи\cite{СоболевВепсскаяЛетопись}? Всё просто\mdash наиболее интересные реки мологского бассейна мы прошли\mdash Лидь, Чагода, Чагодоща, Горюн, Песь. Непокорённой осталась, разве что, Кобожа, но характер берегов её как\sdash то не способствовал склонению нашего желания в её сторону. Хотелось чего\sdash то принципиально нового! Последнний раз мы ходили в сплав в 2019 году, о чём даже была вырезана памятная надпись на беседке у антистапеля$\ldots$ и как накаркал\mdash ситуация в мире пошла как по наклонной после этого. Сначала ужас и сумасшествие пандемии 2020\nbdash 2021 года, сопровождаемое чехардой смены работ. Потом ситуация с вооруженным конфликтом у наших западных границ. Всё это, конечно же, счастья не добавляло ни капельки. 

Давно уже я присматривался к карельским речкам. Хотелось открыть для себя Карелию несложной и красивой рекой. Выбор мой пал на интересный маршрут <<Сунская цепочка>>\mdash кольцевой маршрут, замкнутый на посёлок Гирвас. Далековато, конечно, от Москвы. Но кого и когда это останавливало? Постепенно я всё более свыкался с мыслью, что бригада моя, костяк которой постепенно формировался годами, сможет осилить такое путешествие с двухдневной заброской и выброской. Итак, цепочка. Маршрут привлекал ещё и тем, что начало его относительно непопулярно. Карелия всё же намного более многолюдна в плане сплавщиков, чем чагодощенский край. А вот на сунской цепочке народу поменьше, судя по отчётам.

%\vspace{5mm}
\begin{flushright}
	\textit{Соболев А.А., 2023 г.}
	%\copyright~Соболев~А.А.,~Москва,~27.08.2015
\end{flushright}

}
\fancyhead[LE]{\fancyplain{}{\bfseries \parttitle}}
\fancyhead[RO]{\fancyplain{}{\bfseries \rightmark}}