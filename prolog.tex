{
\cleardoublepage
\phantomsection

\fancyhead[LE]{\fancyplain{}{}}
\fancyhead[RO]{\fancyplain{}{}}

\addcontentsline{toc}{chapter}{Пролог}
%\addtocontents{toc}{\vspace{0mm}}
\section*{Пролог}

Карелия$\ldots$ что чувствует водник, слыша это слово? Сердце начинает стучать чаще, разум затуманивается! Мы~едем в Карелию! Карьяла, Karjala\mdash зовёт, манит, завлекает! Чистейшие сосновые леса на каменистых берегах, а под ногами мшаник или поля черники, голубики$\ldots$ Карелия это красивая и~строгая северная природа, скалы, исполинские янтарные сосны, чистейший воздух, обилие ягод, грибов, рыбы, и, естественно, пороги. Куда же без них? 

Сотни, если не тысячи, водников каждый год посещают эти заповедные места, эти порожистые речки, заставляющие неслабо напрячься на порогах капитанов и впередсмотрящих туристких судёнышек. Но этот небольшой экстрим с~лихвой окупается и природой, окружающей сплавщиков на маршруте, и дарами природы, и чувством единения с карельской природой\mdash северной, порою суровой и~жестковатой, а порою ласковой и приветливой, отчего делается так хорошо на душе, что не передать словами; уха на свежевыловленной рыбе, вкуснейшие ягоды, пьянящий своей чистотой воздух, шум порогов, белая пена и валы которых заставляют тебя испытать порою животный страх и чувство преклонения перед силой стихии\mdash за этим едут сюда люди, которым опостылел город, кто хочет духовно очиститься, опустошить сознание, дать природе омыть тебя бодрящей водой и смыть городскую грязь\mdash физически и духовно. Это место\mdash место припадения к истокам, место силы, место древа жизни. Такой представлялась мне Карелия. Такою она и оказалась, ласково приняв нас и~тёплыми дождями, и ярким августовским солнышком, а~иногда и порывистым ветром, пробирающим до костей даже в штормовке.  

Откуда есть пошла Земля Русская? Да не скажет никто, все эти теории и теоретики\mdash сплошное надувательство. Дальше нескольких сотен лет назад никто не скажет наверняка что и как там было. Считается, что Старая Ладога была местом, куда был призван княжить Рюрик. Впрочем тут же вам скажут, что, скорее всего, этим местом должен быть Великий Новгород и пошло поехало\mdash куча теорий. Но все они, конечно же, не на пустом месте. Считается, что так называемое призвание варягов на Русь таки имело место и Рюрик всё же сначала обосновался в Старой Ладоге. Именно поэтому мы решили остановиться в этом городке по пути в~Карелию, поскольку интересно и крепость посмотреть, и ехать до Карелии от Москвы что называется за 1 присест\mdash очень тяжело. Посему, решили ехать с ночёвкой в Старой Ладоге, отведя утро второго дня на осмотр крепости и окрестностей. 

Отчего же вдруг такой поворот? С чего мы отошли от~привычного бассейна Мологи\cite{СоболевВепсскаяЛетопись}? Всё~просто\mdash наиболее интересные реки мологского бассейна мы уже прошли\mdash Лидь, Чагода, Чагодоща, Горюн, Песь. Непокорённой осталась, разве что, Кобожа, но характер берегов её как\sdash то не способствовал склонению нашего желания в её сторону. Хотелось чего\sdash то принципиально нового! Последнний раз мы ходили в сплав в 2019 году, о~чём даже была вырезана памятная надпись на беседке у~антистапеля: 

{\centering\LARGE{\fontspec{NorseRus}{%
			Посл.\\раз\\2019\\
}}}

\newpage
И как накаркал\mdash ситуация в мире пошла как по~наклонной. Сначала ужас и сумасшествие пандемии 2020\thinspace\nobreakdash--\thinspace 2021 года, сопровождаемое чехардой смены работ, удалёнкой, тихим схождением с ума от удалёнки и всеобщей окружающей обстановки. Потом ситуация с~вооруженным конфликтом у~наших западных границ. Всё~это, конечно~же, счастья не~добавляло абсолютно. Закрадывалась мысль\mdash а может надпись <<Посл. раз 2019>> и вправду уже стала пророческой? Но нет, рановато! Кто\sdash то там лает, а караван идёт\mdash волю к~путешествиям не~заглушишь ничем!

Вспоминаются ставшие для меня чем\sdash то сакральным строки из произведения <<Географ глобус пропил>>: <<\textit{Конечно, обалдели все и от меня, и от такого похода. Всем домой хочется. Половина клянётся, что больше ни~в~жисть из города не вылезет. Но всё это\mdash пустые обещания. <...> Сейчас все хотят одного: тепла, уюта, покоя. Но отрава бродяжничества уже в~крови. И~никакого покоя дома они не~обретут. Снова начнёт тревожить вечное влечение дорог\mdash едва просохнет одежда и отмоется грязь из\sdash под ногтей. Я~это знаю точно. Я~и~сам сто раз зарекался\mdash больше ни ногой. И~где~я~сейчас?}>>\cite{ГеографГлобусПропил}.

Прочитав <<Географа>> после своего первого байдарочного сплава, я был поражён этим произведением, насколько оно легло на душу, настолько пришлось близким и понятным. Однако тогда приведённые выше строки не~вспыхивали так ярким пламенем, безумным пожаром, как сейчас\mdash разве мог я тогда, в 2015 году зарекаться, что, мол, ни ногой больше? А~в~2019 вдруг вырезаю <<Посл. раз>>. И вот, в 2023 я снова сижу поздним вечером у себя на кухне, разложив на столе святая святых\mdash топокарту маршрута и волнительно прикидываю места стоянок, стратегию прохождения и штурм порогов$\ldots$ 

Одно верно\mdash отрава бродяжничества в крови и этого не отнять\mdash поход 2023 года только подтвердил это. Стоит только позволить себе мимолётную слабость\mdash на 1 секунду закрыть глаза и подумать: <<Вот щ\sdash щ\sdash щас бы на реку$\ldots$>> и~всё, ты уже испытываешь то самое сладкое и~томительное \textit{<<вечное влечение дорог>>}$\ldots$

Давно уже я присматривался к карельским речкам. Хотелось открыть для себя Карелию красивой, но в то же время несложной рекой. То и дело в описаниях маршрутов встречались пороги 3\sdash й категории, что в мои планы не~входило, ровно как и обносы. Искал я, короче говоря, реку 2\sdash й категории где\sdash нибудь в южной или центральной Карелии. В итоге выбор мой пал на интересный кольцевой маршрут <<Сунская цепочка>>, замкнутый на посёлок Гирвас. Далековато, конечно, от Москвы. Но кого и когда это останавливало? Постепенно, за разговорами и обсуждениями маршрута, я всё более свыкался с мыслью, что команда моя, костяк которой постепенно формировался годами, сможет осилить такое путешествие с двухдневной заброской и~выброской. Итак, активная, с порогами, основная часть маршрута лежит по реке Суна. Это самая длинная река Карелии. Маршрут этот привлекал ещё и тем, что начало его относительно непопулярно. Карелия всё же намного более многолюдна в плане сплавщиков, чем чагодощенский край, так полюбившийся нам. А вот на <<Cунской цепочке>> народу поменьше, чем на других маршрутах, судя по~отчётам. Возможно это, в совокупности с несложными порогами 2\sdash й категории, так привлекло меня и сложило вполне определённую картину будущего путешествия. Изучение топокарты и последующие пару дней висения на телефоне развеяли сомнения, что посёлок Гирвас станет для нас базой заброски и выброски.

По команде в этот раз выходило человека 4 точно, остальные под вопросом$\ldots$ С.Ю.(см.~\cite{СоболевВепсскаяЛетопись}) сказал, что на пороги не пойдёт, плюс дорога дальняя очень. Что ж, это было не ударом для нас, но вполне ожидаемым решением. С.Ю. был завсегда готов, что называется, почилить на Песи или Лиди, но пороги 2\sdash й категории решил оставить нам, молодёжи. Надо было ещё людей. Всевозможные поиски людей окончились тем, что нашли ещё одного человека. Таким образом, получалось, что пойдём двумя байдарками\mdash моей <<Таймень\sdash 3>>, куда в матросы набрались Паша и Руслан, и кириной <<Таймень\sdash 2>>, где в матросы определили Серёгу, как и в 2019 году. Эх, жаль, что С.Ю. не пошёл\mdash была б ещё и его двушка, и тогда бы у нас была оптимальная, как я считаю, команда и флотилия. Но да что уж поделаешь. 

\renewcommand*{\thefootnote}{\arabic{footnote}}
Поскольку <<Сунская цепочка>>\mdash всё таки 2\sdash я категория, решили прикупить спасжилеты, спасконец Александрова\footnote[1]{Средство для оказания помощи утопающим; представляет собой плавучий тонкий трос, длиной около 30 м; кидается утопающему в~специальном мешочке, обеспечивающем точность броска и последующее разматывание троса в полёте.}, фартук\footnote[2]{Средство, защищающее гребца и байдарку от брызг, осадков, волн, а~также предотвращающее заливание байдарки водой в порогах.} для байдарки и байдарочные юбки\footnote[3]{Приспособление в виде цилиндра из ткани, герметизирующее гребца, сидящего в фартуке, от волн и брызг.}. Словом, подготовиться к бурной реке. Опыт сплава по~категорийной реке был только у Кири\mdash он ходил и по Шуе и по Керети в Карелии, но$\ldots$ в детстве. Сейчас же у нас были собственные корабли и нам предстояло дооснастить их для плавания по порогам. 

Что\sdash то новое предстояло в моей жизни\mdash настоящих всамделишних порогов я ещё ни разу не проходил. То, что на Чагодоще было обозначено как пороги, на деле оказалось ничем. А тут пожалуйста\mdash и фото и видео с реки Суны полно в интернете\mdash вода там бурлит что надо! А посему поехал в пятницу после работы покупать байдарочные юбки, фартуки и т.д. по списку, а после\mdash к Кире\mdash отдать ему его часть снаряги, заламинировать топокарты ну и, ясное дело, отметить это дело в знаменитой кириной <<пивотеке>>.



%\vspace{5mm}

%\begin{flushright}
%	\textit{Соболев А.А., 2023 г.}
%	%\copyright~Соболев~А.А.,~Москва,~27.08.2015
%\end{flushright}

}