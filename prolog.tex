{

{
\cleardoublepage
\phantomsection

%\fancyhead[LE]{\fancyplain{}{}}
%\fancyhead[RO]{\fancyplain{}{}}

\addcontentsline{toc}{chapter}{Пролог}
%\addtocontents{toc}{\vspace{0mm}}
\section*{Пролог}

\fancyhead[LE]{\fancyplain{}{}}
\fancyhead[RO]{\fancyplain{}{}}

Карелия$\ldots$ что чувствует турист\sdash водник, слыша это слово? Сердце начинает стучать чаще, разум затуманивается! Карелия, Карьяла, Karjala\mdash зовёт, манит, завлекает! Чистейшие сосновые леса на каменистых берегах, а под ногами мшаник и поля черники, голубики$\ldots$ Карелия это красивая и~строгая северная природа, скалы, исполинские янтарные сосны, пьянящий воздух, обилие ягод, грибов, рыбы, и, естественно, пороги\footnote{Каменистый участок в русле с повышенной скоростью течения и~относительно большим падением отметок уровня воды.}. Куда же без них? 

Сотни, если не тысячи, водников каждый год посещают эти заповедные места, эти порожистые речки, заставляющие неслабо напрячься на порогах капитанов и~впередсмотрящих туристких судёнышек. Но этот небольшой экстрим с~лихвой окупается всем, что окружает сплавщиков на~маршруте\mdash и~дарами леса, и чувством единения с карельской природой\mdash северной, порою суровой и~жестковатой, а порою ласковой и приветливой, отчего делается так хорошо на душе, что не передать словами; уха на свежевыловленной рыбе, шум порогов, белая пена и валы которых заставляют испытать порою животный страх и чувство преклонения перед силой стихии\mdash за этим едут сюда люди, которым опостылел город, кто хочет духовно очиститься, опустошить сознание, дать природе омыть тебя бодрящей водой и смыть городскую грязь\mdash физически и~духовно. Это место\mdash место припадения к истокам, место силы, место древа жизни. Такой представлялась Карелия. Такой она и оказалась, приняв команду сплавщиков и~тёплыми дождями, и ярким августовским солнышком, а~иногда и порывистым ветром, пробирающим до костей даже в штормовке.  

Откуда есть пошла Земля Русская? Да не скажет никто\mdash сплошное надувательство\mdash дальше нескольких сотен лет назад никто не скажет наверняка что и как там было. Считается, что Старая Ладога была местом, куда был призван княжить Рюрик (Зачем, спрашивается, призван? Своих <<рюриков>> мало было?). Впрочем тут же вам скажут, что, скорее всего, этим местом должен быть Великий Новгород и пошло поехало\mdash куча теорий. Но~все они, конечно же, не на пустом месте. Принято считать, что так называемое призвание варягов на Русь таки имело место и~Рюрик всё же сначала обосновался в Старой Ладоге. Именно поэтому команда сплавщиков решила остановиться в этом городке по пути в~Карелию, поскольку интересно и~крепость посмотреть, и ехать до Карелии от Москвы, что называется, за~1 присест\mdash очень тяжело. Посему, решили ехать с ночёвкой в Старой Ладоге, отведя утро второго дня на осмотр крепости и окрестностей. 

\renewcommand*{\thefootnote}{\fnsymbol{footnote}}
%\renewcommand*{\thefootnote}{\arabic{footnote}}
\setcounter{footnote}{0}
Отчего же вдруг они отошли от~привычного бассейна Мологи\cite{СоболевВепсскаяЛетопись}? Всё~просто\mdash наиболее интересные реки мологского бассейна были пройдены\mdash Лидь, Чагода, Чагодоща, Горюн, Песь. Хотелось чего\sdash то принципиально нового! Последнний раз команда ходила в сплав в 2019 году, о~чём даже была вырезана ножом памятная надпись на~беседке у~антистапеля\footnote{применительно к байдарочным сплавам\mdash место разборки байдарок, как правило, при окончании маршрута.}: %(да, именно такими буквами): 
%Непокорённой осталась, разве что, Кобожа, но характер берегов там как\sdash то не~способствовал склонению желания в её сторону.
%\newpage

%\vspace{-5mm}
%\vspace{-4mm}
{\centering\LARGE{\fontspec{NorseRus}{%\LARGE
			Посл.\\раз\\2019\\
}}}

\newpage
И как накаркали\mdash ситуация в мире пошла, что называется, по~наклонной. Сначала ужас и сумасшествие пандемии 2020\thinspace\nobreakdash--\thinspace 2021 года, сопровождаемое чехардой смены работ, удалёнкой, тихим схождением с ума от удалёнки и всеобщей окружающей обстановки. Потом ситуация с~вооруженным конфликтом у~наших западных границ. Всё~это, конечно~же, счастья не~добавляло абсолютно. Закрадывалась мысль\mdash а может надпись <<Посл. раз 2019>> и вправду уже стала пророческой? Но нет, рановато! Кто\sdash то там лает, а караван идёт\mdash волю к~путешествиям не~заглушишь ничем!

Шурику, предводителю, командиру команды байдарочников, вдруг вспомнились ставшие для него чем\sdash то сакральным строки из произведения <<Географ глобус пропил>>: <<\textit{Конечно, обалдели все и от меня, и от такого похода. Всем домой хочется. Половина клянётся, что больше ни~в~жисть из города не вылезет. Но всё это\mdash пустые обещания. <...> Сейчас все хотят одного: тепла, уюта, покоя. Но отрава бродяжничества уже в~крови. И~никакого покоя дома они не~обретут. Снова начнёт тревожить вечное влечение дорог\mdash едва просохнет одежда и отмоется грязь из\sdash под ногтей. Я~это знаю точно. Я~и~сам сто раз зарекался\mdash больше ни ногой. И~где~я~сейчас?}>>\cite{ГеографГлобусПропил}.

Прочитав <<Географа>> перед своим первым байдарочным сплавом, Шурик был поражён этим произведением, насколько оно легло на душу, насколько пришлось близким и понятным. Однако тогда приведённые выше строки не~вспыхивали так ярким пламенем, безумным пожаром, как сейчас\mdash разве мог он тогда, в 2015 году зарекаться, что, мол, ни ногой больше? А~в~2019 вдруг с~горечью вырезает <<Посл. раз>>. И вот, в 2023 Шурик снова сидит поздним вечером у себя на кухне, разложив на~столе святая святых\mdash топокарту маршрута и волнительно прикидывает места стоянок, стратегию прохождения маршрута и штурм порогов$\ldots$ 

Одно верно\mdash отрава бродяжничества в крови и этого не отнять\mdash поход 2023 года только подтвердил это. Стоит только позволить себе мимолётную слабость\mdash на 1 секунду закрыть глаза и подумать: <<Вот щ\sdash щ\sdash щас бы на реку$\ldots$>> и~всё, ты уже испытываешь то самое сладкое и~томительное \textit{<<вечное влечение дорог>>}$\ldots$

\begin{center}
	\psvectorian[scale=0.4]{88} % Красивый вензелёк :)
\end{center}

%\vspace{5mm}

%\begin{flushright}
%	\textit{Соболев А.А., 2023 г.}
%	%\copyright~Соболев~А.А.,~Москва,~27.08.2015
%\end{flushright}

}

}