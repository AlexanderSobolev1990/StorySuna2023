\chapter*{}

%В сборник вошли повести о байдарочных сплавах, ежегодно проходивших в~период с 2015 по 2019 годы. Каждая повесть представляет собой рассказ\sdash отчёт по одному из водных походов, проведённых на реках Чагодощенского края и оставивших неизгладимые впечатления в душе автора. В~каждом рассказе переплетены воедино сведения о подготовке к сплаву, маршруте и~его прохождении, а~также личные переживания автора и~множество сопутствующих отвлечённых рассуждений вместе с~элементами автобиографии и~различными воспоминаниями.

В повести рассказывается о первом сплаве автора в~Карелии по маршруту <<Сунская цепочка>>\cite{Шилов}. Своеобразный <<кольцевой>> маршрут, замкнутый на посёлок Гирвас, состоит из двух частей: в первой путешественники побывают на череде озёр и небольших речушек, а во второй, начинающейся с Линдозера, предстоит штурм 14 порогов 2 категории сложности. Цепочка сменяющих друг друга рек и озёр приносит путешественникам массу впечатлений, приправленных переменчивой карельской погодой, а также местными ягодами, грибами и, конечно же, свежевыловленной рыбой. При заброске путешественники осмотрят крепость в Старой Ладоге, побывают на водопаде Кивач, а~при~выброске\mdash на палеовулкане Гирвас.


%\vspace{\fill}
%\begin{flushright}
%	\copyright~Соболев~А.А.~2022
%\end{flushright}
